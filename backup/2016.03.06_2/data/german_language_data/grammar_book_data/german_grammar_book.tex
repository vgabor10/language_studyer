\documentclass{article}

\usepackage[utf8]{inputenc}
\usepackage{amsthm}
\usepackage{comment}

\setlength\parindent{0pt}

\theoremstyle{definition}
\newtheorem*{exmp}{Példák}
\newtheorem*{desc}{Leírás}

\title{Német nyelvtan}
\author{Varga Gábor}

\begin{document}

\maketitle

\tableofcontents

\section{Határozott névelő esetei}

GrammarItemIndex = 0

\begin{desc}
\begin{tabular}{ccccc}
eset & hímnem & nőnem & semleges nem & többes szám\\
alany & der & die & das & die\\
tárgy & den & die & das & die\\
részes & dem & der & dem & den -n\\
birtokos & des -s & der & des -s & der\\
\end{tabular}
\end{desc}

\begin{exmp}
1 | der | hím nem, alany eset\\
2 | die | nő nem, alany eset\\
3 | das | semleges nem, alany eset\\
4 | die | többes szám, alany eset\\
5 | den | hím nem, tárgy eset\\
6 | die | nő nem, tárgy eset\\
7 | das | semleges nem, tárgy eset\\
8 | die | többes szám, tárgy eset\\
9 | dem | hím nem, részes eset\\
10 | der | nő nem, részes eset\\
11 | dem | semleges nem, részes eset\\
12 | den -n | többes szám, részes eset\\
13 | des -s | hím nem, birtokos eset\\
14 | der | nő nem, birtokos eset\\
15 | des -s | semleges nem, birtokos eset\\
16 | der | többes szám, birtokos eset\\
\end{exmp}

\subsection{Tárgy eset és részes eset - példák}

GrammarItemIndex = 30

%Maklári 27. oldal
\begin{exmp}
1 | den Vater | az apát\\
2 | die Tische | az asztalokat\\
3 | den Kind | a gyereket\\
4 | den Kindern | a gyerekeknek\\
5 | der Oma | a nagymamának\\
6 | dem Opa | a nagypapának\\
7 | die Tische | az asztalok\\
8 | das Auto | az autót\\
9 | den Stuhl | a széket\\
10 | den Schrank | a szekrényt\\
11 | den Tisch | az asztalt\\
12 | der Mutter | az anyának\\
13 | den Müttern | az anyáknak\\
14 | dem Kind | a gyereknek\\
15 | die Lampe | a lámpát\\
16 | dem Vater | az apának\\
17 | den Vätern | az apáknak\\
18 | das Haus | a házat\\
19 | den Hof | az udvart\\
\end{exmp}

\subsection{Birtokos eset}

GrammarItemIndex = 3214

\begin{desc}
\begin{tabular}{cccc}
hímnem & nőnem & semleges nem & többes szám\\
des ~s & der & des ~s & der\\
\end{tabular}
\end{desc}

\begin{exmp}
1 | der Sohn des Vaters | az apa fia\\
2 | das Fenster des Hauses | a ház ablaka\\
3 | die Tochter der Mutter | az anya lánya\\
4 | der Freund des Vaters | az apa barátja\\
5 | das Spielzeug des Kindes | a gyerek játéka\\
6 | die Scheibe des Wagens | a kocsi ablaka\\
7 | die Kleider der Kinder | a gyerekek ruhái\\
8 | der Bus der Lehrer | a tanárok busza\\
9 | die Freundin der Mutter | az anya barátnője\\
\end{exmp}

\section{Határozatlan névelő esetei}

GrammarItemIndex = 1

\begin{desc}
\begin{tabular}{ccccc}
 & hímnem & nőnem & semleges nem & többes szám \\
 Nominativ & ein & eine & ein & keine \\
 Akkusativ & einen & eine & ein & keine \\
 Dativ & einem & einer & einem & keinen -n \\
 Genitiv & eines -s & einer & eines -s & keiner \\
\end{tabular}
\end{desc}

\begin{exmp}
1 | ein | hím nem, alany eset\\
2 | eine | nő nem, alany eset\\
3 | ein | semleges nem, alany eset\\
4 | einen | hím nem, tárgy eset\\
5 | eine | nő nem, tárgy eset\\
6 | ein | semleges nem, tárgy eset\\
7 | einem | hím nem, részes eset\\
8 | einer | nő nem, részes eset\\
9 | einem | semleges nem, részes eset\\
10 | eines -s | hím nem, birtokos eset\\
11 | einer | nő nem, birtokos eset\\
12 | eines -s | semleges nem, birtokos eset\\
13 | keine | többes szám, alany eset\\
14 | keine | többes szám, tárgy eset\\
15 | keine -n | többes szám, részes eset\\
16 | keiner | többes szám, birtokos eset\\
\end{exmp}

\subsection{Határozatlan névelő esetei - Példák}

GrammarItemIndex = 31

%Maklári: 37. oldal
\begin{exmp}
1 | ein Mädchen | egy lány\\
2 | ein Kind | egy gyerek\\
3 | ein Vater | egy apa\\
4 | einen Wagen | egy kocsit\\
5 | einer Frau | egy nőnek (részére)\\
6 | einen Garten | egy kertet\\
7 | einem Kind | egy gyereknek\\
8 | ein Auto | egy autót\\
9 | ein Buch | egy könyvet\\
10 | ein Mann | egy férfi\\
11 | ein Haus | egy ház\\
12 | einen Schrank | egy szekrényt\\
13 | einem Mädchen | egy lánynak\\
14 | einem Mann | egy férfinak\\
15 | eine Frau | egy nő\\
16 | einer Lehrerin | egy tanárnőnek\\
17 | eine Lampe | egy lámpa\\
18 | einen Stuhl | egy széket\\
19 | einem Gast | egy vendégnek\\
20 | ein Fenster | egy ablakot\\
21 | einen Tisch | egy asztalt\\
22 | ein Junge | egy fiú\\
23 | ein Haus | egy házat\\
24 | ein Fenster | egy ablak\\
25 | einem Hund | egy kutyának\\
26 | ein Auto | egy autó\\
27 | einen Lehrer | egy tanárt\\
28 | ein Buch | egy könyv\\
29 | ein Mann | egy ember\\
30 | einem Lehrer | egy tanárnak\\
31 | ein Mädchen | egy lányt\\
32 | einer Mutter | egy anyának\\
33 | die Feder eines Vogels | egy madár tolla\\
34 | das Heft eines Schülers | egy diák füzete\\
35 | das Fenster eines Hauses | egy ház ablaka\\
36 | der Rock einer Frau | egy nő szoknyája\\
37 | die Frau eines Lehrers | egy tanár felesége\\
38 | das Heft einer Schülerin | egy diáklány füzete\\
39 | die Seiten eines Buches | egy könyv oldalai\\
40 | der Kuli einer Lehrerin | egy tanárnő tolla\\
41 | der Ball eines Kindes | egy gyerek labdája\\
\end{exmp}

\section{Személyes névmás esetei}

GrammarItemIndex = 2

\begin{desc}
\begin{tabular}{cccc}
 nominativ/alany & akusativ/tárgy & dativ/részes & genitiv/birtokos \\
 ich & mich & mir & mein \\
 du & dich & dir & dein \\
 er & ihn & ihm & sein \\
 sie & sie & ihn & ihr \\
 es & es & ihm & sein \\
 wir & uns & uns & unser \\
 ihr & euch & euch & euer \\
 sie & sie & ihnen & ihr \\
 Sie & Sie & Ihnen & Ihr \\
\end{tabular}
\end{desc}

\begin{exmp}
1 | mich | engem\\
2 | dich | téged\\
3 | ihn | őt (himnem)\\
4 | sie | őt (nőnem)\\
6 | es | azt\\
7 | uns | minket\\
8 | euch | titeket\\
9 | sie | őket\\
10 | Sie | Önt, Önöket\\
11 | mir | nekem\\
12 | dir | neked\\
13 | ihm | neki (hímnem)\\
14 | ihr | neki (nőnem)\\
15 | ihm | neki (semleges nem)\\
16 | uns | nekünk\\
17 | euch | nektek\\
18 | ihnen | nekik\\
19 | Ihnen | Önnek, Önöknek\\
20 | mein | enyém\\
21 | dein | tiéd\\
22 | sein | övé (hímnem)\\
23 | ihr | övé (nőnem)\\
24 | sein | övé (semleges nem)\\
25 | unser | miénk\\
26 | euer | tiétek\\
27 | ihr | övék\\
28 | Ihr | Öné, Önöké\\
\end{exmp}

\subsection{Személyes névmás tárgyesete}

GrammarItemIndex = 20

\begin{desc}
\begin{tabular}{cccc}
 nominativ/alany & akusativ/tárgy \\
 ich & mich \\
 du & dich \\
 er & ihn \\
 sie & sie \\
 es & es \\
 wir & uns \\
 ihr & euch \\
 sie & sie \\
 Sie & Sie \\
\end{tabular} 
\end{desc}

%Maklári: 38. oldal
\begin{exmp}
1 | Wir hören euch. | Hallgatunk titeket.\\
2 | István siecht sie. | István nézi őt (nő).\\
3 | Ich höre sie. | Hallom őket.\\
4 | Sie suchen uns. | Keresnek minket.\\
5 | Helga liebt ihn. | Helga szereti őt (ffi).\\
6 | Mich ruft Peter. | Engem hív Peter.\\
7 | Wir fragen euch. | Titeket kérdezünk.\\
8 | Der Lehrer frägt mich. | A tanár engem kérdez.\\
9 | Ich bitte Sie. | Kérem Önt.\\
10 | Wir bitten Sie | Kérjük Önöket.\\
11 | Hört ihr sie? | Halljátok őket?\\
12 | Mutter liebt ihn. | Anya szereti őt (ffi).\\
\end{exmp}

\subsection{Személyes névmás részes esete}

GrammarItemIndex = 155

\begin{desc}
\begin{tabular}{cccc}
 nominativ/alany & dativ/részes \\
 ich & mir \\
 du & dir \\
 er & ihm \\
 sie & ihn \\
 es & ihr \\
 wir & uns \\
 ihr & euch \\
 sie & ihnen \\
 Sie & Ihnen \\
\end{tabular}
\end{desc}

%Maklári: 40. oldal
\begin{exmp}
1 | Er kauft uns etwas. | Vásárol (f) nekünk valamit.\\
2 | Ich gebe dir eine Ohrfeige. | Adok neked egy pofont.\\
3 | Johann und Wolfgang helpen ihnen. | Johann és Wolfgang segítenek nekik.\\
4 | Ich gebe euch Zeit. | Adok nektek időt.\\
5 | Wir kaufen Ihnen das Radio. | Megvesszük önnek a rádiót.\\
6 | Pista schenken ihr einen Rock. | Pista ajándékoz neki egy szoknyát.\\
7 | Das Mädchen gefällt mir. | Tetszik nekem a lány.\\
8 | Gefällst das Spielzeug dir? | Tetszik neked a játék?\\
9 | Gefallen Ihnen das Auto? | Tetszik Önnek az autó?\\
10 | Wir gratulieren ihnen. | Gratulálunk nekik.\\
11 | Ich gebe dir der Wagen. | Odaadom neked a kocsit.\\
12 | Ihr schenkt ihnen eine Torte. | Ajándékozok nekik egy tortát.\\
13 | Ich gebe ihr ein Küssen. | Adok neki (n) egy puszit.\\
14 | Anna hilft ihm Auto reparieren. | Anna segít neki (f) autót szerelni.\\
15 | Ich schenke ihr eine Blus. | Ajándékozok neki (n) egy blúzt.\\
16 | Wir geben euch Geld. | Adunk pénzt nektek.\\
17 | Wir gratulieren Ihnen. | Gratulálunk Önöknek.\\
18 | Sie gibt mir eine Uhr. | Ad (n) nekem egy órát.\\
19 | Er gibt uns das Buch. | Nekünk adja a könyvet.\\
20 | Öffnest du mir die Tür? | Kinyitod nekem az ajtót?\\
21 | Ich erzähle dir die Geschichte. | Elmesélem neked a történetet.\\
\end{exmp}

\subsection{Személyes névmás birtokos esete}

GrammarItemIndex = 18

\begin{desc}
\begin{tabular}{cccc}
 nominativ/alany & dativ/részes \\
 ich & mein\\
 du & dein\\
 er & sein\\
 sie & ihr\\
 es & sein\\
 wir & unser\\
 ihr & euer\\
 sie & ihr\\
 Sie & Ihr\\
\end{tabular}

Ha a birtok nőnemű vagy többes számú, akkor a birtokos névmáshoz
egy -e betűt illesztünk.
\end{desc}

%Maklári: 42. oldal
\begin{exmp}
1 | ihre Tochter | a lánya (nőé)\\
2 | seine Tochter | a lánya (férfié)\\
3 | ihr Sohn | a fia (nőé)\\
4 | sein Sohn | a fia (férfié)\\
5 | unsere Schwester | a testvérünk (lány)\\
6 | euer Freund | barátotok\\
7 | euere Freundin | barátnőtök\\
8 | mein Wagen | a kocsim\\
9 | unsere Freunden | barátaink\\
10 | Ihr Haus | az Ön háza\\
11 | euere Kleider | ruháitok\\
12 | ihre Kleider | ruháik\\
13 | unser Vater | apánk\\
14 | euere Geschwister | testvéreitek\\
15 | ihre Freundin | barátnője (n)\\
16 | meine Freundin | barátnőm\\
17 | dein Freund | barátod\\
18 | Ihre Mutter | az Ön anyja\\
19 | Ihr Vater | az Ön apja\\
20 | unser Kind | gyerekünk\\
21 | euer Stuhl | széketek\\
22 | euere Stühle | székeitek\\
23 | ihr Tisch | asztaluk\\
24 | ihre Tische | asztalaik\\
25 | ihr Bruder | fiútestvére (n)\\
26 | seine Schwester | lánytestvére (f)\\
27 | Ihr Tisch | az Ön asztala\\
28 | seine Tasche | táskája (f)\\
29 | ihr Auto | autója (n)\\
30 | Ihr Haus | Önök háza\\
31 | euere Bücher | könyveitek\\
32 | euer Buch | könyvetek\\
33 | ihr Buch | könyvük\\
\end{exmp}

\section{Birtokos névmás}

\subsection{Birtokos névmás ragozása}

GrammarItemIndex = 40

\begin{desc}
\begin{tabular}{ccccc}
 & hímnem & nőnem & semleges nem & többes szám \\
nominativ & mein & meine & mein & meine \\
akusativ & meinen & meine & mein & meine \\
dativ & meinem & meiner & meinem & meinen -s \\
genitiv & meines -s & meiner & meines ~s & meiner \\
\end{tabular}
\end{desc}

%Maklári: 45. oldal
\begin{exmp}
1 | meinen Vater | az apámat\\
2 | deiner Mutter | az anyágnak\\
3 | meiner Oma | a nagymamámnak\\
4 | eueren Tisch | az asztalotokat\\
5 | unseren Freundinnenen | a barátnőinknek\\
6 | unserer Freundin | a barátnőnknek\\
7 | Ihren Anzug | az Ön öltönyét\\
8 | meine Frau | a feleségemet\\
9 | mein Bruder | a fivérem\\
10 | meinem Bruder | a fivéremnek\\
11 | Ihre Frau | az Ön feleségét\\
12 | deinen Hund | a kutyádat\\
13 | euer Auto | au autótokat\\
14 | sein Haus | a házát (f)\\
15 | seiner Frau | a feleségének\\
16 | unseren Kindernen | a gyerekeinknek\\
17 | euerem Kind | a gyereketeknek\\
18 | unsere Lampe | a lámpánkat\\
19 | deinem Hund | a kutyádnak\\
20 | seinen Freund | a barátodat (f)\\
21 | unseren Elternen | a szüleinknek\\
22 | euere Kinder | a gyerekeiteket\\
23 | Ihrem Mann | az Önök férjének\\
24 | meinen Freun | a barátomat\\
\end{exmp}

\subsection{Birtokos névmás birtokos esete}

GrammarItemIndex = 50

%Maklári: 45. oldal
\begin{exmp}
1 | der Kind unserer Eltern | a szüleink gyereke\\
2 | der Sohn meines Vaters | az apám fia\\
3 | die Tochter meiner Mutter | az anyám lánya\\
4 | die Tür unseres Hauses | a házunk ajtaja\\
5 | die Lampe seines Autos | az autójuk lámpája\\
6 | die Scheibe Ihres Wagens | az Ön kocsijának az ablaka\\
7 | die Bälle euerer Kinder | a gyerekeitek labdái\\
8 | die Shöne euerer Tochter | a lányaitok fiai\\
9 | der Freund ihres Mannes | a férjének a barátja\\
10 | das Rad deines Fahrrades | a kerékpárod kereke\\
11 | das Geld euerer Freunde | a barátaitok pénze\\
12 | das Zeit unseres Lehrers | a tanárunk ideje\\
13 | das Kleid seiner Freundin | a barátnőjének (f) a ruhája\\
14 | die Fenster ihrer Häuser | a házaiknak az ablakai\\
15 | die Sohle meines Schuches | a ipőm talpa\\
16 | das Ende unserer Aufgaben | a feladataink vége\\
17 | der Beginn unserer Leidens | a szenvedéseink kezdete\\
18 | die Freundin seiner Frau | a feleségének a barátnője\\
\end{exmp}

\subsection{Birtokos névmás önálló alakja}

GrammarItemIndex = 4234

\section{Módbeli segédigék}

\subsection{Módbeli segédigék ragozása}

GrammarItemIndex = 3

\begin{desc}
\begin{tabular}{lllllll}
E/1 & kann & darf & mag & will & muss & soll \\
E/2 & kannst & darfst & magst & willst & musst & sollst \\
E/3 & kann & darf & mag & will & muss & soll \\
T/1 & können & dürfen & mögen & wollen & müssen & sollen \\
T/2 & könnt & dürft & mögt & wollt & müsst & sollt \\
T/3 & können & dürfen & mögen & wollen & müssen & sollen \\
\end{tabular}
\end{desc}

\begin{exmp}
1 | kann | tud, E/1\\
2 | kannst | tud, E/2\\
3 | kann | tud, E/3\\
4 | können | tud, T/1\\
5 | könnt | tud, T/2\\
6 | können | tud, T/3\\
7 | darf | szabad, E/1\\
8 | darfst | szabad, E/2\\
9 | darf | szabad, E/3\\
10 | dürfen | szabad, T/1\\
11 | dürft | szabad, T/2\\
12 | dürfen | szabad, T/3\\
13 | mag | szeret, E/1\\
14 | magst | szeret, E/2\\
15 | mag | szeret, E/3\\
16 | mögen | szeret, T/1\\
17 | mögt | szeret, T/2\\
18 | mögen | szeret, T/3\\
19 | will | akar, E/1\\
20 | willst | akar, E/2\\
21 | will | akar, E/3\\
22 | wollen | akar, T/1\\
23 | wollt | akar, T/2\\
24 | wollen | akar, T/3\\
25 | muss | kell (külső kényszer), E/1\\
26 | musst | kell (külső kényszer), E/2\\
27 | muss | kell (külső kényszer), E/3\\
28 | müssen | kell (külső kényszer), T/1\\
29 | müsst | kell (külső kényszer), T/2\\
30 | müssen | kell (külső kényszer), T/3\\
31 | soll | kell (belső kényszer), E/1\\
32 | sollst | kell (belső kényszer), E/2\\
33 | soll | kell (belső kényszer), E/3\\
34 | sollen | kell (belső kényszer), T/1\\
35 | sollt | kell (belső kényszer), T/2\\
36 | sollen | kell (belső kényszer), T/3\\
\end{exmp}

\subsection{Módbeli segédigék használata}

GrammarItemIndex = 4

\begin{desc}
Módbeli segédigék használata:

alany + módbeli segédige ragozva+ többi mondatrész + ige főnévi igenév alakban 

\begin{itemize}
\item können | tud
\item dürfen | szabad
\item mögen | szeret, kedvel
\item müssen | kell
\item sollen | kell
\item wollen | akar
\end{itemize}
\end{desc}

\begin{exmp}
1 | Ich will heute ins Kino gehen. | Ma moziba akarok menni.\\
2 | Ilona darf ins Theater gehen. | Ilonának szabad színházba mennie.\\
3 | Ich kann schnell Bier trinken. | Gyorsan tudok sört inni.\\
4 | Ich kann schwimmen. | Tudok úszni.\\
5 | Sie will kommen. | Akar jönni. (nő)\\
6 | Wir mögen Fußball spielen. | Szeretünk focizni.\\
7 | Er muss nach Hause gehen. | Haza kell mennie. (férfi)\\
8 | Du musst Blume kaufen. | Virágot kell venned.\\
9 | Wir dürfen spielen. | Szabad játszanunk.\\
10 | Wir wollen Fernsehen. | Akarunk tévézni.\\
11 | Wollt ihr ins Bett gehen? | Akartok ágyba menni?\\
12 | Er kann gut sprechen. | Jól tud beszélni. (férfi)\\
13 | Ihr dürft nach Hause gehen. | Haza mehettek.\\
14 | Mögt ihr Korbball spielen? | Szerettek kosárlabdázni?\\
15 | Er soll nett sind. | Kedvesnek kell lennie. (férfi)\\
16 | Mögt ihr Eis essen? | Szerettek fagyit enni?\\
17 | Ich muss lernen. | Tanulnom kell.\\
18 | Sie muss einkaufen. | Be kell vásárolnia (nő).\\
19 | Wollt iht trinken? | Akartok inni?\\
20 | Sie mag Schach spielen. | Szeret sakkozni.\\
21 | Sie wolen nach Hause gehen. | Haza akarnak menni.\\
22 | Darfst du rauchen? | Szabad cigarettáznod?\\
23 | Sie wollen schlafen. | Aludni akarnak.\\
24 | Sie müssen lernen. | Tanulniuk kell.\\
25 | Gizi will Wasser trinken. | Gizi vizet akar inni.\\
26 | Ihr soll höflich sind. | Udvariasnak kell lennetek.\\
\end{exmp}

\subsection{A möchten módbeli segédige}

GrammarItemIndex = 17

\begin{desc}
A möchten módbeli segédige a mögen (szeretni) ige feltételes módú
alakja (tehát: szeretnék, szeretnél...).

Pl.: Ich möchte heute ins Theater gehen. | Szeretnék me színházba menni.

\begin{tabular}{lllllll}
E/1 & möchte \\
E/2 & möchtest \\
E/3 & möchte \\
T/1 & möchten \\
T/2 & möchtet \\
T/3 & möchten \\
\end{tabular}
\end{desc}

%Maklári: 123. oldal
\begin{exmp}
1 | Möchtet ihr laufen? | Szeretnétek futni?\\
2 | Er möchte Fußball spielen. | Szeretne (f) focizni.\\
3 | Ich möchte schon schlafen. | Szeretnék már aludni.\\
4 | Sie möchten einen Haus bekommen. | Szeretnének egy Házat kapni.\\
5 | Möchten Sie zu Hause gehen? | Szeretne Ön haza menni?\\
6 | Möchtest du ein Glas Wasser trinken? | Szeretnél egy pohár vizet inni?\\
7 | Möchtet ihr Abend essen? | Szeretnétek vacsorázni?\\
8 | Ich möchte Fahrrad fahren. | Szeretnék vacsorázni.\\
9 | Ich möchte etwas essen. | Szeretnék valamit enni.\\
10 | Ich möchte einen Fahrrad haben. | Szeretnék egy kerékpárt.\\
11 | Ich möchte nicht in Schule gehen. | Nem szeretnék iskolába menni.\\
12 | Möchten Sie hier schlafen? | Szeretne (Ön) itt aludni?\\
13 | Möchtet ihr eine Tasse Tee trinken? | Szeretnétek egy csésze teát inni?\\
14 | 14 | Sie möchten morgen spazieren. || Szeretnének holnap sétálni.\\
15 | Möchtest du in Deutschland reisen? | Szeretnél Németországba utazni?\\
16 | Möchtet ihr Mittag essen? | Szeretnétek ebédelni?\\
17 | Ich möchte nach Hause gehen. | Haza szeretnék menni.\\
18 | Ich möchte sie sehen. | Szeretném látni őt (n).\\
\end{exmp}

\subsection{A módbeli segédigék másodlagos jelentése}

\subsubsection{A valószínűség fokozatai}

GrammarItemIndex = 5433

\begin{desc}
\begin{enumerate}
\item müssen - egész bitzos, kétségtelen
\item dürfen - valószínűleg
\item können - lehet lehetséges
\item mögen - talán
\end{enumerate}
\end{desc}

\begin{exmp}
1 | Er muss zu hause sein. | Egész biztos, hogy otthon van.\\
2 | Er darf zu Hause sein. | Valószínűleg otthon van.\\
3 | Er kann zu Hause sein. | Lehet, hogy otthon van.\\
4 | Er mag zu Hause sein. | Talán otthon van.\\
\end{exmp}

\subsubsection{Állítólagosság}

GrammarItemIndex = 4235

\begin{desc}
\begin{itemize}
\item sollen | állítólag, szóbeszéd szerint
\item wollen | azt állítja, hogy; állítása szerint
\end{itemize}
\end{desc}

\begin{exmp}
1 | In Peter soll das ganze Dorf verliebt sein. | Péterbe állítólag az egész falu szerelmes.\\
2 | In Peter will das ganze Dorf verliebt sein. | Péter azt állítja, hogy az egész falu belé szerelmes.\\
\end{exmp}

\section{Prepozíciók}

\subsection{Hol? kérdésre felelő prepozíciók}

GrammarItemIndex = 16

\begin{desc}
A hol? kérdésre felelő prepozíciók részes esettel állnak.

\begin{itemize}
\item auf + D | -on, -en, -ön (vízszintes felületen)
\item an + D | -on, -en, -ön (függőleges felületen)
\item in + D | -ban, -ben
\item vor + D | előtt
\item hinter + D | mögött
\item unter + D | alatt
\item über + D | felett
\item neben + D | mellett
\item zwischen + D | között
\end{itemize}
\end{desc}

\begin{exmp}
1 | auf dem Stuhl | a széken\\
2 | an der Wand | a falon\\
3 | unter dem Tisch | az asztal alatt\\
4 | neben der Vase | a váza mellett\\
5 | über der Frau | a nő felett\\
6 | vor der Hauses | a házak előtt\\
7 | zwischen der Lampen | a lámpák között\\
8 | unter dem Baum | a fa alatt\\
9 | an der Bilden | a képeken\\
\end{exmp}

\subsubsection{Prepozíciók összevonása}

GrammarItemIndex = 5432

\begin{desc}
\begin{itemize}
\item an dem = am
\item in dem = im
\item vor dem = vorm (beszédben)
\item hinter dem = hinterm (beszédben)
\item unter dem = unterm (beszédben)
\item über dem = überm (beszédben)
\end{itemize}
\end{desc}

\subsection{Hová? kérdésre felelő prepozíciók}

GrammarItemIndex = 4232

\begin{desc}
A hová? kérdésre felelő prepozíciók tárgy esettel állnak.

\begin{itemize}
\item auf + A = -ra, -re (vízszintes felületre)
\item an + A = -ra, -re (függőleges felületre)
\item in + A = -ba, -be
\item vor + A = elé
\item hinter + A = mögé
\item unter + A = alá
\item über + A = fölé
\item neben + A = mellé
\item zwischen + A = közé
\end{itemize}
\end{desc}

\begin{exmp}
1 | an die Wand | a falra\\
2 | in die Küche | a konyhába\\
3 | auf den Schrank | a szekrényre\\
4 | neben des Tisch | az sztal mellé\\
5 | vor das Zimmer | a szoba elé\\
6 | in den Schrank | a szekrénybe\\
7 | auf das Tisch | az asztalra\\
8 | hinter die Wand | a fal mögé\\
\end{exmp}

\subsubsection{Prepozíciók összevonása}

GrammarItemIndex = 1236

\begin{desc}
\begin{itemize}
\item auf das = aufs
\item an das = ans
\item in das = ins
\item vor dem = vors (beszédben)
\item hinter das = hinters (beszédben)
\item unter das = unters (beszédben)
\item über das = übers (beszédben)
\end{itemize}
\end{desc}

\subsection{Prepozíciók kizárólag tárgy esettel}

GrammarItemIndex = 5

\begin{desc}

\begin{itemize}
\item für - -ért, részére, számára
\item ohne - nélkül
\item gegen - ellen, körül (időben)
\item bis - -ig
\item durch - át, keresztül, által
\item um - -kor, körül (térben)
\end{itemize}

\end{desc}

\begin{exmp}
1 | für den Vater | az apa részére\\
2 | für die Mutter | az anya részére\\
3 | ohne Geld | pénz nélkül\\
4 | ohne Tasche | táska nélkül\\
5 | gegen den Freund | a barát ellen\\
6 | gegen den Feind | az ellenséggel szemben\\
7 | um das Haus | a ház körül\\
8 | durch meine Schwester | a nővérem által\\
9 | durch den Tunnel | keresztül az alagúton\\
10 | durch die Eltern | a szülőkön keresztül\\
11 | ohne Schuhe | cipő nélkül\\
12 | bis Budapest | Budapestig\\
13 | bis 5 Uhr | 5 óráig\\
14 | um den Garten | a kert körül\\
15 | für das Mädchen | a lánynak\\
16 | für fünf Forint | öt forintért\\
17 | für einen Wagen | egy kocsiért\\
18 | ohne Lampe | lámpa nélkül\\
19 | um die Stadt | a város körül\\
20 | ohne Schweigermutter | anyós nélkül\\
21 | durch einen Freund | egy barát által\\
22 | um einen Tisch | egy asztal körül\\
23 | durch ein Haus | egy házon át\\
24 | durch meinen Vater | apám által\\
25 | ohne Heft | füzet nélkül\\
26 | für einen Apfel | egy almáért\\
27 | durch das Dorf | keresztül a falun\\
28 | durch den Park | keresztül a parkon\\
29 | bis zwei | kettőig\\
30 | um meinen Mann | a férjem körül\\
31 | gegen die Eltern | a szülőkkel szemben\\
32 | um einen Garten | egy kert körül\\
33 | durch den Weg | keresztül az úton\\
34 | gegen mich | ellenem\\
35 | für ihn | érte (hímnem)\\
36 | ohne uns | nélkülünk\\
37 | ohne sie | nélküle (nőnem)\\
38 | um euch | körülöttetek\\
39 | für uns | értünk\\
40 | für Sie | Önért\\
41 | um uns | körülöttünk\\
42 | durch mich | általam\\
43 | für sie | részükre\\
44 | gegen dich | ellened\\
45 | um uns | körülöttünk\\
46 | ohne sie | nélkülük\\
47 | gegen Sie | Ön ellen\\
48 | ohne dich | nélküled\\
49 | für uns | nekünk\\
50 | für euch | értetek\\
51 | durch Sie | Ön által\\
52 | für dich | érted\\
53 | ohne mich | nélkülem\\
54 | um mich | körülöttem\\
55 | für sie | számukra\\
56 | gegen euch | ellenetek\\
57 | gegen ihn | ellene (hímnem)\\
58 | um dich | körülötted\\
59 | um sie | körülötte (nőnem)\\
60 | durch sie | általa (nőnem)\\
61 | gegen uns | ellenünk\\
62 | für mich | értem\\
63 | für euch | részetekre\\
\end{exmp}

\subsection{Prepozíciók kizárólag részes esettel}

GrammarItemIndex = 13

\begin{desc}
\begin{enumerate}
\item aus = -ból, -ből
\item bei = -nál, -nél
\item mit = -val, -vel
\item nach = untán; -ba, -be (ország, város)
\item von = -tól, -től, -ról, -ről
\item zu = -hoz, -hez, -höz
\item seit = óta
\item gegenüber = szemben, átellenben
\end{enumerate}
\end{desc}

\begin{exmp}
1 | aus der Küche | a konyhából\\
2 | bei der Großmutter | a nagymamánál\\
3 | mit dem Kuli | a tollal\\
4 | nach der Früchstück | reggeli után\\
5 | nach Budapest | Budapestre\\
6 | nach Vien | Bécsbe\\
7 | zu dem Lehrer | a tanárhoz\\
8 | von dem Schrank | a szekrénytől\\
9 | zu der Oma | a nagyihoz\\
10 | mit dem Bus | a busszal\\
11 | gegenüber der Tür | az ajtóval szeben\\
12 | von dem Gast | a vendégtől\\
\end{exmp}

\subsection{Prepozíciók kizárólag birtokos esettel}

GrammarItemIndex = 15

\begin{desc}
\begin{enumerate}
\item statt = helyett
\item trotz = ellenére
\item wegen = miatt
\item innerhalb = vmin belül
\item außerhalb = vmin kívül
\item während = alatt (időben)
\end{enumerate}
\end{desc}

\begin{exmp}
1 | wegen des Wetters | az időjárás miatt\\
2 | wegen des Kriges | a háború miatt\\
3 | trotz der Kälte | a hideg ellenére\\
4 | außerhalb der Stadt | a városon kívül\\
5 | wegen des Verkaufers | az eladó miatt\\
6 | während der Deutschstunde | a németóra alatt\\
7 | statt des Mittagessens | az ebéd helyett\\
8 | während des Urlaubes | a nyaralás alatt\\
9 | während des Esses | az étkezés alatt\\
10 | statt der Wanderung | a kirándulás helyett\\
\end{exmp}

\subsection{auf és an közti különbség}

GrammarItemIndex = 2463

\subsection{zwischen és an unter közti különbség}

GrammarItemIndex = 7863

\section{Birtokos szerkezet von + D-val}

GrammarItemIndex = 117

\begin{desc}
der Sohn von dem Vater | az apa fia
\end{desc}

%Maklári: 30. oldal
\begin{exmp}
1 | die Tochter von der Mutter | az anya lánya\\
2 | der Schuch von dem Vater | az apa cipője\\
3 | die Brille von den Kindern | a gyerekek szemüvege\\
4 | die Lampe von dem Einbrecher | a betörő lámpája\\
5 | das Zeugnis von dem Schüler | a tanuló bizonyítványa\\
6 | die Prothese von der Großmutter | a nagymama protézise\\
7 | der Kuli von dem Schüler | a tanuló tolla\\
8 | die Spielzeuge den Kindern | a gyerekek játékai\\
9 | das Gewissen dem Kontrolleur | az ellenőr lelkiismerete\\
10 | der Humer dem Kontrolleur | az ellenőr humora\\
\end{exmp}

\section{Kötőszavak}

\subsection{Kötőszavak egyenes szórenddel}

GrammarItemIndex = 53

\begin{desc}
A következő kötöszavak után van egyenes szórend (usoda):
\begin{itemize}
\item und - és
\item sondern - hanem
\item oder - vagy
\item denn - mert
\item aber  de
\end{itemize}
\end{desc}

%Maklári: 144. oldal
\begin{exmp}
1 | Wir kaufen und ihr kochen. | Mi vásárolunk és ti főztök.\\
2 | Ich bin nervös, denn mein Freund kommt nicht. | Ideges vagyok, mert nem jön a barátom.\\
3 | Gehe ich, oder kommst du? | Én megyek, vagy te jössz?\\
4 | Er ist schlecht, aber du bekommst die Ohrfeige. | Ö (f) rossz, de te lapod a pofont.\\
5 | Ich bleibe zu Hause, denn ich bin krank. | Otthon maradok, mert beteg vagyok.\\
6 | Heute ich spüle und du kaufst ver. | Ma én mosogatok és te vásárolsz be.\\
7 | Ich bin müde, aber ich gehe. | Fáradt vagyok, de megyek.\\
8 | Peter esst die Suppe nicht, sondern er trinkt. | Peter nem eszi a levest, hanem issza.\\
9 | Wir besuchen sie nicht, denn sie sind nich zu Hause. | Nem látogatjuk, meg őket, mert nincsenek otthon.\\
10 | Wir reisen nach Italien, aber wir bleiben nur zwei Tage. | Olaszországba utazunk, de csak két napot maradunk.\\
11 | Jetzt wir bleiben hier nicht, sondern wir gehen zu Hause. | Most nem maradunk itt, hanem haza megyünk.\\
12 | Die Gäste steigen in den Bus ein und der Bus fährt ab. | A vendégek beszállnak a buszba és a busz elindul.\\
13 | Ich arbeite und sie (n) hört Musik. | Én dolgozom és ő zenét hallgat.\\
\end{exmp}

\subsection{Alany elhagyása und kötőszó esetén}

GrammarItemIndex = 6452

\begin{desc}
Ha mindkét tagmondatban ugyanaz az alany, akkor nem kell az und után újra kiírni.
\end{desc}

\begin{exmp}
1 | Ich bleibe hier und schreibe einen Brief. | Itt maradok és írok egy levelet.\\
2 | Ich stehe auf und gehe ins Geschäft. | Felkelek és boltba megyek.\\
3 | Klaus geht ins Bett und lest ein Buch. | Klaus ágyba megy és olvas egy könyvet.\\
\end{exmp}

\subsection{Kötőszavak fordított szórenddel}

GrammarItemIndex = 11

\begin{desc}

kötőszó + ige + alany + többi mondatrész

A következő kötöszavak után van fordított szórend:
\begin{itemize}
\item trotzdem | mégis
\item dann | aztán, akkor
\item sonst | különben
\item darum | ezért
\item deshalb | ezért
\end{itemize}

Pl.: Sie ist krank, darum sucht sie einen Arzt. | Beteg, ezért keres egy orvost.
\end{desc}

%Maklári: 147. oldal
\begin{exmp}
1 | Zuerst gehe ich zu Hause, dann esse ich. | Először haza megyek, aztán eszek.\\
2 | Ich esse das Mittagessen, dann spazieren wir. | Megeszem az ebédet, aztán sétálunk.\\
3 | Das Wetter ist schön, deshalb bleiben wir nicht Zu Hause. | Az idő szép, ezért nem maradhatunk otthon.\\
4 | Du schreibst die Lektion, sonst bekommst du eine Eins! | Megírod a leckét, különben kapsz egy egyest!\\
5 | Wir lernen ein klein, dann spielen wir Fußball. | Tanulunk egy kicsit, aztán focizunk.\\
6 | Wir arbeiten, dann gehen wir in Knepie. | Dolgozunk, aztán megyünk a kocsmába.\\
7 | Er arbeitet viel, darum ist er reich. | Sokat dolgozik, ezért gazdag.\\
8 | Er arbeitet viel, trotzdem ist er arm. | Sokat dolgozik, mégis szegény.\\
9 | Peter ist krank, trotzdem kommt er mit uns. | Péter beteg, mégis velünk jön.\\
10 | Er trinkt viel, trotzdem ist er durstig. | Sokat iszik, mégis szomjas.\\
11 | Mein Friend lernt viel, trotzdem ist er dumm. | A barátom sokat tanul, mégis buta.\\
12 | Du machst Ordnung in deinem Zimmer, sonst kommst du nicht in Schwimmhalle! | Rendet csinálsz a szobádban, különben nem mész uszodába! !!!\\
13 | Er gibt mir Schokolade nicht, deshalb gehe ich nicht zu ihm. | Nem ad nekem csokoládét, ezért nem megyek hozzá. !!!\\
14 | Du kaufst das Abendessen, sonst kommst du mit mir nicht. | Megeszed a vacsorát, különben nem jössz velem.\\
16 | Ich stehe auf, dann trinke ich Tea. | Felkelek, aztán teát iszok.\\
17 | Ich schlafe viel, trotzdem bin ich müde. | Sokat alszom, mégis fáradt vagyok.\\
\end{exmp}

\subsection{KATI szórend}

GrammarItemIndex = 10

\begin{desc}
kötőszó + alany + többi mondatrész + ige ragozva.

A következő kötöszavak után van KATI szórend:
\begin{itemize}
\item dass - hogy
\item ob - vajon
\item als - amikor
\item wenn - ha
\item während - amíg
\item weil - mert
\item obwohl - habár
\end{itemize}

Az összes kérdőnévmás után KATI szórend van:
\begin{itemize}
\item wo - hol
\item wann - mikor
\item warum - miért
\item wer - ki
\item was - mi
\item wie - hogyan
\end{itemize}

Pl.: Ich weiß, dass er um 5 Uhr kommt. - Tudom, hogy 5 órakor jön.
\end{desc}

%Maklári: 149. oldal
\begin{exmp}
1 | Ich weiß, dass du weißt. | Tudom, hogy tudod.\\
2 | Er hört, dass wir kommen. | Hallja, hogy jövünk.\\
3 | Ich lese, während du schläfst. | Olvasok, amíg alszol.\\
4 | Sie is traurig, weil ihre Mutter nicht kommt. | Szomorú, mert nem jön az anyja.\\
5 | Ich sage, wenn er kommt. | Mondom, ha jön.\\
6 | Wir weisen nicht, wo ihr lebt. | Nem tudjuk, hol laktok.\\
7 | Ich habe keine Ahnung, wer sie ist. | Fogalmam sincs, ki ő (n).\\
8 | Wir hören, wenn er klingelnt. | Halljuk, ha csönget.\\
9 | Ich komme, wenn ich kann. | Jövök, ha tudok.\\
10 | Ich möchte weißen, ob er heute zu uns geht. | Szeretném tudni, vajon ma eljön-e hozzánk.\\
11 | Kannst du nicht, dass sie heißt? | Nem tudod, hogy hívják?\\
12 | Ich glaube, dass wir noch Zeit haben. | Azt hiszem, hogy még van időnk.\\
13 | Ich gebe alles für ihm, weil er mein Freund ist. | Mindent neki adok, mert a barátom.\\
14 | Weißt du, wer seine Eltern sind? | Tudod kik a szülei? !!!\\
15 | Wir weißen nicht, wann der Chef kommt. | Nem tudjuk, mikor jön a főnök.\\
16 | Er ist nicht sicher, ob er noch lebt. | Nem biztos abban, hogy él-e még.\\
17 | Hörtst du, wann die Kinder singen? | Hallod, mikor énekelnek a gyerekek?\\
18 | Ich rate seinen Name für ihm ver, wenn du zu uns kommst. | Elárulom a neked (f) a nevét (n), ha eljössz hozzánk.\\
\end{exmp}

\subsubsection{KATI-s mellékmondat a mondat elején}

GrammarItemIndex = 6592

\begin{desc}
Ha KATI-s mellékmondat a mondat elején áll, akkor az utána következő főmondat szórendje fordított lesz.

Pl.: Wann er mit seiner Freundin kommt, weiß ich nicht.

Pl.: Wenn sie müde ist, trinkt sie einen Kaffe.
\end{desc}

%Maklári: 152. oldal
\begin{exmp}
1 | Dass sie heißt, können wir nicht. | Hogy hogy hívják, fogalmunk sincs. !!!\\
2 | Warum sie ruft mich nicht an, weißt der Teufel. | Hogy miért nem hív fel, tudja az ördög.\\
\end{exmp}


\section{Az ige ragozása jelen időben}

\subsection{Szabályos ragozású igék}

GrammarItemIndex = 44

\begin{desc}
ich komme
du kommst
er, sie, es kommt
wir kommen
ihr kommt
sie, Sie kommen
\end{desc}

\begin{exmp}
1 | ich höre | én hallok\\
2 | du hörst | te hallassz\\
3 | er hört | ő hall (férfi)\\
4 | sie hört | ö hall (nő)\\
5 | es hört | ez hall\\
6 | wir hören | mi hallunk\\
7 | ihr hört | ti hallotok\\
8 | sie hören | ők hallanak\\
9 | Sie hören | Ön hall\\
\end{exmp}

\subsection{Igeragozás d-re, t-re végződő igető esetén}

GrammarItemIndex = 65

\begin{desc}
Ha az igető d-re vagy t-te végződik, a könnyebb kiejtés miatt egy -e
kötőhangot használunk Esz./2.,3. és Tsz./2. személyben.

Ilyen igék: bitten, finden, reden, antworten, baden, binden, arbeiten

ich bitte
du bittest
er bittet
sie bittet
es bittet
wir bitten
ihr bittet
sie bitten
Sie bitten
\end{desc}

\begin{exmp}
\end{exmp}

\subsection{Tőhangváltós, azaz erős igék}

GrammarItemIndex = 5231

\begin{desc}
A tőhangváltós igék | mint a nevükben is szerepel | megváltoztatják a szótövüket E/2 és E/3 személyben. Két típusuk van.
\end{desc}

\subsubsection{Umlautos igék}

GrammarItemIndex = 5463

\begin{desc}
\begin{tabular}{cc}
 ich & fahre \\
 du & fährst \\
 er, sie, es & fährt \\
 wir & fahren \\
 ihr & fahrt \\
 sie, Sie & fahren \\
\end{tabular}

Példa umlautos igékre: fallen, schlafen, fangen, laufen, tragen, graben, waschen, backen
\end{desc}

\begin{exmp}
\end{exmp}

\subsubsection{Brechungos igék}

GrammarItemIndex = 8341

\begin{desc}
e --> i; e --> ie

Pl.:

\begin{tabular}{ccc}
 ich & gebe & sehe \\
 du & gibst & siehst\\
 er, sie, es & gibt & sieht \\
 wir & geben & sehen \\
 ihr & gebt & seht \\
 sie, Sie & geben & sehen \\
\end{tabular}

Példák brechungos igékre: helfen (i), lesen (ie), sprechen (i), treffen (i), brechen (i), sterben (i), vergessen (i)
\end{desc}

\begin{exmp}
\end{exmp}

\subsubsection{essen és nehmen Brechungos igék ragozása}

GrammarItemIndex = 6453

\begin{desc}
essen = enni, nehmen = venni

\begin{tabular}{ccc}
ich & nehme & esse \\
du & nimmst & isst \\
er, sie, es & nimmt & isst \\
wir & nehmen & essen \\
ihr & nimmt & esst \\
sie, Sie & nehmen & essen \\
\end{tabular}
\end{desc}

\begin{exmp}
\end{exmp}

\subsection{Rendhagyó ragozású igék}

\subsubsection{sein}

GrammarItemIndex = 543266

\begin{desc}
\begin{tabular}{ccc}
 ich & bin \\
 du & bist \\
 er/sie/es & ist \\
 wir & sind \\
 ihr & seid \\
 sie/Sie & sind \\
\end{tabular}
\end{desc}

\begin{exmp}
\end{exmp}

\section{haben}

GrammarItemIndex = 42343

\begin{desc}
A haben igével birtoklást fejezünk ki, és amit birtoklunk, azt tárgyesetbe (Akkusativ) tesszük.

Ragozása:

\begin{tabular}{cc}
ich & habe\\
du & hast\\
er/sie/es & hat\\
wir & haben\\
ihr & habt\\
sie/Sie & haben\\
\end{tabular}
\end{desc}

\begin{exmp}
1 | Mein Vater hat ein Haus. | Apámnak van egy háza.\\
2 | Du hast Zeit. | Van időd.\\
3 | Sie haben einen Garten. | Van egy kertjük.\\
4 | Mutter hat viel Blumen. | Anyának sok virága van.\\
5 | Er hat ein Buch. | Neki (f) van egy könyve.\\
6 | Ich habe eine Freundin. | Nekem van egy barátnőm.\\
7 | Ihr habt Zeit immer. | Mindig van időtök.\\
8 | Józsi hat einen Kuli. | Józsinak van egy tolla.\\
9 | Hans und Gertrud haben einen Garten. | Hansnak és Gertrudnak van egy kertje.\\
\end{exmp}

\subsubsection{werden}

GrammarItemIndex = 54326

\begin{desc}
A werden ige azt fejezi ki, hogy valaki vagy valami válik, lesz
valamivé, tehát egy folyamatot jelöl azidőben:
Ich werde Millionär.

\begin{tabular}{cc}
 ich & werde \\
 du & wirst \\
 er/sie/es & wird \\
 wir & werden \\
 ihr & werdet \\
 sie/Sie & werden \\
\end{tabular}
\end{desc}

\begin{exmp}
1 | Ich werde böse! | Mérges leszek!\\
2 | Józsi wird Lehrer. | Tanár lesz Józsi.\\
3 | Wir werden hungrig. | Éhesek leszünk.\\
4 | Ihr werdet frisch. | Frissek lesztek.\\
5 | Wirst du Lehrer? | Tanár leszel?\\
6 | Ihr werdet durstig. | Szomjasak Lesznek.\\
7 | Bözsi wird Hausfrau. | Bözsi házlasszony lesz.\\
8 | Wir werden reich. | Gazdagok leszünk.\\
9 | Anna wird müde. | Anna fáradt lesz.\\
10 | Die Lehrerin wird gesund. | A tanárnő egészséges lesz.\\
\end{exmp}

\subsubsection{wissen}

GrammarItemIndex = 7452

\begin{desc}
wissen = tud, ismer valamit

Tante Ancika weiß alles. | Ancika néni mindent tud.

\begin{tabular}{cc}
 ich & weiß \\
 du & weißt \\
 er/sie/es & weiß \\
 wir & wissen \\
 ihr & wisst \\
 sie/Sie & wissen \\
\end{tabular}
\end{desc}

\begin{exmp}
1 | Ich weiß nichts. | Semmit se tudok.\\
2 | Weißt du den Weg? | Tudod az utat?\\
3 | Józsi weiß etwas. | Józsi tud valamit.\\
4 | Wisst ihr den Grund? | Tudjátok az okot?\\
5 | Weißt du seinen Name? | Tudod a nevét?\\
6 | Wissen Sie der Name des Hotels? | Tudja a hotel nevét?\\
7 | Wir wissen das. | Azt tudjuk.\\
\end{exmp}

\subsubsection{tun}

GrammarItemIndex = 54327

\begin{desc}
tun A, für + A = tesz, csinál vmit (vkiért/vmiért)

Pl.: Meine Oma tut alles für uns. | A nagyim mindent megtesz értünk.

\begin{tabular}{cc}
 ich & tue \\
 du & tust \\
 er/sie/es & tut \\
 wir & tun \\
 ihr & tut \\
 sie/Sie & tun \\
\end{tabular}
\end{desc}

\begin{exmp}
1 | Er tut nie. | Semmit se tesz (ffi).\\
2 | Tust du das? | Megteszed azt?\\
3 | Sie tut sein Mädchen hierher. | A lányát ide teszi (nő).\\
4 | Iht tut Salz in die Suppe. | Sót tesztek a levesbe. ????\\
5 | Was tust du? | Mit teszel?\\
6 | Er tut alles für uns. | Mindent megtesz (ffi) értünk.\\
7 | Ich tue das Radio in den Tisch. | A rádiót az asztalra teszem. ????\\
8 | Was tust du hier? | Mit csinálsz itt?\\
9 | Sie tut immer etwas Gutes. | Mindig valami jót tesz.\\
10 | Warum tut ihr das? | Miért teszitek ezt?\\
11 | Béla tut immer etwas Wunder. | Béla mindig valami csodát tesz.\\
12 | Ida tut viel für er. | Ida mindig csodát tesz. ????????\\
13 | Tust das Buch auf den Tisch. | Tedd a könyvet az asztalra. ??????????\\
14 | Ich tue das für sie. | Érte tettem ezt.\\
15 | Er tut das recht. | Azt helyesen teszi (ffi).\\
16 | Was tust du damit? | Mit csinálsz ezzel?\\
\end{exmp}

\subsubsection{A tun szélesebb felhasználási lehetőségei}

GrammarItemIndex = 5223

\section{Szórend kijelentő mondatban}

GrammarItemIndex = 6

\begin{desc}
Er wohnt hier. | Ő itt lakik.
\end{desc}

\begin{exmp}
1 | Ich trinke langsam. | Lassan iszok.\\
2 | Ihr kommt schnell. | Gyorsan jöttök.\\
3 | Wir trinken Wein. | Bort iszunk.\\
4 | Sie kommet hier. | Itt jön (nő).\\
5 | Der Bus kommt dort. | Ott jön a busz.\\
6 | Tina arbeitet langsam. | Tina lassan dolgozik.\\
7 | Sie trinkt Wasser. | Vizet iszik (nő).\\
8 | Ihr lernt gut. | Jól tanultok.\\
9 | Wir reden immer. | Mindi beszélünk.\\
10 | Er geht nach Hause. | Hazamegy (ff).\\
11 | Sie singen schön. | Szépen énekel. (Ön)\\
12 | Wir arbeiten hier. | Itt dolgozunk.\\
13 | Wir hören Musik. | Zenét hallgatunk.\\
14 | Sie reden dort. | Ott beszélnek.\\
15 | Sie geht langsam. | Lassan megy. (nő)\\
16 | Er kommt schnell. | Gyorsan jön. (ffi)\\
17 | Der Zug steht. | A vonat áll.\\
18 | Das Mädchen wohnen hier. | A lány itt lakik.\\
19 | Jörg arbeitet. | Jörg dolgozik.\\
20 | Sie wohnen dort. | Ott laknak.\\
\end{exmp}

\subsection{Szórend kijelentő mondatban kiemeléssel}

GrammarItemIndex = 9

\begin{desc}
Ha valamit ki akarunk emelni a mondatból és felhívni rá a figyelmet,
akkor azt az első helyre tesszük, az alany pedig az ige után a 3. helyre kerül.
Pl.: Hier wohnt er. | Itt lakik.
\end{desc}

\begin{exmp}
1 | Limonade trinkt er. | Limonádét iszik. (ffi)\\
2 | Zu Hause singt sie. | Otthon énekel. (nő)\\
3 | Falsch antworten sie. | Helytelen választ ad. (nő)\\
4 | Dort wohnt ihr. | Ott laktok.\\
5 | Die Hausaufgabe mache ich. | Megcsinálom a házifeladatot.\\
6 | Wasser trinkt Judit. | Wizet iszik Judit.\\
7 | In der Nähe arbeiten wir. | A közelben lakunk.\\
8 | Bier holt er. | Ho sört. (ffi)\\
9 | Dor arbeite ich. | Ott dolgozom.\\
10 | Tee trinkt Sie. | Teát iszik. (Ön)\\
11 | Kakao trinken wir. | Kakaót iszunk.\\
12 | Kaffe holt ihr. | Kávét hoztok.\\
13 | Musik hörst du. | Zenét halat. (ffi)\\
14 | Alkohol trinken wir. | Alkoholt iszunk.\\
15 | Zu Hause arbeitet er. | Otthon dolgozik. (ffi)\\
16 | Sofort kommt sie. | Azonnal jön. (nő)\\
17 | Radio hört er. | Rádiót hallgat. (ffi)\\
18 | Ildikó heißt sie. | Ildikónak hívják.\\
19 | Emil heißt er. | Emilnek hívják.\\
\end{exmp}

\section{Birtokos szerkezetek}

\subsection{Többes birtokos szerkezet}

GrammarItemIndex = 4322

\subsection{Birtokos szerkezet von + D-val}

GrammarItemIndex = 4324

\begin{desc}
der Sohn von dem Vater | az apa fia
\end{desc}

\begin{exmp}
1 | die Tochter von der Mutter | az anya lánya\\
2 | der Schuch von dem Vater | az apa cipője\\
3 | die Brille von den Kindern | a gyerekek szemüvege\\
4 | die Lampe von dem Einbrecher | a betörő lámpája\\
5 | das Zeugnis dem  Schüler | a tanuló bizonyítványa\\
\end{exmp}

\subsection{Személynevek birtokos szerkezete}

GrammarItemIndex = 42345

\begin{desc}
Peters Freundin | Péter barátnője
\end{desc}

\subsection{Határozatlan névmás}

GrammarItemIndex = 64534

\begin{desc}
\begin{tabular}{cccc}
 & hímnem & nőnem & semleges nem \\
 Nominativ & einer & eine & eines \\
 Akkusativ & einen & eine & eines \\
 Dativ & einem & einer & einem \\
\end{tabular}

Pl.: Dort steht eines. | Ott áll egy.
\end{desc}

\section{A névmás és főnév sorrendje}

GrammarItemIndex = 54334

\begin{desc}
\begin{enumerate}
\item Ha két főnév áll a mondatban, akkor a dativos áll al első, az akkusativos a második helyen.

Pl.: Der Lehrer erklärt dem Schüler den Satz. | A tanár elmagyarázza a diáknak a mondatot.

\item Ha két névmás áll a mondatban, akkor a sorrend megfordul.

Pl.: Der Lehrer erklärt ihn ihm. | A tanár elmagyarázza azt neki.

\item Ha egy névmás és egy főnév áll a mondatban, akkor a névmás mindig előbb szerepel.

Pl.: Er erklärt ihn dem Schüler. | Elmagyarázza azt a diáknak.

Pl.: Er erklärt ihm den Satz. | Elmagyarázza neki a mondatot.
\end{enumerate}
\end{desc}

\subsection{Kettős elöljárószavak}

GrammarItemIndex = 5323

\begin{desc}
A kettős elöljároszavak körbeveszik a főnevet, vagy a névmást.

\begin{enumerate}
\item an + D ... vorbei = mellet el

Pl.: Ich gehe am Rathaus vorbei. | Elmegyek a városháza mellet.

\item an + D ... entlang = mentén

Pl.:Sie spazieren am Ufer entlang. | A part mentén sétálnak.

\item mit + D ... zusammen = -val, -vel, együtt

Pl.: Er kommt mit Mutti zusammen. | Anyuval együtt jön.

\item auf + A ... zu = felé

Pl.: Die Direktorin kommt auf uns zu. Az igazgatónő felénk jön.

\item von + D ... an/ab = -tól, -től

Pl.: Vom 1. März an wohnt er hier. | Március elsejétől itt lakik.

\item um + G ... willen = kedvéért

Pl.: Um seines Freundes willen komme ich. | A barátja kedvéért jövök. 
\end{enumerate}
\end{desc}

\section{Igekötős igék}

GrammarItemIndex = 3524

\subsection{Elváló igekötős igék}

GrammarItemIndex = 7

\begin{desc}
Képzése: alany + ige ragozva + többi mundatrész + igekötő

Erika steht jeden Tag um 5 Uhr auf. | Erika minden nap 5-kor kel fel.

Pédák elváló igekötős igékre:
\begin{itemize}
\item anrufen - felhív
\item einschlafen - elalszik
\item weggehen - elmegy
\item einkaufen - bevásárol
\item einsteigen - beszáll vhová
\item aussteigen - kiszáll vhonnan
\item umsteigen - átszáll
\item ankommen - megérkezik
\item abfahren - elindul
\item zurückgehen - visszamegy
\end{itemize}

\end{desc}

\begin{exmp}
1 | Wir gehen um 5 Uhr weg. | 5-kor megyünk el.\\
2 | Wir rufen morgen Kati an. | Felhívjuk holnap Katit.\\
3 | Er kauft immer morgen in dem Geschäft. | Mindent bevásárol az üzletben.\\
4 | Der Zug fährt nicht nach Wien morgen ab. | A  vonat holnap nem indul Bécsbe.\\
5 | Meine Freundin schläft schnell ein. | A barátnőm gyorsan elalszik.\\
6 | Kommen die Gästen auf den Bahnhof an? | A vendégek megérkeznek a vasútállomásra?\\
7 | Der Chief kommt auf den Flughafen an. | S főnök megérkezik a repülőtérre.\\
8 | Steigt ihr auf die Strassenbahn um? | Átszálltok a villamosra?\\
9 | Wir steigen in den Bus ein. | Beszállunk a buszba.\\
10 | Ich steige nicht auf den Obus ein. | Nem szállok be a troliba.\\
11 | Die U-Bahn kommt jetzt an. | Most érkezik a földalatti.\\
12 | Gähst du so früh weg? | Ilyen hamar elmész?\\
13 | Der Passagier steigt aus der Strassenbahn aus. | Az utas kiszáll a villamosból.\\
14 | Geht ihr in der Schule zurück? | Visszamentek az iskolába?\\
15 | Dein Bus geht bald weg. | Hamarosan elmegy a buszod.\\
16 | Der Minister kommt jetzt das Parlament an. | A miniszter most érkezik a parlamentbe.\\
17 | Wir kaufen in der Kaufhalle immer ein. | Az ???ben bevásárolunk mindent.\\
18 | Rufst du ihn an? | Felhívod őt (f)?\\
19 | Ich rufe dich an. | Felhívlak.\\
\end{exmp}

\subsection{Nem elváló igekötős igék}

GrammarItemIndex = 8

\begin{desc}
Az alábbi igekötők nem válnak el az igétől:
ge-, be-, er-, re-, ent-, emp-, ver-, zer-, miss-

Néhány ilyen ige:
\begin{itemize}
\item gefallen - tetszik
\item bezahlen - kifizet
\item erzählen - elmesél
\item entdecken - felfedez
\item empfehlen - ajánl
\item verstehen - megért
\item zerstören - lerombol
\item missverstehen - félreért
\end{itemize}
\end{desc}

\begin{exmp}
1 | Gefällst dir das Mädchen? | Tetszik neked a lány?\\
2 | Ich bezahle e Rechnung. | Kifizetem a számlát.\\
3 | Ich erzähle dir eine Geschichte. | Elmesélek neked egy történetet.\\
4 | Warum kannst du nie entdecken? | Miért nem tudsz semmit sem felfedezni?\\
5 | Ich empfehle dir diese Lösung. | Ezt a megoldást ajánlom neked.\\
6 | Ich empfehle mich. | Ajánlom magamat.\\
7 | Ich empfehle dir dieses Buch. | Ajánlom neked ezt a könyvet.\\
8 | Verstähst du dass? | Érted azt?\\
9 | Er zerstört immer. | Mindent lerombol (hím).\\
10 | Sie missversteht mich immer. | Mindig félreért engem.\\
11 | Kannst du verstehe nicht, dass ich kann nicht kommen. | Nem tudod megérteni, hogy nem tudok jönni? !!!\\
12 | Ich empfehle dir ein gut Restaurant. | Ajánlok neked egy jó éttermet. !!!!\\
13 | Der Kind zerstört seine Spielzeuge. | A gyerek összetöri a játékait.\\
14 | Ich bezahle die Suppe und du bezählt die Süßigkeit. | Én fizetem a levest és te az édességet.\\
15 | Gefallt den Eltern die Wohnung? | Tetszik a szülőknek a lakás?\\
16 | Erzählt du mir die Geschichte der Ehre! | Meséld el neken a házasságod történetét! !!!!\\
17 | Wann erzählst du die Geschichte? | Mikor meséled el azt a történetet? !!!!\\
\end{exmp}

\subsection{Hol elváló, hol nem elváló igekötős igék}

GrammarItemIndex = 5232

\subsection{Az um igekötő}

GrammarItemIndex = 5342

\section{Kérdő mondatok}

\subsection{Kérdőszó nélküli mondat}

GrammarItemIndex = 6432

\begin{desc}
Képzése: állítmány + alany + többi mondatrész

Pl: Kommst du morgen ins Kino?
\end{desc}

\begin{exmp}
1 | Kommst du morgen ins Kino? | Jössz holnap moziba?\\
2 | Gehen wir nach Hause? | Haza megyünk?\\
3 | Lesen Sie Zeitung? | Újságot olvas?\\
4 | Essen ihr Suppe? | Esztek levest?\\
\end{exmp}

\subsection{Kérdőmondat kérdőszóval}

GrammarItemIndex = 42357

\begin{desc}
Képzése: kérdőszó + állítmány + alany + többi mondatrész

Pl: Wann kommst du nach Hause?
\end{desc}

\begin{exmp}
1 | Wann kommst du nach Hause? | Mikor jössz haza?\\
2 | Wer seht hier? | Ki áll itt?\\
3 | Wer sind deine Eltern? | Kik a szüleid?\\
4 | Wem gibst du die Blume? | Kinek adod a virágot?\\
5 | Was siehst du dort? | Kit látsz ott?\\
\end{exmp}

\section{Tagadás}

\subsection{Nein}

GrammarItemIndex = 5253

\begin{desc}
nein: egész mondatot tagadunk vele

Pl.: Nein, ich liebe dich nicht. | Nem, én nem szeretlek téged.
\end{desc}

\subsection{nicht}

GrammarItemIndex = 5335

\begin{desc}
nicht: mondatrészt tagadunk vele

\begin{enumerate}
\item Ha az állítmány egy tagú, akkor a mondat végén áll a nicht.

Pl.: Mein Vater kommt heute nicht. | Apám ma nem jön.

\item Ha az állítmány több tagú, akkor a nicht a mondat végére, az állítmány ragozatlan rész elé kerül.

Pl.: Wir stehen heute nicht auf. | Mi nem kelünk fel ma.
Pl.: Ich kann morgen nicht kommen. | Holnap nem tudok jönni.
Pl.: Wir haben den Brief nicht gelesen. | Nem olvastuk a levelet.

\item Ha egy bizonyos mondatrész tagadunk, akkor az elé tesszük.
\item He prepozíciós alak van a mondatben, akkor az elé tesszük.
Pl.: Wir fahren nicht nach Deutschland. | Nem utazunk Németországba.
\end{enumerate}
\end{desc}

\subsection{kein}

GrammarItemIndex = 5235

\begin{desc}
kein: főnevet tagadunk vele

Úgy rugozzuk, mint az ein határozatlan névelőt és a tagadott főnév elé írjuk.
\end{desc}

\begin{exmp}
1 | Er bekommt keinen Tisch. | Nem kap asztalt.\\
2 | Wir haben keine Zeit | Nincs időnk.\\
\end{exmp}

\subsection{nichts}

GrammarItemIndex = 4236

\begin{desc}
nichts = semmi, semmit
\end{desc}

\begin{exmp}
1 | Heute lerne ich nichts. | Ma semmit sem tanulok.\\
2 | Ich sehe nichts. | Semmit se látok.\\
3 | Er trinkt nichts. | Semmit se iszik.\\
4 | Siechst du nichts? | Semmit se látsz?\\
\end{exmp}

\subsection{niemand}

GrammarItemIndex = 54321

\begin{desc}
niemand = senki

Pl.: Ich kenne hier niemand.
\end{desc}

\subsection{nirgendwo}

GrammarItemIndex = 5254

\subsection{nie}

GrammarItemIndex = 6456

\subsection{noch nicht}

GrammarItemIndex = 5344

\begin{desc}
noch nicht = még nem

Pl: Ich gehe noch nicht in die Schule. | Még nem megyek iskolába.
\end{desc}

\subsection{nicht mehr}

GrammarItemIndex = 5233

\begin{desc}
nicht mehr = már nem

Pl.: Er wohnt nicht mehr hier. | Már nem lakik itt.
\end{desc}

\section{Mutató névmás}

GrammarItemIndex = 6546

\section{Visszaható névmás}

GrammarItemIndex = 43246

\begin{desc}
\begin{tabular}{cc}
 | tárgy eset & részes eset \\
 | mich & mir \\
 | dich & dir \\
 | sich & sich \\
 | uns & uns \\
 | euch & euch \\
 | sich & sich \\
\end{tabular}
\end{desc}

\begin{desc}
Ich wasche mich. | Mosakodom. \\
Sie kämmt sich 3 Stunden lange im Badezimmer. | 3 órán keresztül fésüködik a fürdőszobában.\\
Ich wasche mir die Hände. | Mosom a kezeimet.
\end{desc}

\section{Visszaható igék}

\subsection{Visszaható névmás tárgyesete}

GrammarItemIndex = 5453

\begin{desc}
\begin{tabular}{ll}
 | mich & magam \\
 | dich & magad \\
 | sich & magát \\
 | uns & magunkat \\
 | euch & magatokat \\
 | sich & magukat \\
\end{tabular}

\begin{itemize}
\item sich waschen - mosakodik
\item sich kämmen - fésülködik
\item sich rasieren - borotválkozik
\item sich anziehen - felöltözik
\item sich ausziehen - levetkőzik
\item sich fühlen - érzi magát valahogy
\item sich freuen - örül
\end{itemize}
\end{desc}

\begin{exmp}
1 | ich wasche mich | mosakodom\\
2 | sie kämmt sich | fésülködik (n)\\
3 | wir kämmen uns | fésülködünk\\
4 | er rasiert sich | borotválkozik\\
5 | sie säscht sich | mosakszik (n)\\
6 | ich freue mich | örülök\\
\end{exmp}

\subsection{Visszaható igék tárgyesettel}

GrammarItemIndex = 5435

\begin{desc}
Az alábbi igékhez kapcsolódó visszaható névmás mindig tárgyesetben van.

\begin{itemize}
\item sich interessieren für + A | érdeklődik vmi iránt
\item sich beschäftigen mit + D + foglalkozik vmivel
\item sich freuen auf + A | örül előre vminek
\item sich freuen über + A | örül vminek (ami már megvan)
\item sich treffen mit + D  találkozik vkivel
\item sich erinnen an + A | emlékezik vmire
\item sich irren in + D | téved vmiben
\end{itemize}
\end{desc}

\begin{exmp}
1 | Ich interessiere mich für Sport. | Érdeklődöm a sport iránt.\\
2 | Er beschäftigt sich mit der deutschen Sprache. | Ő foglalkozik a német nyelvvel.\\
3 | Die Kinder beschäftigen sich schon im November auf Weihnachten. | A gyerekek már novemberben a karácsonnyal foglalkoznak.\\
4 | Freust du dich über die Geschenke? | ???\\
5 | Bei uns treffen wir uns mit deinen Freunden. | ???\\
6 | Ich erinnere mich noch an deine Eltem. | ???\\
7 | Paul irrt sich immer im Datum. | ???\\
\end{exmp}

\subsection{Visszaható igék részesesettel}

GrammarItemIndex = 54353

\subsection{Ikerigék}

GrammarItemIndex = 5325

\begin{desc}
Vannak olyan ikerigék, melyeknek van sich-es és sich nélküli
formája is, mely jelentésüket is módosítja.
\end{desc}

\section{Idők}

\section{Präteritum - elbeszélő múlt}

GrammarItemIndex = 54358

\begin{desc}
Képzése:
\begin{enumerate}
\item
Gyenge igék: ige + te + személyrag

\begin{tabular}{cc}
 ich & fragte \\
 du & fragtest \\
 er/sie/es & fragte \\
 wir & fragten \\
 ihr & fragtet \\
 sie/Sie & fragten
\end{tabular}

\item
Erős igék: szótő megváltozik.
\item
Vegyes igék: szótő megváltozik és -te képzőt is kap.
\end{enumerate}
\end{desc}

\subsection{Elbeszélő múlt - gyenge igék}

GrammarItemIndex = 42346

\begin{desc}
Képzése: ige + te + személyrag
Pl.:
ich fragte
du fragtest
er/sie/es fragte
wir fragten
iht fragtet
sie/sie fragten

Ha a szótő t, d, n -re végződik, akkor a szótő egy -e hangot kap.
Pl.:
ich	antwortete
du	antwortetest
er/sie/es	antwortete
wir	antworteten
ihr antwortetet
sie	antworteten
\end{desc}

%Maklári: 244. oldal
\begin{exmp}
1 | kauftet | vásárolt\\
2 | suchte | kerestem\\
3 | sagte | mondtam\\
4 | hörte | hallotta\\
5 | machte | csinálta\\
6 | wartetet | wártatok\\
7 | antwortete | felelt\\
8 | arbeitete | dolgoztam\\
9 | kauftet | vásároltak\\
10 | spielten | játszottak\\
11 | öffnete | kinyitotta\\
12 | öffnete | kinyitottam\\
13 | sagetet | mondtátok\\
14 | öffnetet | kinyitottátok\\
15 | hörtet | hallottátok\\
\end{exmp}

\subsection{Erős igék}

GrammarItemIndex = 61

\begin{desc}
Erős igék esetében a szótő megváltozik.

ich fuhr
du fuhrst
er/sie/es fuhr
wir fuhren
ihr fuhrt
sie/Sie fuhren

Példa erős igékre:
\begin{itemize}
\item fahren - fuhr
\item singen - sang
\item schreiben -schrieb
\item essen - aß
\item trinken - trank
\item kommen - kam
\item gehen - ging
\item sprechen - sprach
\item bleiben - blieb
\item geben - gab
\end{itemize}
\end{desc}

\begin{exmp}
1 | schriebten | írtunk\\
2 | aß | evett\\
3 | trankt | itatok\\
4 | fuhren | utazunk\\
5 | kamen | jöttek\\
6 | blieben | maradunk\\
7 | trank | evett\\
8 | ging | ment\\
9 | kam | jött\\
10 | gaben | adtatok\\
11 | sprachen | beszéltünk\\
12 | aßen | ettek\\
13 | kam | jöttem\\
14 | schrieben | írtunk\\
15 | fuhrst | utaztál\\
16 | aßen | ettünk\\
17 | sang | énekelt\\
18 | schriebt | írtatok\\
19 | blieb | maradt\\
20 | gabst | adtál\\
21 | schrieb | írt\\
22 | gab | adott\\
23 | sprach | beszélt\\
24 | schrieben | írtak\\
25 | sang | énekeltem\\
26 | fuhr | utazott\\
27 | bliebst | maradtál\\
28 | trank | ittam\\
\end{exmp}

\subsection{Vegyes igék}

GrammarItemIndex = 54355

\begin{desc}
Vegyes igéknél a tőhang megváltozik és -te végződést is kap.

ich dachte
du dachtest
er/sie/es dachte
wir dachten
ihr dachtet
sie/Sie dachten

Példák vegyes igékre:
\begin{itemize}
\item denken - dachte
\item nennen - nannte
\item kennen - kannte
\item bringen - brachte
\item rennen - rannte
\item wissen - wusste
\item brennen - brannte
\end{itemize}
\end{desc}

%Maklári: 245. oldal
\begin{exmp}
1 | dachte | gondoltam\\
2 | kannte | ismerte\\
3 | kanntet | ismertétek\\
4 | rannten | rohantunk\\
5 | wusste | tudta\\
6 | brachte | hoztam\\
7 | brannte | égett\\
8 | wusstet | tudtátok\\
9 | nannte | neveztem\\
10 | dachte | gondolta\\
11 | wusste | gondoltam\\
12 | nannte | nevezett\\
13 | kannte | ismertem\\
14 | brachtet | hoztátok\\
15 | dachten | gondoltuk\\
16 | brachtest | hoztad\\
17 | kannten | ismertük\\
18 | nanntest | neveztél\\
19 | wussten | tudtuk\\
20 | dachtet | gondoltátok\\
21 | brannte | égett\\
22 | kanntest | ismerted\\
23 | nannten | neveztünk\\
24 | wussten | tudták\\
25 | nanntet | neveztétek\\
26 | dachtest | gondoltad\\
27 | kannten | ismerték\\
28 | brachte | hozta\\
\end{exmp}

\section{Perfekt - befelyezett múlt}

GrammarItemIndex = 54539

\begin{desc}
Ebben a múltidőben kifelyezett események kihatnak a jelenre.
\begin{center}
alany + segédige (haben/sein) + többi mondatrész + ige befelyezett alakja
\end{center}

\textbf{Példák:} Ich bin von euch gekommen. | Tőletek jöttem.\\
Bereits vor ungefähr 20.000 Jahren haben Menschen den Bumerang zur Jagd benutzt.
\end{desc}

\subsection{A sein perfektje}

GrammarItemIndex = 16354

\begin{desc}
\begin{itemize}
\item ich bin gewesen - voltam
\item du bist gewesen - voltál
\item er/sie/es ist gewesen - volt
\item wir sind gewesen - voltatak
\item ihr seid gewesen - voltatok
\item sie/Sie sind gewesen - voltak
\end{itemize}
\end{desc}

%Maklári: 236. oldal
\begin{exmp}
1 | Ich bin schon bei ihm gewesen. | Voltam már nála (f).\\
2 | Sie ist müde gestern gewesen. | Fáradt volt (n) tegnap.\\
3 | Er ist nicht zu Hause gewesen. | Nem volt (f) otthon.\\
4 | Ihr seid klug gewesen. | Okosak voltatok.\\
5 | Ist sie dort auch gewesen? | Ott volt ő (n) is?\\
6 | Warum bist du nicht dort gewesen? | Miért nem voltál ott?\\
7 | Jürgen ist nicht bei ihr gewesen. | Jürgen nem volt nála.\\
8 | Wo sind Sie gester gewesen? | Hol volt (Ön) tegnap?\\
9 | Ist Anna krank gewesen? | Beteg volt Anna?\\
10 | Der Film ist langweilig gewesen. | Unalmas volt a film.\\
11 | Seid ihr nicht in dem Haus gewesen? | Nem voltatok a házban?\\
12 | Wo seid ihr gestern gewesen? | Hol voltatok tegnap?\\
13 | Wann bist du das Letzte Mal in Oper gewesen? | Mikor voltál utóljára operában?\\
14 | Ist der Film schlecht gewesen? | Rossz volt a film?\\
15 | Ich bin nicht in Schule gewesen. | Ma nem voltam iskolában.\\
16 | Sind sie durstig gewesen? | Szomjasak voltak?\\
17 | Sind die Gäste hungrig gewesen? | Éhesek voltak a vendégek?\\
\end{exmp}

\section{A haben perfektje}

GrammarItemIndex = 56423

\begin{desc}
\begin{itemize}
\item ich habe gehabt - nekem volt
\item du hast gehabt - neked volt
\item er/sie/es hat gehabt - neki volt
\item wir haben gehabt - nekünk volt
\item ihr habt gehabt - nektek volt
\item sie/Sie haben gehabt - nekik volt
\end{itemize}
\end{desc}

\begin{exmp}
\end{exmp}

\subsubsection{Kettős infinitiv szerkezet perfektben}

GrammarItemIndex = 43247

\begin{desc}
Ich habe ihn kommen hören. | Hallottam őt jönni.
\end{desc}

\subsection{Plusquamerfekt | előidejűség a múltban}

GrammarItemIndex = 65468

\begin{desc}
alany + haben/sein Präteriuma + többi mondatrész + ige befelyezett alakja
\end{desc}

\subsection{Feltételes jelenidő | würden + infinitiv}

GrammarItemIndex = 43249

\begin{desc}
alany + würden ragozva + többi mondatrész + főnévi igenév.

Pl.: Ich würde gern morgen zu euch fahren. | Szívesen elutaznék holnap hozzátok.
\end{desc}

\section{Feltételes jelen idő}

GrammarItemIndex = 6564

\begin{desc}
alany + würden ragozott alakja + többi mondatrész + ige Infinitives alakja

Pl.: Ich würde gern zu euch fahren. | Szívesen elutaznék hozzátok.
\end{desc}

\section{Cselekvő passiv mondatszerkezet}

\subsection{Passiv jelen időben}

GrammarItemIndex = 52351

\begin{desc}
alany + werden + többi mondatrész + ige befelyezett alakja

Pl.: Ein Haus wird von Peter gebaut. | "Egy ház péter által van építve." 
\end{desc}

\subsection{Passiv elbeszélő múlt időben}

GrammarItemIndex = 423216

\begin{desc}
alany + wurden + többi mondatrész + ige befelyezett alakja

Pl.: Ein Haus wurde von Peter gebaut. - "Egy ház péter által volt építve."
\end{desc}

\subsection{Passiv befejezett múlt időben - Passiv perfect}

GrammarItemIndex = 523523

\begin{desc}
alany + sein + többi mondatrész + ige + worden

Das Haus ist von Peter gabaut worden. | "Egy ház péter által lett építve."
\end{desc}

\subsection{Passiv módbeli segédigével}

GrammarItemIndex = 4232134

\begin{desc}
alany + segédige + többi mondatrész + ige befelyezett alakja + werden

Das Auto must von dem Mechanicer repariert werden. | A szerelőnek meg kell javítani az autót.
\end{desc}

\subsection{Passiv módbeli segédigével elbeszélő múlt időben}

GrammarItemIndex = 523452

\begin{desc}
alany + segédige Präteriumban ragozva + többi mondatrész + ige befelyezett alakja + werden

Pl.: Das Auto musste von dem Mechaniker repariert werden. | A szerelőnek meg kelett javítania az autót.
\end{desc}

\subsection{Passiv módbeli segédigével befelyezett múlt időben}

GrammarItemIndex = 52326

\begin{desc}
alany + haben + többi mondatrész + ige befelyezett alakja + werden + segédige Infinitivben

Pl.: Das Auto hat von dem Mechaniker repariert werden müssen.
\end{desc}

\subsection{Passiv módbeli segédigével KATI szórenddel}

GrammarItemIndex =5234

\subsection{Passiv jövő időben}

GrammarItemIndex = 5413

\begin{desc}
alany + werden ragozott alakja + többi mondatrész + ige befelyezett alakja + werden

Die Vase wird in der Ence gestellt werden.
\end{desc}

\section{Állapot passzív}

\subsection{Jelen időben}

GrammarItemIndex = 42342

\begin{desc}
alany + sein ragozott alakja + ige befelyezett alakja

Pl.: Die Tür ist geschlossen. | Az ajtó be van zárva.
\end{desc}

\subsection{Elbeszélő múlt időben}

GrammarItemIndex = 342423

\begin{desc}
alany + waren ragozott alakja + ige befelyezett alakja

Pl.: Die Tür war geschlossen. | Az ajtó be volt zárva.
\end{desc}

\subsection{Elbeszélő múlt időben}

GrammarItemIndex = 45353

\begin{desc}
alany + sein ragozott alakja + ige befelyezett alakja + gewesen

Pl.: Die Tür ist geschlossen gewesen. - Az ajtó be volt zárva.
\end{desc}

\section{Melléknévfokozás}

\subsection{Melléknévfokozás | Reguláris melléknevek}

GrammarItemIndex = 534452

\begin{desc}

\begin{enumerate}
\item Alapfok: szótári alak
\item Középfok: melléknév + er

Pl.: schöner | szebb

\item Felsőfok:
\begin{enumerate}
\item határozószóként használt melléknév esetén: melléknév + sten

Pl.: schönsten | legszebb

\item jelzőként használt melléknév esetén: (névelő) + melléknév + ste + (főnév)

Pl.: die schönste Frau | a legszebb nő
\end{enumerate}

\item Abszolút felsőfok:
\end{enumerate}
\end{desc}

\begin{exmp}
1 | dicker | kövérebb\\
2 | dickste | legkövérebb\\
3 | magerer | soványabb\\
4 | niedriger | alacsonyabb\\
5 | billiger | olcsóbb\\
6 | breiter | szélesebb\\
7 | geschickter | ügyesebb\\
8 | schneller | gyorsabb\\
9 | langsamer | lassabb\\
10 | teuer | drágább\\
11 | dickste | legkövérebb\\
12 | magerste | legsoványabb\\
13 | niedrigste | legalacsonyabb\\
14 | billigste | legolcsóbb\\
15 | breiterste | legszélesebb\\
16 | geschicktste | legügyesebb\\
17 | schnellste | leggyorsabb\\
18 | langsamte | leglassabb\\
19 | teuerste | legdrágább\\
\end{exmp}

\subsection{Rendhagyó melléknevek}

GrammarItemIndex = 13214

\begin{desc}
gut | besser | beste
vie | mehr | meiste
gern | lieber | liebste
nah | näher | nähste
hoch | höher | höchste
groß | größer | größte
\end{desc}

\subsection{Melléknévfokozás | Umlautos melléknevek}

GrammarItemIndex = 42342321

\begin{desc}
Az egyszótagos a, o, u tövű melléknevek umlautot kapnak.
alt
kalt
jung
stark
lang
warm
kurz
hart
schwach
dumm
klug
arm
scharf
\end{desc}

\begin{exmp}
1 | älter | öregebb\\
2 | kälteste | leghidegebb\\
3 | jünger | fiatalabb\\
4 | stäreker | erősebb\\
5 | länger | hosszabb\\
6 | wärmeste | legmelegebb\\
7 | kürzer | rövidebb\\
8 | härter | keményebb\\
9 | schwächer | gyengébb\\
10 | dümmer | butább\\
11 | klügeste | legokosabb\\
12 | ärmer | szegényebb\\
13 | schärfeste | legélesebb\\
\end{exmp}

\section{Az es gibt szerkezet}

GrammarItemIndex = 423424312

\begin{desc}
Az es gibt+Akk. (= van) szerkezet használatáról: Az "es gibt" szerkezetet akkor használjuk, ha valaminek a létezését vagy a nem létezését hangsúlyozzuk.

Pl.: Es gibt sehr viele Sehenswürdigkeiten in Salzburg. | Nagyon sok látnivaló van Salzburgban.
\end{desc}

\section{Célhatározói mellékmondat képzése um + zu + Infinitiv szerkezettel}

GrammarItemIndex = 423412

\begin{desc}
Célhatározói mellékmondat képzésére használatos.

um + többi mondatrész + zu + ige Infinitiv alakja

Pl.: Sokat dolgozik, azért, hogy több pénzt keressen. | Er arbeitet viel, um mehr Geld zu verdienen.

Pl.: Die Regeirung sah die Notwendigkeit von Konzessionen, um einer Revolution zuvorzukommen. | A kormány látta az engedmények szükségességét, azért, hogy megelőzzenek egy forradalmat.
\end{desc}

\section{Függő beszéd}

GrammarItemIndex = 421412

\section{Páros kötőszavak}

GrammarItemIndex = 44242

\subsection{sowohl | als auch}

GrammarItemIndex = 5324532

\begin{desc}
Jelentése: is | is. A mellékmondatban fordított szórendet használunk.

Pl: Ich gehe heute sowohl zu dir, als auch besuche ich meine Oma. 
\end{desc}

\subsection{weder | noch}

GrammarItemIndex = 123531

\begin{desc}
Jelentése: sem | sem. A mellékmondatban fordított szórendet használunk.

Pl: Ich trinke weder Alkohol, noch esse ich Fleisch.
Pl: Ich bin weder schön noch intelligent.
Pl.: Sie liebt weder mich noch meine Freundin noch Géza.
\end{desc}

\subsection{entweder | oder}

GrammarItemIndex = 543532

\begin{desc}
Jelentése vagy- vagy. Mindkét tagmondatban egyeneszórendet használunk.

Pl: Entweder fahren wir heute zur oma, oder wir besuchen meine Tochter.
Entweder fahren wir mit ihnen oder mit deinen Eltern.
\end{desc}

\subsection{nicht nur | sondern auch}

GrammarItemIndex = 543423

\begin{desc}
Jelentése vagy- vagy.
\end{desc}

\section{Melléknév ragozás}

\subsection{Erős ragozás}

GrammarItemIndex = 43242

\begin{desc}
Erős ragozás használatos, ha:

\begin{itemize}
	\item a főnévhez nem tartozik névelő (Gute nacht!),
	\item tőszámnév tartozik a főnévhez (drei gute Bücher),
	\item határozatlan tartozika főnévhez (viele gute Freunde).
\end{itemize}

\begin{tabular}{ccccc}
 |  & hímnem & nőnem & semleges nem & többes szám\\
 | N & guter Wein & warme Milch & frisches Brot & gute Bücher\\
 | A & guten Wein & warme Milch & frisches Brot & gure Bücher\\
 | D & gutem Wein & warmer Milch & frischem Brot & guten Büchern\\
 | G & guten Weines & warmer Milch & frischen Brotes & guter Bücher\\
\end{tabular}
\end{desc}

\subsection{Gyenge ragozás}

GrammarItemIndex = 43242132

\subsection{Vegyes ragozás}

GrammarItemIndex = 43122

\section{Passzív mondetszerkezet}

\subsection{Passzív jelen időben}

GrammarItemIndex = 13325

\begin{desc}
Képzése: werden ragozott alakja + ige Partizip Perfekt alakja
Passziv mondat alanyát von + D (cselekvő alany), vagy durch + A (élettelen a cselekvés okozója) szószerkezettel fejezhetjük ki.
\end{desc}

\begin{exmp}
1 | Ein Haus wird von Peter gebaut. | Péter épít egy házat.\\
2 | Das Fleisch wird von Opa gebraten. | A húst nagyapa süti.\\
3 | Oma wird durch das Heilwasser geheilt. | A gyógyvíz meggyógyítja nagyit.\\
4 | Die Suppe wird von Mutti gekocht. | Anya levest főz.\\
5 | Viele Motorraden werden von dem Verkäufer verkauft. | A vásárló sok motorkerékpárt vásárol.\\
\end{exmp}

\subsection{Passzív elbeszélő múlt időben}

GrammarItemIndex = 43242321

\begin{desc}
Képzése: wurden ragozott alakja + ige Partizip Perfekt alakja
\end{desc}

\begin{exmp}
1 | Das Haus wurde von Peter gebaut. | A házat Péter építette.\\
2 | Die Suppe wurde von Oma gekocht. | A levest mama főzte.\\
3 | Die Uhren wurden repariert. | Az órákat megjavították.\\
4 | Wohin wurden die Regale gestellt? | Hová állították a polcokat?\\
5 | Die Musik wurde von den Gästen gehört. | A vendégek zenét hallgattak.\\
\end{exmp}

\subsection{Passzív befelyezett múlt időben}

GrammarItemIndex = 44232

\begin{desc}
Képzése: sein ragozott alakja + ige Partizip Perfekt alakja + worden
\end{desc}

\begin{exmp}
1 | Das Haus ist von Peter gebaut worden. | A házat Péter építette.\\
2 | Mein Fieber ist von dem Arzt gemessen worden. | Az orvos megmérte a lázamat.\\
3 | Die Geschenke sind schon gekauft worden. | Az ajándékokat mevették már.\\
\end{exmp}

\subsection{Passzív jelen időben módbeli segédigével}

GrammarItemIndex = 43242654

\begin{desc}
Képzése: módbeli segédige ragozott alakja + ige Partizip Perfekt alakja + werden
\end{desc}

\begin{exmp}
1 | Das Auto muss von dem Mechaniker repariert werden. | Az autót meg kell javítania a szerelőnek.\\
2 | Der Apfel darf nicht gegessen werden. | Az almát nem szabad megenni.\\
3 | Das Zimmer muss aufgeräumt werden. | A szobát rendbe kell tenni.\\
4 | Der Patient darf operiert werden. | A beteget meg szabad operálni.\\
\end{exmp}

\section{Az általános alany}

\subsection{A Man}

GrammarItemIndex = 5864

\begin{desc}
Pl.: Man kann hier gut essen. - Itt jót lehet enni.
Pl.: Man isst nicht viel in der Oper. - Az operában nem eszünk sokat. 

Man után az állítmány csak E/3 személyben állhat.
\end{desc}

%Maklári: 116. oldal
\begin{desc}
1 | Man spricht hier englisch. | Itt beszélnek angolul.\\
2 | Man raucht hier nicht. | Itt nem dohányoznak.\\
3 | Man kann hier billig verkaufen. | Itt olcsón lehet bevásárolni.\\
4 | Man darf in der Restaurant singen. | Az étteremben szabad énekelni.\\
5 | Man arbeitet in Deutschland viel. | Németországban sokat dolgoznak.\\
6 | Man liebt gut in diesem Land. | Ebben az országban jól élnek.\\
7 | Man steht bei uns um 8 Uhr auf. | Nálunk 8-kor kelnek.\\
8 | Man darf hier nicht rauchen. | Itt nem szabad dohányozni.\\
9 | Wie sagt Mann in deutsch? | Hogy mondják németül?\\
10 | Man trinkt nicht vor Autofahren. | Vezetés előtt nem iszunk.\\ 
11 | Man isst Schokolade nicht vor Mittagessen. | Ebéd előtt nem eszünk csokit.\\
12 | Man spricht nicht bei Essen. | Étkezés közben (étkezésnél) nem beszélünk.\\
13 | Man kann hier nicht Fußball spielen. | Itt nem szabad focizni.\\
14 | Man isst viel Fleisch in Ungarn. | Magyarországom sok húst esznek.\\
15 | Man kan ohne Wasser nicht leben. | Víz nélkül nem lehet élni.\\
16 | Man darf dorthin nicht treten. | Oda nem szabad lépni.\\
17 | Was darf man hier tun? | Mit lehet itt tenni?\\
18 | Wie sagt man es in englisch? | Hogy mondják ezt angolul?\\
\end{desc}

\section{Erős igék listája}

GrammarItemIndex = 3424

\begin{exmp}
1 | anbieten | meg-/felkínál, INFINITIV\\
2 | anrufen | felhív telefonon - INFINITIV\\
3 | anfangen | elkezd(ődik) - INFINITIV\\
4 | ankommen | megérkezik - INFINITIV\\
5 | anziehen | felöltöztet - INFINITIV\\
6 | aufstehen | felkel - INFINITIV\\
7 | ausgehen | szórakozni megy - INFINITIV\\
8 | aussteigen | kiszáll - INFINITIV\\
9 | backen | süt (húst) - INFINITIV\\
10 | befehlen | parancsol - INFINITIV\\
11 | beginnen | elkezd(ődik) - INFINITIV\\
12 | beißen | harap - INFINITIV\\
13 | bekommen | kap - INFINITIV\\
14 | bergen | elrejt - INFINITIV\\
15 | betrügen | becsap - INFINITIV\\
16 | bewegen | indít vmire,motivál - INFINITIV\\
17 | biegen | hajlít - INFINITIV\\
18 | bieten | nyújt, kínál - INFINITIV\\
19 | binden | (meg)köt - INFINITIV\\
20 | bitten | kér - INFINITIV\\
21 | blasen | fújni - INFINITIV\\
22 | bleiben | marad - INFINITIV\\
23 | bringen | hoz - INFINITIV\\
24 | brechen | eltör valamit - INFINITIV\\
25 | brechen | valami eltörik - INFINITIV\\
26 | brennen | ég - INFINITIV\\
27 | denken | gondol - INFINITIV\\
28 | eindringen | behatol - INFINITIV\\
29 | dürfen | szabad - INFINITIV\\
30 | einbiegen | bekanyarodik - INFINITIV\\
31 | einladen | meghív - INFINITIV\\
32 | einschlafen | elalszik - INFINITIV\\
33 | einsteigen | beszáll vhová - INFINITIV\\
34 | empfangen | fogad - INFINITIV\\
35 | empfehlen | ajánl - INFINITIV\\
36 | empfinden | érez,észlel vmit - INFINITIV\\
37 | erlöschen | elalszik a tűz - INFINITIV\\
38 | erschrecken | megijed - INFINITIV\\
39 | essen | eszik - INFINITIV\\
40 | fern|sehen (ie) | TV-t néz - INFINITIV\\
41 | fahren | vezet - INFINITIV\\
42 | fahren | utazik - INFINITIV\\
43 | fallen | esik - INFINITIV\\
44 | fangen | (meg)fog,(el)kap - INFINITIV\\
45 | finden | talál - INFINITIV\\
46 | fliegen | repül - INFINITIV\\
47 | fliehen | menekül - INFINITIV\\
48 | fließen | folyik - INFINITIV\\
49 | fressen | zabál,állat eszik - INFINITIV\\
50 | frieren | fázik/megfagy - INFINITIV\\
51 | geben | ad - INFINITIV\\
52 | gehen | megy - INFINITIV\\
53 | gelingen | sikerül - INFINITIV\\
54 | gelten | érvényes - INFINITIV\\
55 | genesen | gyógyul - INFINITIV\\
56 | genießen | élvez - INFINITIV\\
57 | geschehen | történik - INFINITIV\\
58 | gewinnen | nyer - INFINITIV\\
59 | gießen | önt, tölt - INFINITIV\\
60 | graben | ás - INFINITIV\\
61 | greifen | megragad, (meg)fog - INFINITIV\\
62 | haben | birtokol vmit, van neki - INFINITIV\\
63 | halten | tart - INFINITIV\\
64 | hängen | lóg, függ - INFINITIV\\
65 | heben | emel - INFINITIV\\
66 | heißen | hívják vhogyan - INFINITIV\\
67 | helfen | segít - INFINITIV\\
68 | kennen | ismer - INFINITIV\\
69 | klingen | cseng, kong - INFINITIV\\
70 | kommen | jön - INFINITIV\\
71 | können (kann) | tud (csinálni), (csinál)hat,-het - INFINITIV\\
72 | laden | megrak, megterhel  - INFINITIV\\
73 | lesen | olvas - INFINITIV\\
74 | liegen | fekszik - INFINITIV\\
75 | lassen | hagy - INFINITIV\\
76 | laufen | fut - INFINITIV\\
77 | leiden | szenved - INFINITIV\\
78 | leihen | kölcsönöz - INFINITIV\\
79 | lügen | hazudik - INFINITIV\\
80 | meiden  | kerül vkit/vmit - INFINITIV\\
81 | messen | mér (távolságot) - INFINITIV\\
82 | müssen | kell (csinálnia) - INFINITIV\\
83 | mögen | szeret (csinálni) - INFINITIV\\
84 | nehmen | fog, vesz - INFINITIV\\
85 | nennen | nevez - INFINITIV\\
86 | raten | tanácsol - INFINITIV\\
87 | reiten | lovagol - INFINITIV\\
88 | rennen | rohan - INFINITIV\\
89 | rufen | kiált, hív - INFINITIV\\
90 | schaffen | teremt - INFINITIV\\
91 | scheinen | 1.fénylik, süt(a Nap) 2.úgy tűnik - INFINITIV\\
92 | schieben | tol - INFINITIV\\
93 | schießen | lő - INFINITIV\\
94 | schlafen | alszik - INFINITIV\\
95 | schlagen | üt, ver - INFINITIV\\
96 | schließen | zár - INFINITIV\\
97 | schmelzen | olvad - INFINITIV\\
98 | schneiden | vág - INFINITIV\\
99 | schreiben | ír - INFINITIV\\
100 | schreien | kiabál - INFINITIV\\
101 | schreiten | lép(ked) - INFINITIV\\
102 | schweigen | hallgat - INFINITIV\\
103 | schwimmen | úszik - INFINITIV\\
104 | schwören | esküszik - INFINITIV\\
105 | sehen | lát, néz - INFINITIV\\
106 | sein | létige - INFINITIV\\
107 | singen | énekel - INFINITIV\\
108 | sinken | süllyed - INFINITIV\\
109 | sitzen | ül - INFINITIV\\
110 | sollen | kell (csinálni) - INFINITIV\\
111 | sprechen | beszél - INFINITIV\\
112 | stehen | áll - INFINITIV\\
113 | springen | ugrik - INFINITIV\\
114 | stechen | megcsíp - INFINITIV\\
115 | stehlen | lop - INFINITIV\\
116 | steigen | emelkedik - INFINITIV\\
117 | sterben | meghal - INFINITIV\\
118 | stinken | bűzlik - INFINITIV\\
119 | stoßen | lök - INFINITIV\\
120 | streichen | 1.simít 2.mázol 3. kihúz, kitöröl - INFINITIV\\
121 | streiten | veszekedik - INFINITIV\\
122 | rufen | hív, kiabál - INFINITIV\\
123 | tragen | hord,visel,cipel - INFINITIV\\
124 | treffen | találkozik - INFINITIV\\
125 | treiben | űz, hajt - INFINITIV\\
126 | treten | tapos, tipor - INFINITIV\\
127 | treten | lép,jár,megy - INFINITIV\\
128 | trinken | iszik - INFINITIV\\
129 | umsteigen | átszáll - INFINITIV\\
130 | vergessen | elfelejt - INFINITIV\\
131 | verlassen | elhagy vkit/vmit - INFINITIV\\
132 | verlieren | elveszít vmit/vkit - INFINITIV\\
133 | verstehen | (meg)ért - INFINITIV\\
134 | wachsen | nő - INFINITIV\\
135 | waschen | mos - INFINITIV\\
136 | wissen | tud vmit, vmiről - INFINITIV\\
137 | wollen | akar - INFINITIV\\
138 | ziehen | húz - INFINITIV\\
139 | zwingen | kényszerít - INFINITIV\\
140 | bot an | meg-/felkínál - PRÄTERIUM\\
141 | rief an | felhív telefonon - PRÄTERIUM\\
142 | fing an | elkezd(ődik) - PRÄTERIUM\\
143 | kam an  | megérkezik - PRÄTERIUM\\
144 | zog an | felöltöztet - PRÄTERIUM\\
145 | stand auf | felkel - PRÄTERIUM\\
146 | ging aus | szórakozni megy - PRÄTERIUM\\
147 | stieg aus | kiszáll - PRÄTERIUM\\
148 | backte | süt (húst) - PRÄTERIUM\\
149 | befahl | parancsol - PRÄTERIUM\\
150 | begann | elkezd(ődik) - PRÄTERIUM\\
151 | biss | harap - PRÄTERIUM\\
152 | bekam | kap - PRÄTERIUM\\
153 | verbarg | elrejt - PRÄTERIUM\\
154 | betrog | becsap - PRÄTERIUM\\
155 | bewog | indít vmire,motivál - PRÄTERIUM\\
156 | bog | hajlít - PRÄTERIUM\\
157 | bot | nyújt, kínál - PRÄTERIUM\\
158 | band | (meg)köt - PRÄTERIUM\\
159 | bat | kér - PRÄTERIUM\\
160 | blies | fújni - PRÄTERIUM\\
161 | blieb | marad - PRÄTERIUM\\
162 | brachte | hoz - PRÄTERIUM\\
163 | brach | eltör valamit - PRÄTERIUM\\
164 | brach | valami eltörik - PRÄTERIUM\\
165 | brannte | ég - PRÄTERIUM\\
166 | dachte | gondol - PRÄTERIUM\\
167 | drang | be-/áthatol - PRÄTERIUM\\
168 | durfte | szabad (csinálni - PRÄTERIUM\\
169 | bog ein | bekanyarodik - PRÄTERIUM\\
170 | lud ein | meghív - PRÄTERIUM\\
171 | schlief ein | elalszik - PRÄTERIUM\\
172 | stieg ein | beszáll vhová - PRÄTERIUM\\
173 | empfing | fogad - PRÄTERIUM\\
174 | empfahl | ajánl - PRÄTERIUM\\
175 | empfand | érez,észlel vmit - PRÄTERIUM\\
176 | erlosch | elalszik a tűz - PRÄTERIUM\\
177 | erschrak | megijed - PRÄTERIUM\\
178 | aß | eszik - PRÄTERIUM\\
179 | sah fern | TV-t néz - PRÄTERIUM\\
180 | fuhr | vezet - PRÄTERIUM\\
181 | fuhr  | utazik - PRÄTERIUM\\
182 | fiel | esik - PRÄTERIUM\\
183 | fing | (meg)fog,(el)kap - PRÄTERIUM\\
184 | fand | talál - PRÄTERIUM\\
185 | flog | repül - PRÄTERIUM\\
186 | foh | menekül - PRÄTERIUM\\
187 | floß | folyik - PRÄTERIUM\\
188 | fraß | zabál,állat eszik - PRÄTERIUM\\
189 | fror | fázik/megfagy - PRÄTERIUM\\
190 | gab | ad - PRÄTERIUM\\
191 | ging  | megy - PRÄTERIUM\\
192 | gelang | sikerül - PRÄTERIUM\\
193 | galt | érvényes - PRÄTERIUM\\
194 | genas | gyógyul - PRÄTERIUM\\
195 | genoß | élvez - PRÄTERIUM\\
196 | geschah | történik - PRÄTERIUM\\
197 | gewann | nyer - PRÄTERIUM\\
198 | goss | önt, tölt - PRÄTERIUM\\
199 | grub | ás - PRÄTERIUM\\
200 | griff | megragad, (meg)fog - PRÄTERIUM\\
201 | hatte | birtokol vmit, van neki - PRÄTERIUM\\
202 | hielt | tart - PRÄTERIUM\\
203 | hing | lóg, függ - PRÄTERIUM\\
204 | hob | emel - PRÄTERIUM\\
205 | hieß | hívják vhogyan - PRÄTERIUM\\
206 | half | segít - PRÄTERIUM\\
207 | kannte | ismer - PRÄTERIUM\\
208 | klang | cseng, kong - PRÄTERIUM\\
209 | kam | jön - PRÄTERIUM\\
210 | konnte | tud (csinálni), (csinál)hat,-het - PRÄTERIUM\\
211 | lud | megrak, megterhel  - PRÄTERIUM\\
212 | las | olvas - PRÄTERIUM\\
213 | lag | fekszik - PRÄTERIUM\\
214 | ließ | hagy - PRÄTERIUM\\
215 | lief | fut - PRÄTERIUM\\
216 | litt | szenved - PRÄTERIUM\\
217 | lieh | kölcsönöz - PRÄTERIUM\\
218 | log | hazudik - PRÄTERIUM\\
219 | mied | kerül vkit/vmit - PRÄTERIUM\\
220 | maß | mér (távolságot) - PRÄTERIUM\\
221 | musste | kell (csinálnia) - PRÄTERIUM\\
222 | mochte | szeret (csinálni) - PRÄTERIUM\\
223 | nahm | fog, vesz - PRÄTERIUM\\
224 | nannte | nevez - PRÄTERIUM\\
225 | riet | tanácsol - PRÄTERIUM\\
226 | ritt | lovagol - PRÄTERIUM\\
227 | rannte  | rohan - PRÄTERIUM\\
228 | rief | kiált, hív - PRÄTERIUM\\
229 | schuf | teremt - PRÄTERIUM\\
230 | schien | 1.fénylik, süt(a Nap) 2.úgy tűnik - PRÄTERIUM\\
231 | schob | tol - PRÄTERIUM\\
232 | schoss | lő - PRÄTERIUM\\
233 | schlief | alszik - PRÄTERIUM\\
234 | schlug | üt, ver - PRÄTERIUM\\
235 | schloss | zár - PRÄTERIUM\\
236 | schmolz | olvad - PRÄTERIUM\\
237 | schnitt | vág - PRÄTERIUM\\
238 | schrieb | ír - PRÄTERIUM\\
239 | schrie | kiabál - PRÄTERIUM\\
240 | schritt | lép(ked) - PRÄTERIUM\\
241 | schwieg | hallgat - PRÄTERIUM\\
242 | schwamm | úszik - PRÄTERIUM\\
243 | schwor | esküszik - PRÄTERIUM\\
244 | sah | lát, néz - PRÄTERIUM\\
245 | war | létige - PRÄTERIUM\\
246 | sang | énekel - PRÄTERIUM\\
247 | sank | süllyed - PRÄTERIUM\\
248 | saß | ül - PRÄTERIUM\\
249 | sollte | kell - PRÄTERIUM\\
250 | sprach | beszél - PRÄTERIUM\\
251 | stand | áll - PRÄTERIUM\\
252 | sprang | ugrik - PRÄTERIUM\\
253 | stach | megcsíp - PRÄTERIUM\\
254 | stahl | lop - PRÄTERIUM\\
255 | stieg | emelkedik - PRÄTERIUM\\
256 | starb | meghal - PRÄTERIUM\\
257 | stank | bűzlik - PRÄTERIUM\\
258 | stieß | lök - PRÄTERIUM\\
259 | strich | 1.simít 2.mázol 3. kihúz, kitöröl - PRÄTERIUM\\
260 | stritt | veszekedik - PRÄTERIUM\\
261 | rief | hív, kiabál - PRÄTERIUM\\
262 | trug | hord,visel,cipel - PRÄTERIUM\\
263 | traf | találkozik - PRÄTERIUM\\
264 | trieb | űz, hajt - PRÄTERIUM\\
265 | trat | tapos, tipor - PRÄTERIUM\\
266 | trat | lép,jár,megy - PRÄTERIUM\\
267 | trank | iszik - PRÄTERIUM\\
268 | stieg um | átszáll - PRÄTERIUM\\
269 | vergaß | elfelejt - PRÄTERIUM\\
270 | verließ | elhagy vkit/vmit - PRÄTERIUM\\
271 | verlor | elveszít vmit/vkit - PRÄTERIUM\\
272 | verstand | (meg)ért - PRÄTERIUM\\
273 | wuchs | nő - PRÄTERIUM\\
274 | wusch | mos - PRÄTERIUM\\
275 | wusste | tud vmit, vmiről - PRÄTERIUM\\
276 | wollte | akar - PRÄTERIUM\\
277 | zog | húz - PRÄTERIUM\\
278 | zwang | kényszerít - PRÄTERIUM\\
279 | h. angeboten | meg-/felkínál - PERFEKT\\
280 | h. angerufen | felhív telefonon - PERFEKT\\
281 | h. angefangen | elkezd(ődik) - PERFEKT\\
282 | i. angekommen | megérkezik - PERFEKT\\
283 | h. angezogen | felöltöztet - PERFEKT\\
284 | i. aufgestanden | felkel - PERFEKT\\
285 | i. ausgegangen | szórakozni megy - PERFEKT\\
286 | i. ausgestiegen | kiszáll - PERFEKT\\
287 | h. gebacken | süt (húst) - PERFEKT\\
288 | h. befohlen | parancsol - PERFEKT\\
289 | h. begonnen | elkezd(ődik) - PERFEKT\\
290 | h. gebissen | harap - PERFEKT\\
291 | h. bekommen | kap - PERFEKT\\
292 | h. borgen | rejt - PERFEKT\\
293 | h. betrogen | becsap - PERFEKT\\
294 | h. bewogen | indít vmire,motivál - PERFEKT\\
295 | h. gebogen | hajlít - PERFEKT\\
296 | h. geboten | nyújt, kínál - PERFEKT\\
297 | h. gebunden | (meg)köt - PERFEKT\\
298 | h. gebeten | kér - PERFEKT\\
299 | h. geblasen | fújni - PERFEKT\\
300 | i. geblieben | marad - PERFEKT\\
301 | h. gebracht | hoz - PERFEKT\\
302 | h. gebrochen | eltör valamit - PERFEKT\\
303 | i. gebrochen | valami eltörik - PERFEKT\\
304 | h. gebrannt | ég - PERFEKT\\
305 | h. gedacht | gondol - PERFEKT\\
306 | i. gedrungen | be-/áthatol - PERFEKT\\
307 | h. gedurft | szabad (csinálni - PERFEKT\\
308 | i. eingebogen | bekanyarodik - PERFEKT\\
309 | h. eingeladen | meghív - PERFEKT\\
310 | i. eingeschlafen | elalszik - PERFEKT\\
311 | i. eingestiegen | beszáll vhová - PERFEKT\\
312 | h. empfangen | fogad - PERFEKT\\
313 | h. empfohlen | ajánl - PERFEKT\\
314 | h. empfunden | érez,észlel vmit - PERFEKT\\
315 | i. erloschen | elalszik a tűz - PERFEKT\\
316 | i. erschrocken | megijed - PERFEKT\\
317 | h. gegessen | eszik - PERFEKT\\
318 | h. ferngesehen | TV-t néz - PERFEKT\\
319 | h. gefahren | vezet - PERFEKT\\
320 | i. gefahren | utazik - PERFEKT\\
321 | i. gefallen | esik - PERFEKT\\
322 | h. gefangen | (meg)fog,(el)kap - PERFEKT\\
323 | h. gefunden | talál - PERFEKT\\
324 | i. geflogen | repül - PERFEKT\\
325 | i. geflohen | menekül - PERFEKT\\
326 | i. geflossen | folyik - PERFEKT\\
327 | h. gefressen | zabál,állat eszik - PERFEKT\\
328 | i. gefroren | megfagy - PERFEKT\\
329 | h. gegeben | ad - PERFEKT\\
330 | i. gegangen | megy - PERFEKT\\
331 | i. gelungen | sikerül - PERFEKT\\
332 | h. gegolten | érvényes - PERFEKT\\
333 | i. genesen | gyógyul - PERFEKT\\
334 | h. genossen | élvez - PERFEKT\\
335 | i. geschehen | történik - PERFEKT\\
336 | h. gewonnen | nyer - PERFEKT\\
337 | h. gegossen | önt, tölt - PERFEKT\\
338 | h. gegraben | ás - PERFEKT\\
339 | h. gegriffen | megragad, (meg)fog - PERFEKT\\
340 | h. gehabt | birtokol vmit, van neki - PERFEKT\\
341 | h. gehalten | tart - PERFEKT\\
342 | h. gehangen | lóg, függ - PERFEKT\\
343 | h. gehoben | emel - PERFEKT\\
344 | h. geheißen | hívják vhogyan - PERFEKT\\
345 | h. geholfen | segít - PERFEKT\\
346 | h. gekannt | ismer - PERFEKT\\
347 | h. geklungen | cseng, kong - PERFEKT\\
348 | i. gekommen | jön - PERFEKT\\
349 | h. gekonnt | tud - PERFEKT\\
350 | h. geladen | megrak, megterhel - PERFEKT\\
351 | h. gelesen | olvas - PERFEKT\\
352 | h. gelegen | fekszik - PERFEKT\\
353 | h. gelassen | hagy - PERFEKT\\
354 | i. gelaufen | fut - PERFEKT\\
355 | h. gelitten | szenved - PERFEKT\\
356 | h. geliehen | kölcsönöz - PERFEKT\\
357 | h. gelogen | hazudik - PERFEKT\\
358 | h. gemieden | kerül vkit/vmit - PERFEKT\\
359 | h. gemessen | mér (távolságot) - PERFEKT\\
360 | h. gemusst | kell (csinálnia) - PERFEKT\\
361 | h. gemocht | szeret (csinálni) - PERFEKT\\
362 | h. genommen | fog, vesz - PERFEKT\\
363 | h. genannt | nevez - PERFEKT\\
364 | h. geraten | tanácsol - PERFEKT\\
365 | i. geritten | lovagol - PERFEKT\\
366 | i. gerannt | rohan - PERFEKT\\
367 | h. gerufen | kiált, hív - PERFEKT\\
368 | h. geschaffen | teremt - PERFEKT\\
369 | h. geschienen | 1.fénylik, süt(a Nap) 2.úgy tűnik - PERFEKT\\
370 | h. geschoben | tol - PERFEKT\\
371 | h. geschossen | lő - PERFEKT\\
372 | h. geschlafen | alszik - PERFEKT\\
373 | h. geschlagen | üt, ver - PERFEKT\\
374 | h. geschlossen | zár - PERFEKT\\
375 | i. geschmolzen/geschmelzt | olvad - PERFEKT\\
376 | h. geschnitten | vág - PERFEKT\\
377 | h. geschrieben | ír - PERFEKT\\
378 | h. geschrien | kiabál - PERFEKT\\
379 | i. geschritten | lép(ked) - PERFEKT\\
380 | h. geschwiegen | hallgat - PERFEKT\\
381 | i. geschwommen | úszik - PERFEKT\\
382 | h. geschworen | esküszik - PERFEKT\\
383 | h. gesehen | lát, néz - PERFEKT\\
384 | i. gewesen | van (=létige) - PERFEKT\\
385 | h. gesungen | énekel - PERFEKT\\
386 | i. gesunken | süllyed - PERFEKT\\
387 | h. gesessen | ül - PERFEKT\\
388 | h. gesollt | kell (csinálni) - PERFEKT\\
389 | h. gesprochen | beszél - PERFEKT\\
390 | h. gestanden | áll - PERFEKT\\
391 | i. gesprungen | ugrik - PERFEKT\\
392 | h. gestochen | megcsíp - PERFEKT\\
393 | h. gestohlen | lop - PERFEKT\\
394 | i. gestiegen | emelkedik - PERFEKT\\
395 | i. gestorben | meghal - PERFEKT\\
396 | h. gestunken | bűzlik - PERFEKT\\
397 | h. gestoßen | lök - PERFEKT\\
398 | h. gestrichen | 1.simít 2.mázol 3. kihúz, kitöröl - PERFEKT\\
399 | h. gestritten | veszekedik - PERFEKT\\
400 | h. gerufen | hív, kiabál - PERFEKT\\
401 | h. getragen | hord,visel,cipel - PERFEKT\\
402 | h. getroffen | találkozik - PERFEKT\\
403 | h. getrieben | űz, hajt - PERFEKT\\
404 | h. getreten | tapos, tipor - PERFEKT\\
405 | i.  getreten | lép,jár,megy - PERFEKT\\
406 | h. getrunken | iszik - PERFEKT\\
407 | i. umgestiegen | átszáll - PERFEKT\\
408 | h. vergessen | elfelejt - PERFEKT\\
409 | h. verlassen | elhagy vkit/vmit - PERFEKT\\
410 | h. verloren | elveszít vmit/vkit - PERFEKT\\
411 | h. verstanden | (meg)ért - PERFEKT\\
412 | i. gewachsen | nő - PERFEKT\\
413 | h. gewaschen | mos - PERFEKT\\
414 | h. gewusst | tud vmit, vmiről - PERFEKT\\
415 | h. gewollt | akar - PERFEKT\\
416 | h. gezogen | húz - PERFEKT\\
417 | h. gezwungen | kényszerít - PERFEKT\\
\end{exmp}

\end{document}
