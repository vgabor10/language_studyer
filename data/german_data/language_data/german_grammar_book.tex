\title{Névelők/Határozott névelő/Határozott névelő esetei}

GrammarItemIndex = 0

\begin{desc}
eset     | hímnem | nőnem | semleges nem | többes szám
alany    | der    | die   | das          | die
tárgy    | den    | die   | das          | die
részes   | dem    | der   | dem          | den -n
birtokos | des -s | der   | des -s       | der
\end{desc}

\begin{exmp}
1 | der | hím nem, alany eset
2 | die | nő nem, alany eset
3 | das | semleges nem, alany eset
4 | die | többes szám, alany eset
5 | den | hím nem, tárgy eset
6 | die | nő nem, tárgy eset
7 | das | semleges nem, tárgy eset
8 | die | többes szám, tárgy eset
9 | dem | hím nem, részes eset
10 | der | nő nem, részes eset
11 | dem | semleges nem, részes eset
12 | den -n | többes szám, részes eset
13 | des -s | hím nem, birtokos eset
14 | der | nő nem, birtokos eset
15 | des -s | semleges nem, birtokos eset
16 | der | többes szám, birtokos eset
\end{exmp}

\title{Névelők/Határozott névelő/Tárgy eset és részes eset - példák}

GrammarItemIndex = 30

\begin{desc}
\end{desc}

%Maklári 27. oldal
\begin{exmp}
1 | den Vater | az apát
2 | die Tische | az asztalokat
3 | das Kind | a gyereket
4 | den Kindern | a gyerekeknek
5 | der Oma | a nagyinak
6 | dem Großvater | a nagypapának
7 | die Tische | az asztalok
8 | das Auto | az autót
9 | den Stuhl | a széket
10 | den Schrank | a szekrényt
11 | den Tisch | az asztalt
12 | der Mutter | az anyának
13 | den Müttern | az anyáknak
14 | dem Kind | a gyereknek
15 | die Lampe | a lámpát
16 | dem Vater | az apának
17 | den Vätern | az apáknak
18 | das Haus | a házat
19 | den Hof | az udvart
\end{exmp}

\title{Névelők/Határozott névelő/Birtokos eset}

GrammarItemIndex = 3214

\begin{desc}
hímnem       | des -s
nőnem        | der
semleges nem | des -s
többes szám  | der
\end{desc}

\begin{exmp}
1 | der Sohn des Vaters | az apa fia
2 | das Fenster des Hauses | a ház ablaka
3 | die Tochter der Mutter | az anya lánya
4 | der Freund des Vaters | az apa barátja
5 | das Spielzeug des Kindes | a gyerek játéka
6 | die Scheibe des Wagens | a kocsi ablaka
7 | die Kleider der Kinder | a gyerekek ruhái
8 | der Bus der Lehrer | a tanárok busza
9 | die Freundin der Mutter | az anya barátnője
\end{exmp}

\title{Névelők/Határozatlan névelő/Határozatlan névelő esetei}

GrammarItemIndex = 1

\begin{desc}
          | hímnem   | nőnem | semleges nem | többes szám
Nominativ | ein      | eine  | ein          | keine
Akkusativ | einen    | eine  | ein          | keine
Dativ     | einem    | einer | einem        | keinen -n
Genitiv   | eines -s | einer | eines -s     | keiner
\end{desc}

\begin{exmp}
1 | ein | hím nem, alany eset
2 | eine | nő nem, alany eset
3 | ein | semleges nem, alany eset
4 | einen | hím nem, tárgy eset
5 | eine | nő nem, tárgy eset
6 | ein | semleges nem, tárgy eset
7 | einem | hím nem, részes eset
8 | einer | nő nem, részes eset
9 | einem | semleges nem, részes eset
10 | eines -s | hím nem, birtokos eset
11 | einer | nő nem, birtokos eset
12 | eines -s | semleges nem, birtokos eset
13 | keine | többes szám, alany eset
14 | keine | többes szám, tárgy eset
15 | keinen -n | többes szám, részes eset
16 | keiner | többes szám, birtokos eset
\end{exmp}

\title{Névelők/Határozatlan névelő/Határozatlan névelő esetei - Példák}

GrammarItemIndex = 31

\begin{desc}
          | hímnem   | nőnem | semleges nem | többes szám
Nominativ | ein      | eine  | ein          | keine
Akkusativ | einen    | eine  | ein          | keine
Dativ     | einem    | einer | einem        | keinen -n
Genitiv   | eines -s | einer | eines -s     | keiner
\end{desc}

%Maklári: 37. oldal
\begin{exmp}
1 | ein Mädchen | egy lány
2 | ein Kind | egy gyerek
3 | ein Vater | egy apa
4 | einen Wagen | egy kocsit
5 | einer Frau | egy nőnek (részére)
6 | einen Garten | egy kertet
7 | einem Kind | egy gyereknek
8 | ein Auto | egy autót
9 | ein Buch | egy könyvet
10 | ein Mann | egy férfi
11 | ein Haus | egy ház
12 | einen Schrank | egy szekrényt
13 | einem Mädchen | egy lánynak
14 | einem Mann | egy férfinak
15 | eine Frau | egy nő
16 | einer Lehrerin | egy tanárnőnek
17 | eine Lampe | egy lámpa
18 | einen Stuhl | egy széket
19 | einem Gast | egy vendégnek
20 | ein Fenster | egy ablakot
21 | einen Tisch | egy asztalt
22 | ein Junge | egy fiú
23 | ein Haus | egy házat
24 | ein Fenster | egy ablak
25 | einem Hund | egy kutyának
26 | ein Auto | egy autó
27 | einen Lehrer | egy tanárt
28 | ein Buch | egy könyv
29 | ein Mann | egy ember
30 | einem Lehrer | egy tanárnak
31 | ein Mädchen | egy lányt
32 | einer Mutter | egy anyának
33 | die Feder eines Vogels | egy madár tolla
34 | das Heft eines Schülers | egy diák füzete
35 | das Fenster eines Hauses | egy ház ablaka
36 | der Rock einer Frau | egy nő szoknyája
37 | die Frau eines Lehrers | egy tanár felesége
38 | das Heft einer Schülerin | egy diáklány füzete
39 | die Seiten eines Buches | egy könyv oldalai
40 | der Kuli einer Lehrerin | egy tanárnő tolla
41 | der Ball eines Kindes | egy gyerek labdája
\end{exmp}

\title{Névmások/Személyes névmás/Személyes névmás esetei}

GrammarItemIndex = 295361

\begin{desc}
nominativ/alany | akusativ/tárgy | dativ/részes | genitiv/birtokos
ich             | mich           | mir          | mein
du              | dich           | dir          | dein 
er              | ihn            | ihm          | sein
sie             | sie            | ihn          | ihr
es              | es             | ihm          | sein
wir             | uns            | uns          | unser
ihr             | euch           | euch         | euer
sie             | sie            | ihnen        | ihr
Sie             | Sie            | Ihnen        | Ihr 
\end{desc}

\begin{exmp}
1 | mich | engem
2 | dich | téged
3 | ihn | őt (himnem)
4 | sie | őt (nőnem)
5 | es | azt
6 | uns | minket
7 | euch | titeket
8 | sie | őket
9 | Sie | Önt, Önöket
10 | mir | nekem
11 | dir | neked
12 | ihm | neki (hímnem)
13 | ihr | neki (nőnem)
14 | ihm | neki (semleges nem)
15 | uns | nekünk
16 | euch | nektek
17 | ihnen | nekik
18 | Ihnen | Önnek, Önöknek
19 | mein | enyém
20 | dein | tiéd
21 | sein | övé (hímnem)
22 | ihr | övé (nőnem)
23 | sein | övé (semleges nem)
24 | unser | miénk
25 | euer | tiétek
26 | ihr | övék
27 | Ihr | Öné, Önöké
\end{exmp}

\title{Névmások/Személyes névmás/Személyes névmás tárgyesete}

GrammarItemIndex = 206354

\begin{desc}
nominativ/alany | akusativ/tárgy
ich             | mich
du              | dich
er              | ihn
sie             | sie
es              | es
wir             | uns
ihr             | euch
sie             | sie
Sie             | Sie
\end{desc}

%Maklári: 38. oldal
\begin{exmp}
1 | Wir hören euch. | Hallgatunk titeket.
2 | István sieht sie. | István nézi őt (n).
3 | Ich höre sie. | Hallom őket.
4 | Sie suchen uns. | Keresnek minket.
5 | Helga liebt ihn. | Helga szereti őt (f).
6 | Peter ruft mich. | Engem hív Peter.
7 | Wir fragen euch. | Titeket kérdezünk.
8 | Der Lehrer fragt mich. | A tanár engem kérdez.
9 | Ich bitte Sie. | Kérem Önt.
10 | Wir bitten Sie. | Kérjük Önöket.
11 | Hört ihr sie? | Halljátok őket?
12 | Mutter liebt ihn. | Anya szereti őt (f).
\end{exmp}

\title{Névmások/Személyes névmás/Személyes névmás részes esete}

GrammarItemIndex = 155

\begin{desc}
nominativ/alany | dativ/részes 
ich             | mir 
du              | dir 
er              | ihm 
sie             | ihn
es              | ihr 
wir             | uns 
ihr             | euch 
sie             | ihnen  
Sie             | Ihnen 
\end{desc}

%Maklári: 40. oldal
\begin{exmp}
1 | Er kauft uns etwas. | Vásárol (f) nekünk valamit.
2 | Ich gebe dir eine Ohrfeige. | Adok neked egy pofont.
3 | Johann und Wolfgang helfen ihnen. | Johann és Wolfgang segítenek nekik.
4 | Ich gebe euch Zeit. | Adok nektek időt.
5 | Wir kaufen Ihnen das Radio. | Megvesszük Önnek a rádiót.
6 | Pista schenkt ihr einen Rock. | Pista ajándékoz neki egy szoknyát.
7 | Das Mädchen gefällt mir. | Tetszik nekem a lány.
8 | Gefällst das Spielzeug dir? | Tetszik neked a játék?
9 | Gefallen das Auto Ihnen? | Tetszik Önnek az autó?
10 | Wir gratulieren ihnen. | Gratulálunk nekik.
11 | Ich gebe dir den Wagen. | Odaadom neked a kocsit.
12 | Ihr schenkt ihnen eine Torte. | Ajándékozok nekik egy tortát.
13 | Ich gebe ihr ein Buch. | Adok neki (n) egy könyvet.
14 | Anna hilft ihm Auto reparieren. | Anna segít neki (f) autót szerelni.
15 | Ich schenke ihr eine Bluse. | Ajándékozok neki (n) egy blúzt. (e Bluse)
16 | Wir geben euch Geld. | Adunk pénzt nektek.
17 | Wir gratulieren Ihnen. | Gratulálunk Önöknek.
18 | Sie gibt mir eine Uhr. | Ad (n) nekem egy órát.
19 | Er gibt uns das Buch. | Nekünk adja (f) a könyvet.
20 | Öffnest du mir die Tür? | Kinyitod nekem az ajtót?
21 | Ich erzähle dir die Geschichte. | Elmesélem neked a történetet.
\end{exmp}

\title{Névmások/Személyes névmás/Személyes névmás birtokos esete}

GrammarItemIndex = 18

\begin{desc}
ALANY | BIRTOKOS 
------+----------
ich   | mein
du    | dein
er    | sein
sie   | ihr
es    | sein
wir   | unser
ihr   | euer
sie   | ihr
Sie   | Ihr

Ha a birtok nőnemű vagy többes számú, akkor a birtokos névmáshoz
egy -e betűt illesztünk.
\end{desc}

%Maklári: 42. oldal
\begin{exmp}
1 | ihre Tochter | a lánya (nőé)
2 | seine Tochter | a lánya (f)
3 | ihr Sohn | a fia (n)
4 | sein Sohn | a fia (f)
5 | unsere Schwester | a testvérünk (n)
6 | euer Freund | barátotok
7 | euere Freundinnen | barátnőtök
8 | mein Wagen | a kocsim
9 | unsere Freunde | barátaink
10 | Ihr Haus | az Ön háza
11 | euere Kleider | ruháitok
12 | ihre Kleider | ruháik
13 | unser Vater | apánk
14 | euer Geschwister | testvéreitek (s Geschwister)
15 | ihre Freundin | barátnője (n)
16 | meine Freundin | barátnőm
17 | dein Freund | barátod
18 | Ihre Mutter | az Ön anyja
19 | Ihr Vater | az Ön apja
20 | unser Kind | gyerekünk
21 | euer Stuhl | széketek
22 | euere Stühle | székeitek
23 | ihr Tisch | asztaluk
24 | ihre Tische | asztalaik
25 | ihr Bruder | fiútestvére (n)
26 | seine Schwester | lánytestvére (f)
27 | Ihr Tisch | az Ön asztala
28 | seine Tasche | táskája (f)
29 | ihr Auto | autója (n)
30 | Ihr Haus | Önök háza
31 | euere Bücher | könyveitek
32 | euer Buch | könyvetek
33 | ihr Buch | könyvük
\end{exmp}

\title{Névmások/Birtokos névmás/Birtokos névmás ragozása}

GrammarItemIndex = 40

\begin{desc}
          | hímnem    | nőnem  | semleges nem | többes szám 
nominativ | mein      | meine  | mein         | meine 
akusativ  | meinen    | meine  | mein         | meine 
dativ     | meinem    | meiner | meinem       | meinen -n 
genitiv   | meines -s | meiner | meines -s    | meiner 
\end{desc}

%Maklári: 45. oldal
\begin{exmp}
1 | meinen Vater | az apámat
2 | deiner Mutter | az anyádnak
3 | meiner Großmutter | a nagymamámnak
4 | eueren Tisch | az asztalotokat
5 | unseren Freundinnen | a barátnőinknek
6 | unserer Freundin | a barátnőnknek
7 | Ihren Anzug | az Ön öltönyét
8 | meine Frau | a feleségemet
9 | mein Bruder | a fivérem
10 | meinem Bruder | a fivéremnek
11 | Ihre Frau | az Ön feleségét
12 | deinen Hund | a kutyádat
13 | euer Auto | az autótokat
14 | sein Haus | a házát (f)
15 | seiner Frau | a feleségének
16 | unseren Kindern | a gyerekeinknek
17 | euerem Kind | a gyereketeknek
18 | unsere Lampe | a lámpánkat
19 | deinem Hund | a kutyádnak
20 | deinen Freund | a barátodat
21 | unseren Eltern | a szüleinknek
22 | euere Kinder | a gyerekeiteket
23 | Ihrem Mann | az Önök férjének
24 | meinen Freun | a barátomat
\end{exmp}

\title{Névmások/Birtokos névmás/Birtokos névmás birtokos esete}

GrammarItemIndex = 50

\begin{desc}
nominativ/alany | genitiv/birtokos 
ich             | mein
du              | dein
er              | sein
sie             | ihr
es              | sein
wir             | unser
ihr             | euer
sie             | ihr
Sie             | Ihr
\end{desc}

%Maklári: 45. oldal
\begin{exmp}
1 | der Kind unserer Eltern | a szüleink gyereke
2 | der Sohn meines Vaters | az apám fia
3 | die Tochter meiner Mutter | az anyám lánya
4 | die Tür unseres Hauses | a házunk ajtaja
5 | die Lampe ihres Autos | az autójuk lámpája
6 | die Scheibe Ihres Wagens | az Ön kocsijának az ablaka
7 | die Bälle euerer Kinder | a gyerekeitek labdái
8 | die Shöne euerer Tochter | a lányaitok fiai
9 | der Freund ihres Mannes | a férjének a barátja
10 | das Rad deines Fahrrads | a kerékpárod kereke
11 | das Geld euerer Freunde | a barátaitok pénze
12 | das Zeit unseres Lehrers | a tanárunk ideje
13 | das Kleid seiner Freundin | a barátnőjének (f) a ruhája
14 | die Fenster ihrer Häuser | a házaiknak az ablakai
15 | der Knopf meiner Hose | a nadrágom gombja (r Knopf)
16 | das Ende unserer Aufgaben | a feladataink vége
17 | der Beginn unserer Leidens | a szenvedéseink kezdete (r Beginn, s Leiden)
18 | die Freundin seiner Frau | a feleségének a barátnője
\end{exmp}

\title{Névmások/Birtokos névmás/Birtokos névmás önálló alakja}

GrammarItemIndex = 423432

\begin{desc}
\end{desc}

\begin{exmp}
\end{exmp}

\title{Névmások/Visszaható névmás}

GrammarItemIndex = 873514

\begin{desc}
\end{desc}

\begin{exmp}
\end{exmp}

\title{Névmások/Mutató névmás}

GrammarItemIndex = 786532

\begin{desc}
\end{desc}

\begin{exmp}
\end{exmp}

\title{Névmások/Mutató névmás/Közelre mutatás - dieser/e/es}

GrammarItemIndex = 532876

\begin{desc}
Ha közelban levő dolgokra akarunk mutatni, akkor a dieser mutatónévmást használjuk.

Hasonlóan ragozzuk, mint a határozott névelőt:

  | HÍMNEM     | NŐNEM  | SEMLEGES NEM | TÖBBES SZÁM
--+------------+--------+--------------+--------------
N | dieser     | diese  | dieses       | diese
A | diesen     | diese  | dieses       | diese
D | diesem     | dieser | diesem       | diesen -en
G | dieses -es | dieser | dieses -es   | dieser

Pl.: * dieser Schüler ist frech - ez a tanuló pimasz
* diese Frau ist hübsch - ez a nő csinos
* dieses Kind ist schäflig - ez a gyerek álmos
* das Kind dieses Vaters - ennek az apának a gyereke
\end{desc}

%Maklári: 137. oldal
\begin{exmp}
1 | dieser Stuhl | ez a szék
2 | diesen Mann | ezt a férfit
3 | diesem Mädchen | ennek a lánynak
4 | dieses Wasser | ezt a vizet
5 | diesen Kindern | ezeknek a gyerekeknek
6 | dieses Fenster | ezt az ablakot
7 | diesen Mädchen | ezeknek a lányoknak
8 | diese Frauen | ezek a nők
9 | diese Männer | ezeket a férfiakat
10 | dieser Mensch | ez az ember
11 | dieser Frau | ennek a nőnek
12 | diesem Mann | ennek a férfinak
13 | dieses Buch | ezt a könyvet
14 | diese Bücher | ezeket a könyveket
15 | diese Taschen | ezek a táskák
16 | diese Stühle | ezeket a székeket
17 | diesen Stuhl | ezt a széket
18 | diesen Leuten | ezeknek az embereknek
19 | diese Fenster | ezeket az ablakokat
20 | dieses Auto | ez az autó
21 | diese Kinder | ezek a gyerekek
22 | diesen Tisch | ezt az asztalt
23 | diesen Schülern | ezeknek a diákoknak
24 | dieses Fenster | ez az ablak
25 | diesen Männern | ezeknek a férfiaknak
26 | diesen Lehrer | ezt a tanárt
27 | diese Männer | ezek a férfiak
28 | die Tochter dieser Mutter | ennek az anyának a lánya
29 | das Rad dieses Wagens | ennek a kocsinak a kereke
30 | die Wagen dieser Eltern | ezeknek a szülőknek a kocsijai
31 | die Tasche dieser Frau | ennek a nőnek a táskája
32 | die Tür dieser Wagen | ezeknek a kocsiknak az ajtaja
33 | das Auge dieses Mädchens | ennek a lánynak a szeme
34 | die Nase dieses Mannes | ennek a férfinak az orra
35 | die Tür dieses Hauses | ennek a háznak az ajtaja
36 | der Kuli dieses Mädchens | ennek a lánynak a tolla
37 | die Türen dieser Schränke | ezeknek a szekrényeknek az ajtajai
38 | die Kinder dieser Eltern | ezeknek a szülőknek a gyerekei
39 | die Bücher dieser Leute | ezeknek az embereknek a könyvei
40 | der Rock dieses Mädchens | ennek a lánynak a szoknyája
41 | das Ende dieser Aufgabe | ennek a feladatnak a vége (s Ende)
\end{exmp}

\title{Névmások/Mutató névmás/Távolra mutatás - jener/e/es}

GrammarItemIndex = 332876

\begin{desc}
Ha távolban levő dolgokra akarunk mutatni, akkor a jener/e/es mutatónévmást használjuk.

Hasonlóan ragozzuk, mint a határozott névelőt:

  | HÍMNEM    | NŐNEM | SEMLEGES NEM | TÖBBES SZÁM
--+-----------+-------+--------------+--------------
N | jener     | jene  | jenes        | jene
A | jenen     | jene  | jenes        | jene
D | jenem     | jener | jenem        | jenen -en
G | jenes -es | jener | jenes -es    | jener

Pl.: * Jener Hund läuft schnell. - Az a kutya gyorsan fut.
* Jenes Haus gefällt mir. - Az a ház tetszik nekem.
\end{desc}

%Maklári: 138. oldal
\begin{exmp}
1 | jene Katze | az a macska
2 | jenes Fenster | az az ablak
3 | jenes Mädchen | azt a lányt
4 | jenem Lehrer | annak a tanárnak
5 | jene Lampen | azokat a lámákat
6 | jenes Auto | azt az autót
7 | jene Bücher | azok a könyvek
8 | jenen Mann | azt a férfit
9 | jene Frau | azt a nőt
10 | jenem Hund | annak a kutyának
11 | jenen Kindern | azoknak a gyerekeknek
12 | jener Schrank | az a szekrény
13 | jenen Mädchen | azoknak a lányoknak
14 | jenen Jungen | azoknak  a fiúknak
15 | jenes Fernsehen | azt a TV-t
16 | jenen Eltern | azoknak a szülőknek
17 | jener Kaffee | az a kávé
18 | jene Bilder | azok a képek
19 | jene Lampe | aza lámpa
20 | jenen Schrank | azta szekrényt
21 | jene Bücher | azokat a könyveket
22 | der Sohn jenes Vaters | annak az apának a fia
23 | das Rad jenes Fahrrads | annak a kerékpárnak a kereke
24 | das Haus jener Frau | annak a nőnek a háza
25 | die Uhr jener Lehrerin | annak a tanárnőnek az órája
26 | die Bücher jenes Lehrers | annak a tanárnak a könyvei
27 | die Kleider jener Menschen | azoknak az embereknek a ruhái (r Mensch)
28 | die Eltern jenen Kindern | azoknak a gyerekeknek a szülei
29 | die Kinder jener Eltern | azoknak a szülőknek a gyerekei
30 | das Geld jenes Mädchens | annak a lánynak a pénze
31 | die Kinder jener Mutter | annak az anyának a gyerekei
32 | die Räder jenes Wagens | annak a kocsinak a kerekei
33 | die Beine jenes Tisches | annak az asztalnak a lábai (s Bein)
\end{exmp}

\title{Névmások/Határozatlan névmás}

GrammarItemIndex = 645347

\begin{desc}
          | hímnem | nőnem | semleges nem 
Nominativ | einer  | eine  | eines 
Akkusativ | einen  | eine  | eines 
Dativ     | einem  | einer | einem 

Pl.: Dort steht eines. - Ott áll egy.
\end{desc}

\begin{exmp}
\end{exmp}

\title{Birtokos szerkezetek/Többes birtokos szerkezet}

GrammarItemIndex = 432275

\begin{desc}
\end{desc}

\begin{exmp}
\end{exmp}

\title{Birtokos szerkezetek/Személynevek birtokos szerkezete}

GrammarItemIndex = 234578

\begin{desc}
Peters Freundin - Péter barátnője
\end{desc}

\begin{exmp}
\end{exmp}

\title{Birtokos szerkezetek/Birtokos szerkezet von + D-val}

GrammarItemIndex = 117336

\begin{desc}
Pl.: der Sohn von dem Vater - az apa fia
\end{desc}

%Maklári: 30. oldal
\begin{exmp}
1 | die Tochter von der Mutter | az anya lánya
2 | der Schuh von dem Vater | az apa cipője
3 | die Brille von den Kindern | a gyerekek szemüvege
4 | die Lampe von dem Einbrecher | a betörő lámpája (r Einbrecher)
5 | das Zeugnis von dem Schüler | a tanuló bizonyítványa (s Zeugnis)
6 | die Torte von der Großmutter | a nagymama tortája
7 | der Kuli von dem Schüler | a tanuló tolla
8 | die Spielzeuge von den Kindern | a gyerekek játékai
9 | die Mütze von dem Kontrolleur | az ellenőr sapkája
10 | der Humor von dem Kontrolleur | az ellenőr humora
\end{exmp}

\title{Módbeli segédigék/Módbeli segédigék ragozása}

GrammarItemIndex = 3

\begin{desc}
E/1 | kann   | darf   | mag   | will   | muss   | soll 
E/2 | kannst | darfst | magst | willst | musst  | sollst 
E/3 | kann   | darf   | mag   | will   | muss   | soll 
T/1 | können | dürfen | mögen | wollen | müssen | sollen 
T/2 | könnt  | dürft  | mögt  | wollt  | müsst  | sollt 
T/3 | können | dürfen | mögen | wollen | müssen | sollen 
\end{desc}

\begin{exmp}
1 | kann | tud, E/1
2 | kannst | tud, E/2
3 | kann | tud, E/3
4 | können | tud, T/1
5 | könnt | tud, T/2
6 | können | tud, T/3
7 | darf | szabad, E/1
8 | darfst | szabad, E/2
9 | darf | szabad, E/3
10 | dürfen | szabad, T/1
11 | dürft | szabad, T/2
12 | dürfen | szabad, T/3
13 | mag | szeret, E/1
14 | magst | szeret, E/2
15 | mag | szeret, E/3
16 | mögen | szeret, T/1
17 | mögt | szeret, T/2
18 | mögen | szeret, T/3
19 | will | akar, E/1
20 | willst | akar, E/2
21 | will | akar, E/3
22 | wollen | akar, T/1
23 | wollt | akar, T/2
24 | wollen | akar, T/3
25 | muss | kell (külső kényszer), E/1
26 | musst | kell (külső kényszer), E/2
27 | muss | kell (külső kényszer), E/3
28 | müssen | kell (külső kényszer), T/1
29 | müsst | kell (külső kényszer), T/2
30 | müssen | kell (külső kényszer), T/3
31 | soll | kell (belső kényszer), E/1
32 | sollst | kell (belső kényszer), E/2
33 | soll | kell (belső kényszer), E/3
34 | sollen | kell (belső kényszer), T/1
35 | sollt | kell (belső kényszer), T/2
36 | sollen | kell (belső kényszer), T/3
\end{exmp}

\title{Módbeli segédigék/Módbeli segédigék használata}

GrammarItemIndex = 4

\begin{desc}
Módbeli segédigék használata:
alany + módbeli segédige ragozva+ többi mondatrész + ige főnévi igenév alakban 

Módbeli segédigék:
* können - tud
* dürfen - szabad
* mögen - szeret, kedvel
* müssen - kell
* sollen - kell
* wollen - akar
\end{desc}

\begin{exmp}
1 | Ich will heute in das Kino gehen. | Ma moziba akarok menni.
2 | Ilona darf in das Theater gehen. | Ilonának szabad színházba mennie.
3 | Ich kann schnell Bier trinken. | Gyorsan tudok sört inni.
4 | Ich kann schwimmen. | Tudok úszni.
5 | Sie will kommen. | Akar jönni. (n)
6 | Wir mögen Fußball spielen. | Szeretünk focizni.
7 | Er muss nach Hause gehen. | Haza kell mennie. (f)
8 | Du musst Blume kaufen. | Virágot kell venned.
9 | Wir dürfen spielen. | Szabad játszanunk.
10 | Wir wollen fernsehen. | Akarunk tévézni. (fernsehen - TV-t néz)
11 | Wollt ihr in das Bett gehen? | Akartok ágyba menni?
12 | Er kann gut sprechen. | Jól tud beszélni. (f)
13 | Ihr dürft nach Hause gehen. | Haza mehettek.
14 | Mögt ihr Korbball spielen? | Szerettek kosárlabdázni? (Korbball spielen)
15 | Er soll nett sind. | Kedvesnek kell lennie. (f)
16 | Mögt ihr Eis essen? | Szerettek fagyit enni?
17 | Ich muss lernen. | Tanulnom kell.
18 | Sie muss einkaufen. | Be kell vásárolnia (n).
19 | Wollt ihr trinken? | Akartok inni?
20 | Sie mag Schach spielen. | Szeret (n) sakkozni.
21 | Sie wollen nach Hause gehen. | Haza akarnak menni.
22 | Darfst du rauchen? | Szabad cigarettáznod?
23 | Sie wollen schlafen. | Aludni akarnak.
24 | Sie müssen lernen. | Tanulniuk kell.
25 | Gizi will Wasser trinken. | Gizi vizet akar inni.
26 | Ihr musst höflich sein. | Udvariasnak kell lennetek. (höflich - udvarias)
\end{exmp}

\title{Módbeli segédigék/A möchten módbeli segédige}

GrammarItemIndex = 17

\begin{desc}
A möchten módbeli segédige a mögen (szeretni) ige feltételes módú alakja (tehát: szeretnék, szeretnél...).

A möchten ragozása:
ich möchte
du möchtest
er/sie/es möchte
wir möchten
ihr möchtet
sie/Sie möchten

Pl.: Ich möchte heute ins Theater gehen. - Szeretnék me színházba menni.
\end{desc}

%Maklári: 123. oldal
\begin{exmp}
1 | Möchtet ihr laufen? | Szeretnétek futni?
2 | Er möchte Fußball spielen. | Szeretne (f) focizni.
3 | Ich möchte schon schlafen. | Szeretnék már aludni.
4 | Sie möchten ein Haus bekommen. | Szeretnének egy Házat kapni.
5 | Möchten Sie nach Hause gehen? | Szeretne Ön haza menni?
6 | Möchtest du ein Glas Wasser trinken? | Szeretnél egy pohár vizet inni?
7 | Möchtet ihr Abend essen? | Szeretnétek vacsorázni?
8 | Ich möchte fahrradfahren. | Szeretnék biciklizni. (fahrradfahren - biciklizik)
9 | Ich möchte etwas essen. | Szeretnék valamit enni.
10 | Ich möchte ein Fahrrad haben. | Szeretnék egy kerékpárt.
11 | Ich möchte in die Schule nicht gehen. | Nem szeretnék iskolába menni.
12 | Möchten Sie hier schlafen? | Szeretne (Ön) itt aludni?
13 | Möchtet ihr eine Tasse Tee trinken? | Szeretnétek egy csésze teát inni?
14 | Sie möchten morgen spazieren. | Szeretnének holnap sétálni.
15 | Möchtest du nach Deutschland reisen? | Szeretnél Németországba utazni? (reisen)
16 | Möchtet ihr Mittag essen? | Szeretnétek ebédelni? (Mittag essen)
17 | Ich möchte nach Hause gehen. | Haza szeretnék menni.
18 | Ich möchte sie sehen. | Szeretném látni őt (n).
\end{exmp}

\title{Módbeli segédigék/A módbeli segédigék másodlagos jelentése/A valószínűség fokozatai}

GrammarItemIndex = 543358

\begin{desc}
müssen -> egész bitzos, kétségtelen
dürfen -> valószínűleg
können -> lehet, lehetséges
mögen -> talán
\end{desc}

%Maklári: 126. oldal
\begin{exmp}
1 | Er muss zu hause sein. | Egész biztos, hogy otthon van.
2 | Er darf zu Hause sein. | Valószínűleg otthon van.
3 | Er kann zu Hause sein. | Lehet, hogy otthon van.
4 | Er mag zu Hause sein. | Talán otthon van.
\end{exmp}

\title{Módbeli segédigék/A módbeli segédigék másodlagos jelentése/Állítólagosság}

GrammarItemIndex = 4235

\begin{desc}

sollen - állítólag, szóbeszéd szerint
wollen - azt állítja, hogy; állítása szerint

\end{desc}

%Maklári: 128. oldal
\begin{exmp}
1 | In Peter soll das ganze Dorf verliebt sein. | Péterbe állítólag az egész falu szerelmes.
2 | In Peter will das ganze Dorf verliebt sein. | Péter azt állítja, hogy az egész falu belé szerelmes.
\end{exmp}

\title{Prepozíciók/Hol? kérdésre felelő prepozíciók}

GrammarItemIndex = 161485

\begin{desc}
A hol? kérdésre felelő prepozíciók részes esettel állnak.

Példák ilyen prepozíciókra:
* auf + D ------> -on, -en, -ön (vízszintes felületen)
* an + D -------> -on, -en, -ön (függőleges felületen)
* in + D -------> -ban, -ben
* vor + D ------> előtt
* hinter + D ---> mögött
* unter + D ----> alatt
* über + D -----> felett
* neben + D ----> mellett
* zwischen + D -> között
\end{desc}

%Maklári: 53. oldal
\begin{exmp}
1 | auf dem Stuhl | a széken
2 | an der Wand | a falon
3 | unter dem Tisch | az asztal alatt
4 | neben der Vase | a váza mellett
5 | über der Frau | a nő felett
6 | vor den Häusern | a házak előtt
7 | zwischen den Lampen | a lámpák között
8 | unter dem Baum | a fa alatt
9 | an den Bildern | a képeken
10 | vor der Mutter | az anya előtt
11 | hinter dem Schrank | a szekrény mögött
12 | auf dem Teppich | a szőnyegen
13 | in dem Haus | a házban
14 | in der Wohnung | a lakásban
15 | unter den Betten | az ágyak alatt
16 | zwischen der Tür und der Treppe | az ajtó és a lépcső között
17 | vor dem Haus | a ház előtt
18 | an dem Baum | a fán
19 | auf der Straße | az utcán
20 | vor der Tür | az ajtó előtt
21 | unter dem Teppich | a szőnyeg alatt
22 | in dem Wagen | a kocsiban
23 | neben den Lampen | a lámpák mellett
24 | in dem Bus | a buszban
25 | an dem Vorrang | a függönyön (r Vorhang)
26 | neben dem Mädchen | a lány mellett
27 | zwischen dem Kind und dem Stuhl | a gyerek és a szék között
28 | an dem Tafel | a táblán
29 | zwischen der Tür un dem Fenster | az ajtó és az ablak között
30 | an der Uhr | az órán
31 | über dem Haus | a ház felett
32 | zwischen dem Stuhl und dem Bett | a szék és az ágy között
\end{exmp}

\title{Prepozíciók/Hol? kérdésre felelő prepozíciók/Prepozíciók összevonása}

GrammarItemIndex = 543285

\begin{desc}
an dem = am
in dem = im
vor dem = vorm (beszédben)
hinter dem = hinterm (beszédben)
unter dem = unterm (beszédben)
über dem = überm (beszédben)
\end{desc}

\begin{exmp}
\end{exmp}

\title{Prepozíciók/Hová? kérdésre felelő prepozíciók}

GrammarItemIndex = 423265

\begin{desc}
A hová kérdésre felelő prepozíciók tárgy esettel állnak.

auf + A      | -ra, -re (vízszintes felületre)
an + A       | -ra, -re (függőleges felületre)
in + A       | -ba, -be
vor + A      | elé
hinter + A   | mögé
unter + A    | alá
über + A     | fölé
neben + A    | mellé
zwischen + A | közé
\end{desc}

%Maklári: 54. oldal
\begin{exmp}
1 | an die Wand | a falra
2 | in die Küche | a konyhába
3 | auf den Schrank | a szekrényre
4 | neben den Tisch | az asztal mellé
5 | vor das Zimmer | a szoba elé
6 | in die Schränke | a szekrényekbe
7 | auf den Tisch | az asztalra
8 | hinter die Wand | a fal mögé
9 | unter das Buch | a könyv alá
10 | über das Haus | a ház fölé
11 | hinter die Leute | az emberek mögé (e Leute)
12 | in den Wagen | a kocsiba
13 | auf den Boden | a padlóra
14 | an den Vorhang | a függönyre
15 | in das Zimmer | a szobába
16 | neben das Auto | az autó mellé
17 | über den Garten | a kert fölé
18 | unter den Boden | a föld alá
19 | neben die Garage | a garázs mellé
20 | zwischen das Zimmer und die Küche | a szoba és a kunyha közé
21 | zwischen die Stühle | a székek közé
22 | zwischen das Bett un die Tür | az ágy és a székek közé
23 | vor die Küche | az konyha elé
24 | an die Wände | a falakra
25 | in die Betten | az ágyakba
26 | auf das Auto | az autóra
27 | vor die Lehrerin | a tanárnő elé
28 | unter den Teppich | a szőnyeg alá
29 | zwischen die Lampe und die Blume | a lámpa és a virág közé
\end{exmp}

\title{Prepozíciók/Hová? kérdésre felelő prepozíciók/Prepozíciók összevonása}

GrammarItemIndex = 1236

\begin{desc}
* auf das = aufs
* an das = ans
* in das = ins
* vor dem = vors (beszédben)
* hinter das = hinters (beszédben)
* unter das = unters (beszédben)
* über das = übers (beszédben)
\end{desc}

\begin{exmp}
\end{exmp}

\title{Prepozíciók/Prepozíciók kizárólag tárgy esettel}

GrammarItemIndex = 5

\begin{desc}
* für - -ért, részére, számára
* ohne - nélkül
* gegen - ellen, körül (időben)
* bis - -ig
* durch - át, keresztül, által
* um - -kor, körül (térben)
\end{desc}

%Maklári: 59. oldal
\begin{exmp}
1 | für den Vater | az apa részére
2 | für die Mutter | az anya részére
3 | ohne Geld | pénz nélkül
4 | ohne Tasche | táska nélkül
5 | gegen den Freund | a barát ellen
6 | gegen den Feind | az ellenséggel szemben
7 | um das Haus | a ház körül
8 | durch meine Schwester | a nővérem által
9 | durch den Tunnel | keresztül az alagúton
10 | durch die Eltern | a szülőkön keresztül
11 | ohne Schuh | cipő nélkül
12 | bis Budapest | Budapestig
13 | bis 5 Uhr | 5 óráig
14 | um den Garten | a kert körül
15 | für das Mädchen | a lánynak
16 | für fünf Forint | öt forintért
17 | für einen Wagen | egy kocsiért
18 | ohne Lampe | lámpa nélkül
19 | um die Stadt | a város körül
20 | ohne die Schweigermutter | az anyós nélkül
21 | durch einen Freund | egy barát által
22 | um einen Tisch | egy asztal körül
23 | durch ein Haus | egy házon át
24 | durch meinen Vater | apám által
25 | ohne Heft | füzet nélkül
26 | für einen Apfel | egy almáért
27 | durch das Dorf | keresztül a falun
28 | durch den Park | keresztül a parkon
29 | bis zwei | kettőig
30 | um meinen Mann | a férjem körül
31 | gegen die Eltern | a szülőkkel szemben
32 | um einen Garten | egy kert körül
33 | durch den Weg | keresztül az úton
34 | gegen mich | ellenem
35 | für ihn | érte (hímnem)
36 | ohne uns | nélkülünk
37 | ohne sie | nélküle (nőnem)
38 | um euch | körülöttetek
39 | für uns | értünk
40 | für Sie | Önért
41 | um uns | körülöttünk
42 | durch mich | általam
43 | für sie | részükre
44 | gegen dich | ellened
45 | um uns | körülöttünk
46 | ohne sie | nélkülük
47 | gegen Sie | Ön ellen
48 | ohne dich | nélküled
49 | für uns | nekünk
50 | für euch | értetek
51 | durch Sie | Ön által
52 | für dich | érted
53 | ohne mich | nélkülem
54 | um mich | körülöttem
55 | für sie | számukra
56 | gegen euch | ellenetek
57 | gegen ihn | ellene (hímnem)
58 | um dich | körülötted
59 | um sie | körülötte (nőnem)
60 | durch sie | általa (nőnem)
61 | gegen uns | ellenünk
62 | für mich | értem
63 | für euch | részetekre
\end{exmp}

\title{Prepozíciók/Prepozíciók kizárólag részes esettel}

GrammarItemIndex = 13

\begin{desc}
* aus = -ból, -ből
* bei = -nál, -nél
* mit = -val, -vel
* nach = untán; -ba, -be (ország, város)
* von = -tól, -től, -ról, -ről
* zu = -hoz, -hez, -höz
* seit = óta
* gegenüber = szemben, átellenben
\end{desc}

%Maklári: 61. oldal
\begin{exmp}
1 | aus der Küche | a konyhából
2 | aus dem Zimmer | a szobából
3 | bei der Großmutter | a nagymamánál
4 | mit dem Kuli | a tollal
5 | nach der Frühstück | reggeli után
6 | nach Budapest | Budapestre
7 | nach Vien | Bécsbe
8 | zu dem Lehrer | a tanárhoz
9 | vod dem Schrank | a szekrénytől
10 | zu der Oma | a nagyihoz
11 | mit dem Bus | busszal
12 | gegenüber der Tür | az ajtóval szemben
13 | von dem Gast | a vendégtől
14 | seit zwei Stunden | két órája
15 | bei der Firma | a cégnél
16 | von dem Tisch | az asztalról
17 | von dem Schrank | a szekrényről
18 | gegenüber dem Haus | a házzal szemben
19 | seit dem Frühstück | a reggeli óta
20 | aus dem Wagen | a kocsiból
21 | seit vier Uhr | négy órája
22 | nach München | Münchenbe
23 | nach dem Mittagessen | ebéd után
24 | seit ein Jahr | egy éve
25 | bei dem Schriftsteller | az írónál
26 | mit dem Vater | az apával
27 | mit der Mutter | az anyával
28 | nach Ungarn | Magyarországra
29 | von dem Freund | a baráttól
30 | bei der Freundin | a barátnőnél
31 | zu dem Fenster | az ablakhoz
32 | nach der Schule | iskola után
33 | gegenüber dem Sohn | a fiával szemben
34 | nach Deutschland | Németországba
35 | mit einer Strassenbahn | egy villamossal
36 | seit zehn Uhr | tíz órája
37 | zu der Tür | az ajtóhoz
38 | beu uns | nálunk
39 | zu euch | hozzátok
40 | von dir | tőled
41 | zu dir | hozzád
42 | aus ihm | belőle (s)
43 | mit uns | velünk
44 | zu mir | hozzám
45 | von ihr | tőle (n)
46 | mit euch | veletek
47 | mit Ihnen | Önnel
48 | von Ihnen | Önöktől
49 | mit dir | veled
50 | zu ihnen | hozzájuk
51 | bei uns | nálunk
52 | zu ihr | hozzá (n)
53 | mit ihm | vele (f)
54 | nach dir | utánad
55 | mit uns | velünk
56 | zu Ihnen | Önhöz
57 | bei euch | nálatok
58 | zu uns | hozzánk
59 | zu ihm | hozzá (f)
60 | bei ihm | nála (f)
61 | mit ihr | vele (n)
62 | bei dir | nálad
63 | von Ihnen | Öntől
64 | nach euch | utánatok
65 | von mir | tőlem
66 | bei Ihnen | Önnél
67 | bei ihr | nála
68 | bei mir | nálam
69 | von ihm | tőle (f)
\end{exmp}

\title{Prepozíciók/Prepozíciók kizárólag birtokos esettel}

GrammarItemIndex = 536851

\begin{desc}
* statt = helyett
* trotz = ellenére
* wegen = miatt
* innerhalb = vmin belül
* außerhalb = vmin kívül
* während = alatt (időben)
\end{desc}

%Maklári: 63. oldal
\begin{exmp}
1 | wegen des Wetters | az időjárás miatt
2 | wegen des Krigs | a háború miatt
3 | trotz der Kälte | a hideg ellenére (e Kälte)
4 | außerhalb der Stadt | a városon kívül
5 | wegen des Verkäufers | az eladó miatt
6 | während der Deutschstunde | a németóra alatt
7 | statt des Mittagessens | az ebéd helyett
8 | während des Urlaubs | a nyaralás alatt
9 | während des Mahls | az étkezés alatt (s Mahl)
10 | statt des Ausflugs | a kirándulás helyett
11 | innerhalb des Dorfes | a falun bellül
12 | innerhalb des Gartens | a kerten bellül
13 | während der Reise | az utazás alatt (e Reise)
14 | statt des Brotes | a kenyér helyett
15 | statt der Schule | iskola helyett
16 | innerhalb zwei Stunden | két órán bellül
17 | statt des Kakaos | a kakaó helyett
18 | während des Films | a film alatt
19 | trotz des Regens | az eső ellenére
20 | während der Stunde | az óra alatt
21 | statt des Tees | tea helyett
22 | außerhalb des Landes | az országon kívül
23 | innerhalb der vier Wand | a négy fal között (innerhalb)
24 | während der Vorstellung | az előadás alatt (e Vorstellung)
25 | trotz des Befehls | a parancs ellenére
26 | während zwei Stunden | két óra alatt
27 | während des Krieges | a háború alatt
28 | trotz des Verbots | a tiltás ellenére (s Verbot)
\end{exmp}

\title{Prepozíciók/Kettős elöljárószavak}

GrammarItemIndex = 532386

\begin{desc}
A kettős elöljárószavak körbeveszik a főnevet, vagy a névmást.

Példák kettős elöljárószavakra:
* um + A ... herum -----> vmi körül
* an + D ... vorbei ----> mellet el
* an + D ... entlang ---> mentén
* mit + D ... zusammen -> -val, -vel, együtt
* auf + A ... zu -------> felé
* von + D ... an/ab ----> -tól, -től
* um + G ... willen ----> kedvéért

Pl.: * Ich fahre um die Stadt herum. - A város körül megyek.
* Ich gehe am Rathaus vorbei. - Elmegyek a városháza mellet.
* Sie spazieren am Ufer entlang. - A part mentén sétálnak.
* Er kommt mit Mutti zusammen. - Anyuval együtt jön.
* Die Direktorin kommt auf uns zu. Az igazgatónő felénk jön.
* Vom 1. März an wohnt er hier. - Március elsejétől itt lakik.
* Um seines Freundes willen komme ich. - A barátja kedvéért jövök.
\end{desc}

\begin{exmp}
1 | an mir vorbei | mellettem el
2 | um dich herum | körülötted
3 | mit ihr zusammen | vele (n) együtt
4 | um der Mutter willen | az anya kedvéért
5 | an seinem Haus vorbei | a háza (f) mellet el
6 | um unseren Wagen herum | a kocsink körül
7 | mit ihrem Vater zusammen | az apjával (n) együtt
8 | auf dich zu | feléd
9 | mit uns zusammen | velünk együtt
10 | um uns herum | körülöttünk
11 | an mir vorbei | mellettem el
12 | auf seinen Sohn zu | a fia (f) felé
13 | um ihrer Tochter willen | a lánya (n) kedvéért
14 | mit unseren Eltern zusammen | a szüleinkel együtt
15 | an dem Fluss entlang | a folyó mentén (r Fluss)
16 | an der Wand entlang | a fal mentén
17 | auf uns zu | felénk
18 | mit seiner Mutter zusammen | az anyjával (f) együtt
19 | von April 12 an | április 12-től
20 | um die Schule herum | az iskola körül
21 | von November 23 an | november 23-tól
\end{exmp}

\title{Prepozíciók/Az auf és an közti különbség}

GrammarItemIndex = 246365

\begin{desc}
auf + D/A használatos:
* vízszintes felületeken: auf dem Boden (a padlón);
* kiemelkedő tereptárgyaknál: auf dem Dach (a tetőn);
* nyílt, fedetlen területeken: auf dem Sportplatz (a sportpályán);
* intézményeknél: auf der Post (a postán);

an + D/A használatos:
* függőleges felületen: an der Wand (a falon);
* víznél, vízparton: an der Küste (a tengerparton);
* vminek a szélén: an der Ecke (a sarkon);
* vmi mellet közvetlenül: am Schran (a szekrénynél);
\end{desc}

\begin{exmp}
\end{exmp}

\title{Prepozíciók/A zwischen és an unter közti különbség}

GrammarItemIndex = 786385

\begin{desc}
\end{desc}

\begin{exmp}
\end{exmp}

\title{Kötőszavak/Kötőszavak egyenes szórenddel}

GrammarItemIndex = 536812

\begin{desc}
A következő kötöszavak után van egyenes szórend (usoda):
* und - és
* sondern - hanem
* oder - vagy
* denn - mert
* aber  de
\end{desc}

%Maklári: 144. oldal
\begin{exmp}
1 | Wir kaufen und ihr kocht. | Mi vásárolunk és ti főztök.
2 | Ich bin nervös, denn mein Freund kommt nicht. | Ideges vagyok, mert nem jön a barátom. (nervös)
3 | Ich gehe, oder du kommst. | Én megyek, vagy te jössz.
4 | Er ist schlecht, aber du bekommst die Ohrfeige. | Ö (f) rossz, de te kapod a pofont. (schlecht, bekommen)
5 | Ich bleibe zu Hause, denn ich bin krank. | Otthon maradok, mert beteg vagyok. (krank)
6 | Ich spüle heute und du kaufst ein. | Ma én mosogatok és te vásárolsz be. (spülen)
7 | Ich bin müde, aber ich gehe. | Fáradt vagyok, de megyek.
8 | Peter isst die Suppe nicht, sondern er trinkt. | Peter nem eszi a levest, hanem issza.
9 | Wir besuchen sie nicht, denn sie sind nicht zu Hause. | Nem látogatjuk meg őket, mert nincsenek otthon.
10 | Wir reisen nach Italien, aber wir bleiben nur zwei Tage. | Olaszországba utazunk, de csak két napot maradunk. (reisen, Italien)
11 | Wir bleiben hier nicht, sondern wir gehen nach Hause. | Nem maradunk itt, hanem haza megyünk.
12 | Die Gäste steigen in den Bus ein und der Bus fährt ab. | A vendégek beszállnak a buszba és a busz elindul. (einsteigen, abfahren)
13 | Ich arbeite und sie hört Musik. | Én dolgozom és ő (n) zenét hallgat.
\end{exmp}

\title{Kötőszavak/Kötőszavak egyenes szórenddel/Alany elhagyása und kötőszó esetén}

GrammarItemIndex = 645275

\begin{desc}
Ha mindkét tagmondatban ugyanaz az alany, akkor nem kell az und után újra kiírni.
\end{desc}

\begin{exmp}
1 | Ich bleibe hier und schreibe ein Brief. | Itt maradok és írok egy levelet.
2 | Ich stehe auf und gehe in Geschäft. | Felkelek és boltba megyek.
3 | Klaus geht in das Bett und liest ein Buch. | Klaus ágyba megy és olvas egy könyvet.
4 | Der Verkäufer begrüßt mir und öffnet die Tür. | Az eladó köszön nekem és kinyitja az ajtót. (begrüßen)
5 | Er isst Brot und trinkt Wasser. | Kenyeret eszik (f) és vizet iszik.
6 | Wir gehen morgen und kommen übermorgen. | Holnap megyünk és holnapután jövünk.
7 | Ich schreibe die Hausaufgabe und spiele danach Fußball. | Megírom a házit és azután focizok. (danach)
8 | Johann ruft seine Freundin an und spricht mit ihr. | Johann felhívja a barátnőjét és beszél vele.
9 | Sie lebt allein und arbeitet viel. | Egyedül él (n) és sokat dolgozik. (allein)
10 | Meine Freunde essen das Mittagessen und gehen nach Hause. | A barátaim megeszik az ebédet és haza mennek.
11 | Er hört das Radio und sieht fern. | Hallgatja (f) a rádiót és nézi a tévét. (fernsehen)
\end{exmp}

\title{Kötőszavak/Kötőszavak fordított szórenddel}

GrammarItemIndex = 116354

\begin{desc}
Képzése: kötőszó + ige + alany + többi mondatrész

A következő kötöszavak után van fordított szórend:
* trotzdem - mégis
* dann - aztán, akkor
* sonst - különben
* darum - ezért
* deshalb - ezért

Pl.: Sie ist krank, darum sucht sie einen Arzt. - Beteg, ezért keres egy orvost.
\end{desc}

%Maklári: 147. oldal
\begin{exmp}
1 | Ich gehe nach Hause, dann esse ich. | Haza megyek, aztán eszek.
2 | Ich esse das Mittagessen, dann spazieren wir. | Megeszem az ebédet, aztán sétálunk. (s Mittagessen)
3 | Das Wetter ist schön, deshalb bleiben wir nicht zu Hause. | Az idő szép, ezért nem maradunk otthon.
4 | Du schreibst die Lektion, sonst bekommst du eine Eins. | Megírod a leckét, különben kapsz egy egyest. (e Lektion, bekommen, e Eins)
5 | Wir lernen klein, dann spielen wir Fußball. | Tanulunk egy kicsit, aztán focizunk.
6 | Wir arbeiten, dann gehen wir in die Kneipe. | Dolgozunk, aztán megyünk a kocsmába.
7 | Er arbeitet viel, darum ist er reich. | Sokat dolgozik (f), ezért gazdag.
8 | Er arbeitet viel, trotzdem ist er arm. | Sokat dolgozik (f), mégis szegény.
9 | Peter ist krank, trotzdem kommt er mit uns. | Peter beteg, mégis velünk jön.
10 | Er trinkt viel, trotzdem ist er durstig. | Sokat iszik (f), mégis szomjas.
11 | Mein Freund lernt viel, trotzdem ist er dumm. | A barátom sokat tanul, mégis buta. (dumm)
12 | Du machst Ordnung in deinem Zimmer, sonst gehst du nicht in Schwimmbad. | Rendet csinálsz a szobádban, különben nem mész uszodába. (s Schwimmbad)
13 | Er gibt mir Schokolade nicht, deshalb gehe ich nicht zu ihm. | Nem ad (f) nekem csokoládét, ezért nem megyek hozzá.
14 | Du isst das Abendessen, sonst kommst du nicht mit mir. | Megeszed a vacsorát, különben nem jössz velem. (s Abendessen)
15 | Ich stehe auf, dann trinke ich Tee. | Felkelek, aztán teát iszok.
16 | Ich schlafe viel, trotzdem bin ich müde. | Sokat alszom, mégis fáradt vagyok.
\end{exmp}

\title{Kötőszavak/KATI szórend}

GrammarItemIndex = 10

\begin{desc}
Képzése:
főmondat + kötőszó + alany + többi mondatrész + ige ragozva.

A következő kötöszavak után van KATI szórend:
* dass - hogy
* ob - vajon
* als - amikor
* wenn - ha
* während - amíg
* weil - mert
* obwohl - habár

Az összes kérdőnévmás után KATI szórend van:
* wo - hol
* wann - mikor
* warum - miért
* wer - ki
* was - mi
* wie - hogyan

Pl.: Ich weiß, dass er um 5 Uhr kommt. - Tudom, hogy 5 órakor jön.
\end{desc}

%Maklári: 149. oldal
\begin{exmp}
1 | Ich weiß, dass du es schon weißt. | Tudom, hogy ezt már tudod.
2 | Er hört, dass wir bald kommen. | Hallja (f), hogy nemsokára jövünk. (bald)
3 | Ich lese, während du in deinem Zimmer schläfst. | Olvasok, amíg a szobádban alszol.
4 | Sie ist traurig, weil ihre Mutter nicht kommt. | Szomorú (n), mert nem jön az anyja.
5 | Ich sage, wenn er nach Hause kommt. | Mondom, ha haza jön (f).
6 | Wir wissen nicht, wo ihr wohnt. | Nem tudjuk, hol laktok.
7 | Ich weiß nicht, wer sie ist. | Nem tudom, ki ő (n).
8 | Wir hören, wenn er klingelt. | Halljuk, ha csönget (f). (klingeln)
9 | Ich komme, wenn ich kann. | Jövök, ha tudok.
10 | Ich möchte wissen, ob er heute zu uns kommt. | Szeretném tudni, vajon ma eljön-e (f) hozzánk. (wissen)
11 | Weißt du nicht, wie sie heißt? | Nem tudod, hogy hívják (n)?
12 | Ich glaube, dass wir noch Zeit haben. | Azt hiszem, hogy még van időnk.
13 | Ich gebe ihm alles, weil er mein Freund ist. | Mindent neki (f) adok, mert a barátom.
14 | Weißt du, wer seine Eltern sind? | Tudod, kik a szülei (f)?
15 | Wir wissen nicht, wann der Chef in das Büro kommt. | Nem tudjuk, mikor jön a főnök az irodába. (s Büro)
16 | Er ist nicht sicher, ob er noch lebt. | Nem biztos abban (f), hogy él-e (f) még.
17 | Hörst du, als die Kinder in der Kirche singen? | Hallod, amikor a templomban énekelnek a gyerekek? (als, e Kirche)
18 | Ich sage dir, wenn du zu uns kommst. | Elmondom neked, ha eljössz hozzánk.
\end{exmp}

\title{Kötőszavak/KATI szórend/KATI-s mellékmondat a mondat elején}

GrammarItemIndex = 6592

\begin{desc}
Ha KATI-s mellékmondat a mondat elején áll, akkor az utána következő főmondat szórendje fordított lesz.

Pl.: Wann er mit seiner Freundin kommt, weiß ich nicht.

Pl.: Wenn sie müde ist, trinkt sie einen Kaffe.
\end{desc}

%Maklári: 152. oldal
\begin{exmp}
1 | Dass sie heißt, können wir nicht. | Hogy hogy hívják, fogalmunk sincs. !!!
2 | Warum sie ruft mich nicht an, weißt der Teufel. | Hogy miért nem hív fel, tudja az ördög.
\end{exmp}

\title{Az ige ragozása jelen időben/Szabályos ragozású igék}

GrammarItemIndex = 446658

\begin{desc}
ich komme
du kommst
er/sie/es kommt
wir kommen
ihr kommt
sie/Sie kommen
\end{desc}

%Maklári 10.oldal
\begin{exmp}
1 | ich höre | én hallok
2 | du hörst | te hallassz
3 | er hört | ő hall (f)
4 | sie hört | ö hall (n)
5 | es hört | ez hall
6 | wir hören | mi hallunk
7 | ihr hört | ti hallotok
8 | sie hören | ők hallanak
9 | Sie hören | Ön hall
10 | ich trinke | én iszok
11 | du trinkst | te iszol
12 | er trinkt | ő (f) iszik
13 | sie trinkt | ő (n) iszik
14 | es trinkt | ez iszik
15 | wir trinken | mi iszunk
16 | ihr trinkt | ti isztok
17 | sie trinken | ők isznak
18 | Sie trinken | Ön iszik
\end{exmp}

\title{Az ige ragozása jelen időben/Igeragozás d-re, t-re végződő igető esetén}

GrammarItemIndex = 658954

\begin{desc}
Ha az igető d-re vagy t-te végződik, a könnyebb kiejtés miatt egy -e
kötőhangot használunk E/2, E/3 és T/2 személyben.

Pl.:
ich bitte
du bittest
er bittet
sie bittet
es bittet
wir bitten
ihr bittet
sie bitten
Sie bitten

Ilyen igék:
* bitten - kér
* finden - talál
* reden - beszél
* antworten - válaszol
* baden - fürdik
* binden - köt
* arbeiten - dolgozik
\end{desc}

%Maklári: 10. oldal
\begin{exmp}
1 | du bittest | te kérsz
2 | er bittet | ö (f) kér
3 | ihr bittet | ti kértek
4 | du findest | te találsz
5 | sie findet | ö (n) talál
6 | ihr findet | ti találtok
7 | du redest | te beszélsz
8 | es redet | ez beszél
9 | ihr redet | ti beszéltek
10 | du antwortest | te felelsz
11 | er antwortet | ő (f) felel
12 | ihr antwortet | ti feleltek
13 | du badest | te fürdessz
14 | sie badet | ő (n) fürdik
15 | ihr badet | ti fürdötök
16 | du bindest | te kötsz
17 | es bindet | ez köt
18 | ihr bindet | ti köttök
19 | du arbeitest | te dolgozol
20 | er arbeitet | ő (f) dolgozik
21 | ihr arbeitet | ti dolgoztok
\end{exmp}

\title{Az ige ragozása jelen időben/Tőhangváltós, azaz erős igék/Umlautos igék}

GrammarItemIndex = 546385

\begin{desc}
A tőhangváltós igék megváltoztatják a szótövüket E/2 és E/3 személyben. Két típusuk van, az egyik az umlautos igék típusa.

ich fahre 
du fährst 
er/sie/es fährt 
wir fahren 
ihr fahrt 
sie/Sie fahren 

Példa umlautos igékre:
* fallen
* schlafen
* fangen
* laufen
* tragen
* graben
* waschen
* backen
\end{desc}

\begin{exmp}
\end{exmp}

\title{Az ige ragozása jelen időben/Tőhangváltós, azaz erős igék/Brechungos igék}

GrammarItemIndex = 834185

\begin{desc}
A tőhangváltós igék megváltoztatják a szótövüket E/2 és E/3 személyben. Két típusuk van, az egyik az brechungos igék típusa.

e --> i; e --> ie

Pl.:
ich       | gebe  | sehe 
du        | gibst | siehst
er/sie/es | gibt  | sieht 
wir       | geben | sehen 
ihr       | gebt  | seht 
sie/Sie   | geben | sehen 

Példák brechungos igékre:
* helfen (i)
* lesen (ie)
* sprechen (i)
* treffen (i)
* brechen (i)
* sterben (i)
* vergessen (i)
\end{desc}

\begin{exmp}
\end{exmp}

\title{Az ige ragozása jelen időben/Tőhangváltós, azaz erős igék/Az essen és nehmen Brechungos igék ragozása}

GrammarItemIndex = 645385

\begin{desc}
essen - eszik
nehmen - vesz

ich         | nehme  | esse 
du          | nimmst | isst 
er, sie, es | nimmt  | isst 
wir         | nehmen | essen 
ihr         | nehmt  | esst 
sie, Sie    | nehmen | essen 
\end{desc}

%Maklári: 15. oldal
\begin{exmp}
1 | Er isst Suppe. | Sevest eszik (f).
2 | Du isst Apfel. | Almát eszel.
3 | Wir nehmen Käse. | Veszünk sajtot.
4 | Ihr nehmt Mittagessen. | Vesztek Ebédet.
5 | Ich esse Abendessen. | Eszek vacsorát.
6 | Peter nimmt Brot. | Péter vesz kenyeret.
7 | Ági isst Torte. | Ági tortát eszik.
8 | Wir essen Schinken. | Sonkát eszünk.
9 | Ich nehme Gemüse. | Veszek zöldséget.
10 | Joe isst Zwiebel. | Joe hagymát eszik.
11 | Ihr esst Nudeln. | Tésztát esztek. (Nudeln)
12 | Vater isst Strudel. | Apa rétest eszik. (Strudel)
13 | Klaus nimmt ein Haus. | Klaus vesz egy házat.
14 | Du nimmst Brot. | Veszel kenyeret.
15 | Mutter nimmt Fleisch. | Anya húst vesz.
16 | Ihr esst zu Hause. | Otthon esztek.
17 | Er nimmt Brötchen. | Zsömlét vesz (f).
18 | Sie essen hier. | Ők itt esznek.
19 | Sie nehmen Orange. | Ők narancsot vesznek.
20 | Du nimmst ein Haus. | Egy házat veszel.
\end{exmp}

\title{Az ige ragozása jelen időben/Rendhagyó ragozású igék/A sein}

GrammarItemIndex = 543266

\begin{desc}
ich bin 
du bist 
er/sie/es ist 
wir sind 
ihr seid 
sie/Sie sind 
\end{desc}

\begin{exmp}
\end{exmp}

\title{Az ige ragozása jelen időben/Rendhagyó ragozású igék/A haben}

GrammarItemIndex = 42343

\begin{desc}
A haben igével birtoklást fejezünk ki, és amit birtoklunk, azt tárgyesetbe (Akkusativ) tesszük.

Ragozása:
ich       | habe
du        | hast
er/sie/es | hat
wir       | haben
ihr       | habt
sie/Sie   | haben
\end{desc}

\begin{exmp}
1 | Mein Vater hat ein Haus. | Apámnak van egy háza.
2 | Du hast Zeit. | Van időd.
3 | Sie haben einen Garten. | Van egy kertjük.
4 | Mutter hat viel Blumen. | Anyának sok virága van.
5 | Er hat ein Buch. | Neki (f) van egy könyve.
6 | Ich habe eine Freundin. | Nekem van egy barátnőm.
7 | Ihr habt Zeit immer. | Mindig van időtök.
8 | Józsi hat einen Kuli. | Józsinak van egy tolla.
9 | Hans und Gertrud haben einen Garten. | Hansnak és Gertrudnak van egy kertje.
\end{exmp}

\title{Az ige ragozása jelen időben/Rendhagyó ragozású igék/A werden}

GrammarItemIndex = 543264

\begin{desc}
A werden ige azt fejezi ki, hogy valaki vagy valami válik, lesz
valamivé.
Pl.: Ich werde Millionär.

werden ragozása:
ich werde - leszek 
du wirst - leszel
er/sie/es wird - lesz
wir werden - leszünk
ihr werdet - lesztek
sie/Sie werden - lesznek
\end{desc}

%Maklári: 22.oldal
\begin{exmp}
1 | Ich werde alt. | Öreg leszek.
2 | Józsi wird Lehrer. | Tanár lesz Józsi.
3 | Wir werden hungrig. | Éhesek leszünk.
4 | Anna wird müde. | Anna fáradt lesz.
5 | Ihr werdet frisch. | Frissek lesztek.
6 | Wirst du Lehrer? | Tanár leszel?
7 | Sie werden durstig. | Szomjasak lesznek.
8 | Sie wird traurig. | Szomorú lesz (n).
9 | Ich werde böse. | Mérges leszek.
10 | Die Lehrerin wird gesund. | A tanárnő egészséges lesz.
11 | Pista wird Ingenieur. | Pista mérnök lesz. (r Ingenieur)
12 | Wirst du Arzt? | Orvos leszel?
13 | Bözsi wird Hausfrau. | Bözsi házlasszony lesz.
14 | Wir werden reich. | Gazdagok leszünk.
15 | Ich werde stark und klug. | Erős és okos leszek.
16 | Die Kinder werden hungrig. | Éhesek lesznek a gyerekek.
\end{exmp}

\title{Az ige ragozása jelen időben/Rendhagyó ragozású igék/A wissen}

GrammarItemIndex = 745285

\begin{desc}
wissen = tud, ismer valamit

ich       | weiß
du        | weißt
er/sie/es | weiß
wir       | wissen
ihr       | wisst
sie/Sie   | wissen

Pl: Tante Ancika weiß alles. - Ancika néni mindent tud.
\end{desc}

%Maklári: 23. oldal
\begin{exmp}
1 | Ich weiß nichts. | Semmit se tudok.
2 | Weißt du den Weg? | Tudod az utat?
3 | Józsi weiß etwas. | Józsi tud valamit.
4 | Wisst ihr den Grund? | Tudjátok az okot?
5 | Weißt du seinen Name? | Tudod a nevét (f)?
6 | Wissen Sie den Name des Hotels? | Tudja (Ön) a hotel nevét?
7 | Was weißt du? | Mit tudsz?
8 | Wir wissen das. | Azt tudjuk.
9 | Wisst ihr die Adresse? | Tudjátok a címet?
10 | Ági weiß die Lösung. | Ági tudja a megoldást.
11 | Ich weiß das Datum nicht. | Nem tudom a dátumot. (s Datum)
12 | Ihr wisst immer alles. | Ti mindig mindent tudtok.
13 | Weiß sie die Telefonnumber? | Tudja (n) a telefonszámot?
14 | Ich weiß nichts. | Semmit sem tudok.
15 | Woher wisst ihr es? | Honnan tudjátok ezt?
16 | Mein Freund weiß viel. | A barátom sokat tud.
\end{exmp}

\title{Az ige ragozása jelen időben/Rendhagyó ragozású igék/A tun}

GrammarItemIndex = 54327

\begin{desc}
tun A, für + A = tesz, csinál vmit (vkiért/vmiért)

Pl.: Meine Oma tut alles für uns. - A nagyim mindent megtesz értünk.

a Tun ige ragozása:
ich tue
du tust
er/sie/es tut
wir tun
ihr tut
sie/Sie tun
\end{desc}

%Maklári: 24. oldal
\begin{exmp}
1 | Er tut nichts. | Semmit se tesz (f).
2 | Tust du das? | Megteszed azt?
3 | Sie tut ihre Tochter hierher. | A lányát ide teszi (n). (hierher)
4 | Ihr tut Salz in die Suppe. | Sót tesztek a levesbe.
5 | Was tust du? | Mit teszel?
6 | Er tut alles für uns. | Mindent megtesz (f) értünk.
7 | Ich tue das Radio auf den Tisch. | A rádiót az asztalra teszem.
8 | Was tust du hier? | Mit csinálsz itt?
9 | Sie tut immer etwas Gutes. | Mindig valami jót tesz (n). (tun Gut)
10 | Warum tut ihr es? | Miért teszitek ezt?
11 | Béla tut immer Wunder. | Béla mindig csodát tesz.
12 | Ida tut viel für ihn. | Ida sokat tesz érte (f). (für)
13 | Du tust das Buch auf den Tisch. | A könyvet az asztalra teszed.
14 | Ich tue für sie es. | Érte (n) teszem ezt.
15 | Er tut das recht. | Azt helyesen teszi (f).
16 | Was tust du damit? | Mit csinálsz ezzel? (damit)
\end{exmp}

\title{Szórend kijelentő mondatban/Szórend kijelentő mondatban alap esetben}

GrammarItemIndex = 6

\begin{desc}
Er wohnt hier. - Ő itt lakik.
\end{desc}

\begin{exmp}
1 | Ich trinke langsam. | Lassan iszok.
2 | Ihr kommt schnell. | Gyorsan jöttök.
3 | Wir trinken Wein. | Bort iszunk.
4 | Sie kommt hier. | Itt jön (n).
5 | Der Bus kommt dort. | Ott jön a busz.
6 | Tina arbeitet langsam. | Tina lassan dolgozik.
7 | Sie trinkt Wasser. | Vizet iszik (n).
8 | Ihr lernt gut. | Jól tanultok.
9 | Wir sprechen immer. | Mindig beszélünk.
10 | Er geht nach Hause. | Hazamegy (f).
11 | Sie singen schön. | Szépen énekel. (Ön)
12 | Wir arbeiten hier. | Itt dolgozunk.
13 | Wir hören Musik. | Zenét hallgatunk.
14 | Sie sprechen dort. | Ott beszélnek.
15 | Sie geht langsam. | Lassan megy. (n)
16 | Er kommt schnell. | Gyorsan jön. (f)
17 | Der Zug haltet. | A vonat áll. (halten)
18 | Das Mädchen wohnt hier. | A lány itt lakik.
19 | Jörg arbeitet. | Jörg dolgozik.
20 | Sie wohnen dort. | Ott laknak.
\end{exmp}

\title{Szórend kijelentő mondatban/Szórend kijelentő mondatban kiemeléssel}

GrammarItemIndex = 9

\begin{desc}
Ha valamit ki akarunk emelni a mondatból, akkor azt az első helyre tesszük, az alany pedig az ige után a 3. helyre kerül.

Pl.: Hier wohnt er. - Itt lakik.
\end{desc}

\begin{exmp}
1 | Limonade trinkt er. | Limonádét iszik (f).
2 | Zu Hause singt sie. | Otthon énekel (n).
3 | Falsch Antwort gibt sie. | Helytelen választ ad (n).
4 | Dort wohnt ihr. | Ott laktok.
5 | Die Hausaufgabe mache ich. | Megcsinálom a házifeladatot.
6 | Wasser trinkt Judit. | Vizet iszik Judit.
7 | In der Nähe wohnen wir. | A közelben lakunk.
8 | Bier holt er. | Hoz sört (f). (holen)
9 | Dort arbeite ich. | Ott dolgozom.
10 | Tee trinken Sie. | Teát iszik (Ön). (r Tee)
11 | Kakao trinken wir. | Kakaót iszunk.
12 | Bier bringt ihr. | Sört hoztok. (bringen)
13 | Musik hört er. | Zenét hallgat (f).
14 | Alkohol trinken wir. | Alkoholt iszunk.
15 | Zu Hause arbeitet er. | Otthon dolgozik (f).
16 | Sofort kommt sie. | Azonnal jön (n).
17 | Radio hört er. | Rádiót hallgat (f).
18 | Ildikó heißt sie. | Ildikónak hívják.
19 | Emil heißt er. | Emilnek hívják.
\end{exmp}

\title{A névmás és főnév sorrendje}

GrammarItemIndex = 543348

\begin{desc}
* Ha két főnév áll a mondatban, akkor a dativos áll az első, az akkusativos a második helyen.
* Ha két névmás áll a mondatban, akkor a akkusativos áll az első, az dativos a második helyen.
* Ha egy névmás és egy főnév áll a mondatban, akkor a névmás mindig előbb szerepel.

Pl.: * Der Lehrer erklärt dem Schüler den Satz. - A tanár elmagyarázza a diáknak a mondatot.
* Der Lehrer erklärt ihn ihm. - A tanár elmagyarázza azt neki.
* Er erklärt ihm den Satz. - Elmagyarázza neki a mondatot.
\end{desc}

%Maklári: 49.oldal
\begin{exmp}
1 | Ich gebe dem Gast meinen Mantel. | A vendégnek adom a kabátomat.
2 | Die Verkäuferin schenkt ihrem Sohn eine Tasche. | Az eladónő ajándékoz a fiának egy táskát.
3 | Der Kellner öffnet dem Gast die Tür. | A pincér kinyitja a vendégnek az ajtót. (r Kellner)
4 | Wir kaufen unseren Eltern eine Vase. | Vásárolunk a szüleinknek egy vázát.
5 | Die Direktorin zeigt den Gästen den Weg. | Az igazgatónő megmutatja a vendégeknek az utat.
6 | Die Lehrerin erzählt ihm die Geschichte. | A tanárnő elmagyarázza neki (f) a történelmet.
7 | Der Lehrer diktiert es den Kindern. | A tanár diktálja ezt a gyerekeknek. (diktieren)
8 | Die Eltern geben ihr eine Blume. | A szülők adnak egy virágot neki (n).
9 | Er schreibt seiner Freundin ein Brief. | Egy levelet ír (f) barátnőjének.
10 | Der Polizist erklärt ihm den Weg. | A rendőr elmagyarázza neki (f) az utat. (erklären)
11 | Die Familie kauft ihnen es. | A család ezt vásárolja nekik.
12 | Wir geben ihr es. | Ezt adjuk neki (n).
13 | Sie erklärt ihm den Satz. | Elmagyarázza (n) neki (f) a mondatot.
14 | Der Kellner empfiehlt sie den Gästen. | A pincér őt (n) ajánlja a vendégeknek. (empfehlen)
\end{exmp}

\title{Igekötős igék/Elváló igekötős igék}

GrammarItemIndex = 7

\begin{desc}
Képzése: alany + ige ragozva + többi mundatrész + igekötő

Pl.: Erika steht jeden Tag um 5 Uhr auf. - Erika minden nap 5-kor kel fel.

Pédák elváló igekötős igékre:
* anrufen - felhív
* einschlafen - elalszik
* weggehen - elmegy
* einkaufen - bevásárol
* einsteigen - beszáll vhová
* aussteigen - kiszáll vhonnan
* umsteigen - átszáll
* ankommen - megérkezik
* abfahren - elindul
* zurückgehen - visszamegy
\end{desc}

%Maklári: 101. oldal
\begin{exmp}
1 | Wir gehen um 5 Uhr weg. | 5-kor megyünk el.
2 | Wir rufen morgen Kati an. | Felhívjuk holnap Katit.
3 | Er kauft alles in dem Geschäft ein. | Mindent bevásárol (f) az üzletben.
4 | Der Zug fährt nicht nach Vien ab. | A vonat nem indul Bécsbe.
5 | Meine Freundin schläft schnell ein. | A barátnőm gyorsan elalszik.
6 | Kommen die Gäste auf den Bahnhof an? | A vendégek megérkeznek a vasútállomásra?
7 | Der Chef kommt auf den Flughafen an. | A főnök megérkezik a repülőtérre.
8 | Steigt ihr auf die Strassenbahn um? | Átszálltok a villamosra?
9 | Wir steigen in den Bus ein. | Beszállunk a buszba.
10 | Ich steige in den Bus nicht ein. | Nem szállok be a buszba.
11 | Die U-Bahn kommt jetzt an. | Most érkezik a metró.
12 | Gehst du so früh weg? | Ilyen hamar elmész? (so früh)
13 | Der Passagier steigt aus der Strassenbahn aus. | Az utas kiszáll a villamosból. (r Passagier)
14 | Geht ihr in die Schule zurück? | Visszamentek az iskolába?
15 | Dein Bus fährt bald ab. | Hamarosan elindul a buszod. (bald)
16 | Der Minister kommt jetzt in das Parlament an. | A miniszter most érkezik a parlamentbe. (s Parlament)
17 | Wir kaufen alles ein. | Bevásárolunk mindent.
18 | Rufst du ihn an? | Felhívod őt (f)?
19 | Ich rufe dich an. | Felhívlak.
\end{exmp}

\title{Igekötős igék/Nem elváló igekötős igék}

GrammarItemIndex = 8

\begin{desc}
Az alábbi igekötők nem válnak el az igétől:
ge-, be-, er-, re-, ent-, emp-, ver-, zer-, miss-

Néhány ilyen ige:
* gefallen - tetszik
* vbezahlen - kifizet
* erzählen - elmesél
* entdecken - felfedez
* empfehlen - ajánl
* verstehen - megért
* zerstören - lerombol
* missverstehen - félreért
\end{desc}

\begin{exmp}
1 | Gefällt das Mädchen dir? | Tetszik neked a lány?
2 | Ich bezahle die Rechnung. | Kifizetem a számlát. (e Rechnung)
3 | Ich erzähle dir eine Geschichte. | Elmesélek neked egy történetet.
4 | Warum kannst du nichts entdecken? | Miért nem tudsz semmit sem felfedezni?
5 | Ich empfehle dir diese Lösung. | Ezt a megoldást ajánlom neked.
6 | Ich empfehle mich. | Ajánlom magamat.
7 | Ich empfehle dir dieses Buch. | Ajánlom neked ezt a könyvet.
8 | Verstehst du das? | Érted azt?
9 | Er zerstört alles. | Mindent lerombol (f).
10 | Sie missversteht mich immer. | Mindig félreért (n) engem.
11 | Kannst du den Satz nicht verstehen? | Nem tudod megérteni a mondatot?
12 | Ich empfehle dir eine Restaurant. | Ajánlok neked egy éttermet.
13 | Der Kind zerstört seine Spielzeuge. | A gyerek összetöri a játékait.
14 | Ich bezahle die Suppe und du bezahlst die Süßigkeit. | Én fizetem a levest és te fizeted az édességet.
15 | Gefällt die Wohnung für die Eltern? | Tetszik a szülőknek a lakás?
16 | Erzählst du mir die Geschichte deiner Ehre? | Elemséled neken a házasságod történetét?
17 | Wann erzählst du die Geschichte? | Mikor meséled el a történetet?
\end{exmp}

\title{Igekötős igék/Hol elváló, hol nem elváló igekötős igék}

GrammarItemIndex = 523278

\begin{desc}
\end{desc}

\begin{exmp}
\end{exmp}

\title{Igekötős igék/Az um igekötő}

GrammarItemIndex = 534278

\begin{desc}
\end{desc}

\begin{exmp}
\end{exmp}

\title{Kérdő mondatok/Kérdőszó nélküli mondat}

GrammarItemIndex = 6432

\begin{desc}
Képzése: állítmány + alany + többi mondatrész

Pl: Kommst du morgen ins Kino?
\end{desc}

\begin{exmp}
1 | Kommst du morgen ins Kino? | Jössz holnap moziba?
2 | Gehen wir nach Hause? | Haza megyünk?
3 | Lesen Sie Zeitung? | Újságot olvas?
4 | Essen ihr Suppe? | Esztek levest?
\end{exmp}

\title{Kérdő mondatok/Kérdőmondat kérdőszóval}

GrammarItemIndex = 42357

\begin{desc}
Képzése: kérdőszó + állítmány + alany + többi mondatrész

Pl: Wann kommst du nach Hause?
\end{desc}

\begin{exmp}
1 | Wann kommst du nach Hause? | Mikor jössz haza?
2 | Wer steht hier? | Ki áll itt?
3 | Wer sind deine Eltern? | Kik a szüleid?
4 | Wem gibst du die Blume? | Kinek adod a virágot?
5 | Was siehst du dort? | Kit látsz ott?
\end{exmp}

\title{Kérdő mondatok/Prepozíciós kérdőszavak/Személyekre kérdezve}

GrammarItemIndex = 335433

\begin{desc}
A prepozíció esetébe kell tenni a kérdőnévmást.

tárgyeset -> wen
részes eset -> wem
\end{desc}

%Maklári: 112
\begin{exmp}
1 | von wem | kitől
2 | bei wem | kinél
3 | auf wen | kire
4 | zu wem | kihez
5 | für wen | kiért
6 | vor wen | ki elé
7 | vor wem | ki előtt
8 | in wem | kiben
9 | um wen | ki körül
10 | unter wem | ki alatt
11 | über wem | ki felett
12 | hinter wen | ki mögé
13 | aus wem | kiből
14 | über wen | kiből (über)
15 | ohne wen | ki nélül
16 | durch wen | ki által
17 | gegen wen | ki ellen
18 | in wen | kibe
19 | neben wem | ki mellett
20 | hinter wem | ki mögött
21 | über wen | ki fölé
22 | unter wen | ki alá
23 | neben wen | ki mellé
24 | mit wem | kivel
\end{exmp}

\title{Kérdő mondatok/Prepozíciós kérdőszavak/Éllettelenre kérdezve}

GrammarItemIndex = 765433

\begin{desc}
wo(r) + prepozíció
\end{desc}

%Maklári: 114
\begin{exmp}
1 | wovon | mitől
2 | worunter | mi alá
3 | worüber | mi fölé
4 | wovor | mi elé
5 | woraus | miből
6 | wohinter | mi után
7 | wozu | mihez
8 | worunter | mialatt
9 | wofür | miért (für)
10 | worüber | miről (über)
11 | worin | miben
12 | womit | mivel
13 | woran | min (függőleges)
14 | worauf | min (vízszintes)
15 | wobei | minél
16 | woran | mire (függőleges)
17 | wodurch | miáltal
18 | worin | miben
19 | worüber | mi fölött
20 | wovor | mi előtt
21 | worin | mibe
22 | wohinter | mi mögött
23 | wogegen | mi ellen
24 | worauf | mire (vízszintes)
\end{exmp}

\title{Kérdő mondatok/Prepozíciós kérdőszavak/Éllettelenre vonatkozó kérdésre való válaszok}

GrammarItemIndex = 423425

\begin{desc}
da(r) + prepozíció.
\end{desc}

%Maklári: 114
\begin{exmp}
1 | davon | attól
2 | darunter | azalá
3 | darüber | a fölé
4 | davor | az elé
5 | daraus | abból
6 | danach | az után
7 | dazu | ahhoz
8 | darunter | az alatt
9 | dafür | azért
10 | davon | arról
11 | darin | abban
12 | damit | azzal
13 | daran | azon (függőleges)
14 | darauf | azon (vízszintes)
15 | dabei | annál
16 | daran | arra (függőleges)
17 | dadurch | azáltal
18 | darin | abban
19 | darüber | afölött
20 | davor | azelőtt
21 | darin | abba
22 | dahinter | amögött
23 | dagegen | az ellen
24 | darauf | arra (vízszintes)
\end{exmp}

\title{Tagadás/Nein}

GrammarItemIndex = 5253

\begin{desc}
A nein szóval egész mondatot tagadunk. A nein után mindig pont, vagy vessző áll.

Pl.: Nein, ich liebe dich nicht. - Nem, én nem szeretlek téged.
\end{desc}

\begin{exmp}
\end{exmp}

\title{Tagadás/A nicht}

GrammarItemIndex = 533598

\begin{desc}
A nicht szóval mondatrészt tagadunk.

* Ha az állítmány egy tagú, akkor a mondat végén áll a nicht.
  Pl.: Mein Vater kommt heute nicht. - Apám ma nem jön.
* Ha az állítmány több tagú, akkor a nicht a mondat végére, az állítmány ragozatlan része elé kerül.
  Pl.: - Wir stehen heute nicht auf. - Mi nem kelünk fel ma.
       - Ich kann morgen nicht kommen. - Holnap nem tudok jönni.
       - Wir haben den Brief nicht gelesen. - Nem olvastuk a levelet.
* Ha egy bizonyos mondatrész tagadunk, akkor az elé tesszük.
  Pl.: Ich suche nich dich, sondern ihn. - Nem téged kereslek, hanem őt.
* Ha prepozíciós alak van a mondatban, akkor az elé tesszük.
  Pl.: Wir fahren nicht nach Deutschland. - Nem utazunk Németországba.
\end{desc}

%Maklári: 130.oldal
\begin{exmp}
1 | Ich gehe nicht. | Nem megyek.
2 | Er kommt nicht. | Nem jön (f).
3 | Sie schläft jetzt nicht. | Most nem alszik (n).
4 | Wir lesen nicht. | Nem olvasunk.
5 | Sie schreiben ihm nicht. | Nem írnak neki (f).
6 | Ich trinke jetzt nicht. | Most nem iszok.
7 | Vater geht jetzt nicht weg. | Apa most nem megy el.
8 | Der Zug fährt nicht an. | A vonat nem indul el. (anfahren - elindul)
9 | Sie schläft schnell nicht ein. | Nem alszik (n) el gyorsan.
10 | Oma kann nicht gehen. | Nagyi nem tud menni.
11 | Ich kann jetzt nicht laufen. | Nem tudok most futni.
12 | Ich habe sie nicht gesehen. | Nem láttam őt (n). (sehen (Perfekt))
13 | Wir haben nicht getrunken. | Nem ittunk. (trinken (Perfekt))
14 | Ich habe gestern nicht geschlafen. | Tegnap nem aludtam.
15 | Ich suche nicht sie. | Nem őt (n) keresem.
16 | Ich sehe nicht dich. | Nem téged nézlek.
17 | Wir suchen nicht den Schlüssel. | Nem a kulcsot keressük.
18 | Ich esse nicht deine Suppe. | Nem a te levesedet eszem.
19 | Nicht seine Freundin steht dort. | Nem az ő (f) barátnője áll ott.
20 | Nicht unser Lehrer steht hier. | Nem a mi tanárunk áll itt.
21 | Unsere Freunde sind nicht bei euch. | A barátaink nincsenek nálatok.
22 | Sie kommt nich bei uns. | Nem jön (n) velünk.
23 | Die Gäste gehen morgen nicht nach Hause. | A vendégek holnap nem mennek haza.
24 | Wir schlafen jetzt nicht zu Hause. | Most nem alszunk otthon.
25 | Wir essen heute nicht bei euch. | Ma nem eszünk nálatok.
26 | Er ist nicht in seinem Zimmer. | Nincs a szobájában (f).
\end{exmp}

\title{Tagadás/A kein}

GrammarItemIndex = 423837

\begin{desc}
A kein szóval főnevet tagadunk. Úgy ragozzuk, mint az ein határozatlan névelőt és a tagadott főnév elé írjuk.

Pl.: * Er bekommt keinen Tisch. - Nem kap asztalt.
* Wir haben keine Zeit - Nincs időnk.
\end{desc}

%Maklári: 131. oldal
\begin{exmp}
1 | Ich habe kein Geld. | Nincs pénzem.
2 | Er hat keinen Vater. | Nincs apja (f).
3 | Wir haben kein Brot. | Nincs kenyerünk.
4 | Habt ihr kein Auto? | Nincs autótok?
5 | Wir haben keine Bücher. | Nincsenek könyveink.
6 | Hast du keinen Zucker? | Nincs cukrod? (r Zucker)
7 | Wir haben keine Kinder. | Nincsenek gyerekeink.
8 | Kaufst du kein Brot? | Nem vettél kenyeret?
9 | Ich habe keine Idee. | Nincs ötletem.
10 | Er ist kein Mörder. | Ő (f) nem gyiklos.
11 | Ich habe keine Lust. | Nincs kedvem.
12 | Hast du keine Hunde? | Nincsenek kutyáid?
13 | Hast du keinen Garage? | Nincsen garázsod?
14 | Er bekommt keinen Tisch. | Nem kap (f) asztalt. (bekommen)
15 | Sie hat keine Zeit. | Nincs ideje (n).
16 | Haben sie kein Fieber? | Nincs lázuk? (s Fieber)
17 | Ich habe keinen Bruder. | Nincs fivérem.
18 | Habt ihr keine Idee? | Nincs ötletetek?
19 | Wir haben keine Idee. | Nincs ötletünk.
20 | Hattet ihr keinen Apfel? | nincs almátok?
21 | Wir haben keinen Apfel. | Nincs almánk.
\end{exmp}

\title{Tagadás/Nichts, niemand, nirgendwo, nirgendwohin, nie}

GrammarItemIndex = 423685

\begin{desc}
A magyarral ellentétben a németben nincs kettős tagadás.

* nichts = semmi, semmit
* niemand = senki
* nirgendwo = sehol
* nirgendwohin = sehová
* nie = sohe
\end{desc}

%Maklári: 132. oldal
\begin{exmp}
1 | Ich sehe nichts. | Semmit se látok.
2 | Er hört nichts. | Semmit se hall (f).
3 | Sie trinkt nichts. | Semmit sem iszik (n).
4 | Siehst du nichts? | Semmit se látsz?
5 | Ich habe nichts. | Semmim sincs.
6 | Es ist nichts hier. | Itt semmi nincs.
7 | Seht ihr nichts? | Semmit se láttok?
8 | Er sagt nichts. | Semmit sem mond (f).
9 | Du bekommst nichts. | Semmit sem kapsz.
10 | Es ist nichts in der Tasche. | Semmi sincs a táskában.
11 | Wir kaufen nichts. | Semmit se vásárolunk.
12 | Ich möchte nichts. | Nem szeretnék semmit.
13 | Niemand kommt. | Senki se jön.
14 | Ich sehe niemand. | Senkit sem látok.
15 | Ich kaufe niemand Geschenk. | Senkinek sem veszek ajándékot.
16 | Niemand liebt mein Hund. | Senki sem szereti a kutyámat.
17 | Niemand isst Eis. | Senki sem eszik fagyit.
18 | Sie hilfen niemand. | Senkinek sem segítenek.
19 | Sie lädt niemand ein. | Senkit sem hív (n) meg.
20 | Er liebt niemand. | Senkit sem szeret (f).
21 | Er gibt niemand seinen Wagen. | Senkinek sem adja (f) a kocsiját.
22 | Niemand steht hier. | Senki sem áll itt.
23 | Niemand liebt mich. | Senki sem szeret engem.
24 | Ich erzähle es niemand. | Senkinek se mesélem el.
25 | Ich sehe sie nirgendwo. | Sehol sem látom őt (n).
26 | Wir gehen nirgendwohin. | Sehová nem megyünk.
27 | Du bekommst Schuh nirgendwo. | Sehol sem kapsz cipőt.
28 | Sie findet ihre Tasche nirgendwo. | Sehol sem találja (n) a táskáját.
29 | Ich spaziere nirgendwohin. | Sehov nem sétálok.
30 | Ich finde nirgendwo das Brot. | Sehol sem találom a kenyeret.
31 | Sie ist froh nirgendwo. | Sehol sem boldog (n). (froh)
32 | Du gehst nirgendwo. | Sehová sem mész.
33 | Du bist nie zu Hause. | Sohasem vagy otthon.
34 | Ihr spracht nie miteinander. | Sohasem beszéltek egymással.
35 | Ich habe sie noch nie gesehen. | Még sohasem láttam őt (n).
36 | Er singt nie. | Sohase énekel.
37 | Jetzt oder nie! | Most vagy soha!
38 | Ich bin noch nie geschwommen. | Még sohasem úsztam.
39 | Er arbeitet nie. | Sohase dolgozik (f).
40 | Sie ist nie müde. | Sohase fáradt (n).
41 | Du kommst nie zu uns. | Sohase jössz hozzánk.
42 | Er geht nie in die Schule. | Sohase megy (f) iskolába.
\end{exmp}

\title{Tagadás/Kéttagú tagadószavak}

GrammarItemIndex = 5344

\begin{desc}
* noch nicht = még nem
* nicht mehr = már nem
* kein ... mehr = már többet nem
* nichts mehr = már semmit se

Pl: * Ich gehe noch nicht in die Schule. - Még nem megyek iskolába.
* Er wohnt nicht mehr hier. - Már nem lakik itt.
* Wir essen keine Schokolade mehr. - Már nem eszünk több csokit.
* Ich möchte nichts mehr kaufen. - Már semmit sem szeretnék vásárolni.
\end{desc}

%Maklári: 134
\begin{exmp}
1 | Mein Freund kommt noch nicht. | Még nem jön a barátom.
2 | Ich bin schläfrig noch nicht. | Még nem vagyok álmos. (schläfrig)
3 | Wir sind durstig noch nicht. | Még nem vagyunk szomjasak.
4 | Sie ist noch nicht zu Hause. | Még nincs (n) otthon.
5 | Wir sehen noch nicht fern. | Még nem nézünk TV-t.
6 | Die Gäste sind müde noch nicht. | A vendégek még nem fáradtak.
7 | Ich stehe noch nicht auf. | Még nem kelek fel.
8 | Wir sind fertig noch nicht. | Még nem vagyunk kész.
9 | Sie schlafen nicht mehr. | Már nem alszanak.
10 | Ich bin krank nicht mehr. | Már nem vagyok beteg.
11 | Er steht hier nicht mehr. | Már nem áll (f) itt.
12 | Sie kommt nicht mehr. | Már nem jön (n).
13 | Die Sonne scheint nicht mehr. | A Nap már nem süt. (die Sonne scheinen)
14 | Der Gast schläft nicht mehr. | A vendég már nem alszik.
15 | Wir sind müde nicht mehr. | Nem vagyunk már fáradtak.
16 | Sie liest so viel nicht mehr. | Már nem olvas (n) olyan sokat. (so)
17 | Ich schlafe nicht mehr. | Már nem alszok.
18 | Mein Großvater liebt nicht mehr. | A nagyapám már nem él.
19 | Sie isst kein Brot mehr. | Már nem eszik (n) kenyeret.
20 | Ich nehme keinen Apfel mehr. | Már nem veszek almát.
21 | Wir haben kein Zeit mehr. | Már nincs időnk.
22 | Ich habe keine Angst mehr. | Már nem félek.
23 | Sie liest keine Bücher mehr. | Már nem olvas (n) könyveket.
24 | Er sieht keine Filme mehr. | Már nem néz (f) filmeket.
25 | Ich habe keine Lust mehr. | Már nincs kedvem.
26 | Sie lesen Keine Zeitung mehr. | Már nem olvasnak újságot.
27 | Trinkt ihr keinen Wein mehr? | Nem isztok már bort?
28 | Ich esse keinen Kuchen mehr. | Nem eszek már több sütit.
29 | Er hört nichts mehr. | Már semmit sem hall (f).
30 | Wir kaufen nichts mehr. | Már semmit sem vásárolunk.
31 | Sie trinkt nichts mehr. | Már semmit sem iszik (n).
32 | Er wollt nichts mehr. | Már semmit sem akar (f).
33 | Wir lesen nichts mehr. | Már semmit sem olvasunk.
34 | Die Gäste essen nicht mehr. | A vendégek már nem esznek.
\end{exmp}

\title{Visszaható igék/Visszaható névmás tárgyesete}

GrammarItemIndex = 5453

\begin{desc}
mich | magam 
dich | magad 
sich | magát 
uns  | magunkat 
euch | magatokat 
sich | magukat 

sich waschen - mosakodik
sich kämmen - fésülködik
sich rasieren - borotválkozik
sich anziehen - felöltözik
sich ausziehen - levetkőzik
sich fühlen - érzi magát valahogy
sich freuen - örül
\end{desc}

\begin{exmp}
1 | ich wasche mich | mosakodom
2 | sie kämmt sich | fésülködik (n)
3 | wir kämmen uns | fésülködünk
4 | er rasiert sich | borotválkozik
5 | sie säscht sich | mosakszik (n)
6 | ich freue mich | örülök
\end{exmp}

\title{Visszaható igék/Visszaható igék tárgyesettel}

GrammarItemIndex = 543556

\begin{desc}
Az alábbi igékhez kapcsolódó visszaható névmás mindig tárgyesetben van:
* sich interessieren für + A -> érdeklődik vmi iránt
* sich beschäftigen mit + D --> foglalkozik vmivel
* sich freuen auf + A --------> örül előre vminek
* sich freuen über + A -------> örül vminek (ami már megvan)
* sich treffen mit + D -------> találkozik vkivel
* sich erinnern an + A --------> emlékezik vmire
* sich irren in + D ----------> téved vmiben

A visszaható névmás ragozása tárgyesetben:
mich | magam
dich | magad
sich | magát
uns  | magunkat
euch | magatokat
sich | magukat
\end{desc}

%Maklári: 157.oldal
\begin{exmp}
1 | Ich interessiere mich für die Sport. | Érdeklődöm a sport iránt.
2 | Interessierst du dich für die Mathematik? | Érdeklődsz a matematika iránt?
3 | Er beschäftigt sich mit der deutschen Sprache. | Ő (f) foglalkozik a német nyelvvel.
4 | Die Kinder freuen sich schon in November auf das Weihnachten. | A gyerekek már novemberben örülnek a karácsonynak. (s Weihnachten)
5 | Freust du dich über das Geschenk? | Örülsz az ajándéknak? (s Geschenk)
6 | Wir treffen uns bei mir mit deinen Freunden. | Nálam találkozunk a barátaiddal.
7 | Trefft ihr euch oft mit ihnen? | Gyakran találkoztok velük? 
8 | Ich erinnere mich noch an deine Eltern. | Emlékszem még a szüleidre.
9 | Erinnert ihr euch noch an jenen Film? | Emlékeztek még arra a filmre? (jener) 
10 | Er irrt sich darin. | Tévedett (f) ebben. (darin)
11 | Paul irrt sich immer in den Daten. | Paul mindig téved a dátumokban.
12 | Sein Telefonnumber startet mit null, oder irre ich mich? | Az ő (f) telefonszáma nullával kezdődik, vagy tévedek?
13 | Er interessiert sich nicht für unsere Probleme. | Őt (f) nem érdekli a problémánk.
14 | Mein Chef irrt sich immer in dieser Sache. | A főnököm mindig téved ebben az ügyben. (e Sache)
15 | Freust du dich darüber? | Örülsz annak? (darüber)
\end{exmp}

\title{Visszaható igék/Visszaható igék részes esettel}

GrammarItemIndex = 543531

\begin{desc}
* sich anhören + A -----> meghallgat vmit
* sich ansehen + A -----> megnéz vmit
* sich merken + A ------> megjegyez vmit
* sich schaden mit + D -> árt magának vmivel
* sich verschaffen + A -> bescherez vmit

A visszaható névmás ragozása részes esetben:
mir  | magamnak
dir  | magadnak
sich | magának
uns  | magunknak
euch | magatoknak
sich | maguknak
\end{desc}

%Maklári: 158.oldal
\begin{exmp}
1 | Wir hören uns das Programm an. | Meghallgatjuk a programot.
2 | Siehst du dir den Film an? | Megnézed a filmet?
3 | Ich sehe mir das Theaterstück an. | Megnézem a színházi darabot. (s Theaterstück)
4 | Siehst du dir das Plakat? | Megnézed a plakátot? (s Plakat)
5 | Merkst du dir die Adresse? | Megjegyzed a címemet?
6 | Ich merke mir alle. | Én mindent megjegyzek. (alle)
7 | Merkt ihr euch die Regel? | Megjegyzitek a szabályt? (e Regel)
8 | Ich merke mir diesen Mensch. | Megjegyzem magamnak ezt az embert. (r Mensch)
9 | Schadest du dir nicht mit dem vielen Lehre? | Nem ártasz magadnak a sok tanulással? (e Lehre)
10 | Verschafft ihr euch noch zwei Kinokarten? | Beszereztek még két mozijegyet? (e Kinokarte)
11 | Ich verschaffe mir einen Wagen. | Beszerzek magamnak egy kocsit.
12 | Ich kann es mir schwer merken. | Ezt nehezen tudon megjegyezni.
13 | Hört ihr euch das Konzert an? | Meghallgatjátok a koncertet?
14 | Siehst du dir den Rundblick an? | Megnézed a panorámát? (r Rundblick)
15 | Er sieht sich jede Mädchen an. | Minden lányt megnéz (f) magának. (jede)
16 | Du hörst dir die Musik an. | Meghallgatod a zenét. (e Musik)
17 | Die Gäste sehen sich die Gemäldegalerie an. | A vendégek megnézik a képtárat. (e Gemäldegalerie)
18 | Der Polizist merkt sich das Kennzeichen des Wagens. | A rendőr megjegyzi a kocsi rendszámát. (s Kennzeichen)
19 | Hören wir uns diese Musik an? | Meghallgatjuk ezt a zenét?
20 | Ich kann es mir nicht merken. | Ezt nem tudom megjegyezni.
21 | Seht ihr euch morgen den Film an? | Megnézitek holnap a filmet?
22 | Du musst dir nur ein Telefonnumber merken. | Csak egy telefonszámot kell megjegyezned.
23 | Ich sehe mir dieses Ausstellen an. | Megnézem ezt a kiállitást. (s Ausstellen)
24 | Du verschadest dir mit dem Rauchen | Ártassz magadnak a dohányzással. (s Rauchen)
\end{exmp}

\title{Visszaható igék/Ikerigék}

GrammarItemIndex = 532558

\begin{desc}
Vannak olyan ikerigék, melyeknek van sich-es és sich nélküli formája is, mely jelentésüket is módosítja.
\end{desc}

\begin{exmp}
\end{exmp}

\title{Idők/Präteritum - elbeszélő múlt/Áttekintés}

GrammarItemIndex = 54358

\begin{desc}
Képzése:
* Gyenge igék: ige + te + személyrag
  Pl.: ich fragte 
       du fragtest 
       er/sie/es fragte 
       wir fragten 
       ihr fragtet 
       sie/Sie fragten
* Erős igék: szótő megváltozik.
* Vegyes igék: szótő megváltozik és -te képzőt is kap.
\end{desc}

\begin{exmp}
\end{exmp}

\title{Idők/Präteritum - elbeszélő múlt/Gyenge igék ragozása}

GrammarItemIndex = 42346

\begin{desc}
Képzése: ige + te + személyrag
Pl.:
ich fragte
du fragtest
er/sie/es fragte
wir fragten
ihr fragtet
sie/sie fragten

Ha a szótő t, d, n -re végződik, akkor a szótő egy -e hangot kap.
Pl.:
ich antwortete
du antwortetest
er/sie/es antwortete
wir antworteten
ihr antwortetet
sie/Sie antworteten
\end{desc}

%Maklári: 244. oldal
\begin{exmp}
1 | sie kaufte | vásárolt (n)
2 | ich suchte | kerestem
3 | ich sagte | mondtam
4 | er hörte | hallotta (f)
5 | sie machte | csinálta (n)
6 | ihr wartetet | vártatok
7 | er antwortete | felelt (f)
8 | ich arbeitete | dolgoztam
9 | sie kauften | vásároltak
10 | sie spielten | játszottak
11 | sie öffnete | kinyitotta (n)
12 | ich öffnete | kinyitottam
13 | ihr sagtet | mondtátok
14 | ihr öffnetet | kinyitottátok
15 | ihr hörtet | hallottátok
\end{exmp}

\title{Idők/Präteritum - elbeszélő múlt/Erős igék}

GrammarItemIndex = 615687

\begin{desc}
Erős igék esetében a szótő megváltozik.

ich fuhr
du fuhrst
er/sie/es fuhr
wir fuhren
ihr fuhrt
sie/Sie fuhren

Példák erős igékre:
* fahren -> fuhr
* singen -> sang
* schreiben -> schrieb
* essen -> aß
* trinken -> trank
* kommen -> kam
* gehen -> ging
* sprechen -> sprach
* bleiben -> blieb
* geben -> gab
\end{desc}

%Maklári: 245. oldal
\begin{exmp}
1 | wir schrieben | írtunk
2 | er aß | evett (f)
3 | ihr trankt | itatok
4 | wir fuhren | utazunk
5 | sie kamen | jöttek
6 | wir blieben | maradunk
7 | sie trank | ivott (n)
8 | er ging | ment (f)
9 | sie kam | jött (n)
10 | ihr gaben | adtatok
11 | wir sprachen | beszéltünk
12 | sie aßen | ettek
13 | ich kam | jöttem
14 | wir schrieben | írtunk
15 | du fuhrst | utaztál
16 | wir aßen | ettünk
17 | sie sang | énekelt (n)
18 | ihr schriebt | írtatok
19 | sie blieb | maradt (n)
20 | du gabst | adtál
21 | er schrieb | írt (f)
22 | er gab | adott (f)
23 | sie sprach | beszélt (n)
24 | sie schrieben | írtak
25 | ich sang | énekeltem
26 | er fuhr | utazott (f)
27 | du bliebst | maradtál
28 | ich trank | ittam
\end{exmp}

\title{Idők/Präteritum - elbeszélő múlt/Vegyes igék}

GrammarItemIndex = 543558

\begin{desc}
Vegyes igéknél a tőhang megváltozik és -te végződést is kap.

ich dachte
du dachtest
er/sie/es dachte
wir dachten
ihr dachtet
sie/Sie dachten

Példák vegyes igékre:
* denken -> dachte (gondol)
* nennen -> nannte (nevez)
* kennen -> kannte (ismer)
* bringen -> brachte (hoz)
* rennen -> rannte (rohan)
* wissen -> wusste (tud)
* brennen -> brannte (ég)
\end{desc}

%Maklári: 245. oldal
\begin{exmp}
1 | ich dachte | gondoltam
2 | er kannte | ismerte (f)
3 | ihr kanntet | ismertétek
4 | wir rannten | rohantunk
5 | sie wusste | tudta (n)
6 | ich brachte | hoztam
7 | es brannte | égett
8 | ihr wusstet | tudtátok
9 | ich nannte | neveztem
10 | er dachte | gondolta (f)
11 | ich wusste | tudtam
12 | sie nannte | nevezett (n)
13 | ich kannte | ismertem
14 | ihr brachtet | hoztátok
15 | wir dachten | gondoltuk
16 | du brachtest | hoztad
17 | wir kannten | ismertük
18 | du nanntest | neveztél
19 | wir wussten | tudtuk
20 | ihr dachtet | gondoltátok
21 | Sie dachten | gondolt (Ön)
22 | du kanntest | ismerted
23 | wir nannten | neveztünk
24 | sie wussten | tudták
25 | ihr nanntet | neveztétek
26 | du dachtest | gondoltad
27 | sie kannten | ismerték
28 | er brachte | hozta (f)
\end{exmp}

\title{Idők/Präteritum - elbeszélő múlt/A sein Präteriuma}

GrammarItemIndex = 4132

\begin{desc}
ich       | war
du        | warst
er/sie/es | war
wir       | waren
ihr       | wart
sie/Sie   | waren

Pl.: Ich war schon in Österreich. - Voltam már ausztriában.
\end{desc}

%Maklári: 249.oldal
\begin{exmp}
1 | Wo warst du? | Hol voltál?
2 | War Klara krank? | Beteg volt Klara?
3 | Wart ihr bei ihr? | Voltatok nála (n)?
4 | Sie war nett. | Kedves volt (n). (nett)
5 | Sie waren dort nicht. | Nem voltak ott.
6 | Wir waren frisch. | Frissek voltunk.
7 | Ihr wart schnell. | Gyorsak voltatok.
8 | Wie war die Speise? | Milyen volt az étel?
9 | Wo waren Sie gestern? | Hol volt Ön tegnap?
10 | Wann wart ihr in Kanada? | Mikor voltatok Kanadában?
11 | Wie war das Wetter? | Milyen volt az idő?
12 | Wart ihr böse auf mich? | Mérgesek voltatok rám?
13 | Waren Sie schon in dem Wald? | Volt (Ön) már az erdőben?
14 | Wir waren in Vien. | Bécsben voltunk.
15 | Ich war müde. | Fáradt voltam.
16 | Wir waren nicht bei ihnen. | Nem voltunk náluk.
17 | Wie war der Film? | Milyen volt a film?
18 | Wart ihr in dem Garten? | A kertben voltatok?
19 | Ich war gespannt darauf. | Kíváncsi voltam rá. (gespannt, darauf)
20 | Ich war dumm. | Buta voltam. (dumm)
21 | Wann war sie hier? | Mikor volt (n) itt?
\end{exmp}

\title{Idők/Präteritum - elbeszélő múlt/A haben Präteriuma}

GrammarItemIndex = 764658

\begin{desc}
ich hatte
du hattest
er/sie/es hatte
wir hatten
ihr hattet
sie hatten

Pl.: Ich hatte zwei Flöhe. - Kér bolhám volt.
\end{desc}

%Maklári: 250.oldal
\begin{exmp}
1 | Er hatte ein Haus. | Volt (f) egy háza.
2 | Hatten sie zehn Kinder? | Volt tíz gyerekük?
3 | Ich hatte Angst. | Féltem.
4 | Sie hatte Kopfschmerzen. | Fájt (n) a feje.
5 | Sie hatte Lust dazu. | Volt kedve (n) hozzá. (dazu)
6 | Hattet ihr Hausaufgabe? | Volt házifeladatotok?
7 | Hattest du Zeit? | Volt időd?
8 | Hatten sie Bier? | Volt sörük?
9 | Hattet ihr Angst? | Féltetek?
10 | Hattet ihr Bauchweh? | Fájt a hasatok? (Bauchweh haben)
11 | Hattet ihr Durst? | Szomjasak voltatok? (Durst haben)
12 | Hattest du Hunger? | Éhes voltál?
13 | Hattet ihr Geld? | Volt pénzetek?
14 | Ich hatte keine Idee. | Nem volt ötletem.
15 | Hattest du Buch? | Volt könyved?
\end{exmp}

\title{Idők/Präteritum - elbeszélő múlt/A werden Präteriuma}

GrammarItemIndex = 6721

\begin{desc}
ich wurde
du wurdes
er/sie/es wurde
wir wurden
ihr wurdet
si wurden

Pl.: Er wurde böse. - Mérges lett.
\end{desc}

%Maklári: 250.oldal
\begin{exmp}
1 | Mein Freund wurde berühmt. | A barátom híres lett. (berühmt)
2 | Er wurde arm. | Szegény lett (f).
3 | Wir wurden reich. | Gazdagok lettünk.
4 | Sie wurde gesund. | Egészséges lett (n).
5 | Es wurde dunkel. | Sötét lett. (dunkel)
6 | Es wurde kalt. | Hideg lett.
7 | Ich wurde fertig nicht. | Nem lettem kész.
8 | Wir wurden alt und hässlich. | Öregek lettünk és csúnyák. (hässlich)
9 | Wurden Sie krank? | Beteg lett (Ön)?
10 | Die Tage wurden länger. | A napok hosszabbak lettek.
11 | Die Suppe wurde sauer. | A leves megsavanyodott. (sauer)
12 | Es wurde hell in dem Zimmer. | Világos lett a szobában. (hell)
13 | Wurden sie müde? | Fáradtak lettek?
14 | Er wurde Stein. | Kővé vált (f). (r Stein)
15 | Ich wurde Soldat. | Katona lettem.
16 | Mátyás wurde Soldat. | Mátyás katona lett.
17 | Józsi wurde böse. | Józsi mérges lett.
18 | Es wurde Winter. | Tél lett.
\end{exmp}

\title{Idők/Präteritum - elbeszélő múlt/Módbeli segédigék Präteriuma}

GrammarItemIndex = 863147

\begin{desc}
\end{desc}

\begin{exmp}
\end{exmp}

\title{Idők/Perfekt - befelyezett jelen/Áttekintés}

GrammarItemIndex = 54539

\begin{desc}
Ebben a időben kifelyezett események kihatnak a jelenre.
\begin{center}
alany + segédige (haben/sein) + többi mondatrész + ige befelyezett alakja
\end{center}

Pl.: Ich bin von euch gekommen. - Tőletek jöttem.
\end{desc}

\begin{exmp}
\end{exmp}

\title{Idők/Perfekt - befelyezett jelen/A sein perfektje}

GrammarItemIndex = 16354

\begin{desc}
ich bin gewesen - voltam
du bist gewesen - voltál
er/sie/es ist gewesen - volt
wir sind gewesen - voltatak
ihr seid gewesen - voltatok
sie/Sie sind gewesen - voltak
\end{desc}

%Maklári: 236. oldal
\begin{exmp}
1 | Ich bin schon bei ihm gewesen. | Voltam már nála (f).
2 | Sie ist müde gestern gewesen. | Fáradt volt (n) tegnap.
3 | Er ist nicht zu Hause gewesen. | Nem volt (f) otthon.
4 | Ihr seid klug gewesen. | Okosak voltatok.
5 | Ist sie dort auch gewesen? | Ott volt ő (n) is?
6 | Warum bist du nicht dort gewesen? | Miért nem voltál ott?
7 | Jürgen ist nicht bei ihr gewesen. | Jürgen nem volt nála (n).
8 | Wo sind Sie gestern gewesen? | Hol volt (Ön) tegnap?
9 | Ist Anna krank gewesen? | Beteg volt Anna?
10 | Der Film ist langweilig gewesen. | Unalmas volt a film.
11 | Seid ihr in dem Haus nicht gewesen? | Nem voltatok a házban?
12 | Wo seid ihr gestern gewesen? | Hol voltatok tegnap?
13 | Wann bist du in dem Theater gewesen? | Mikor voltál színházban?
14 | Ist der Film schlecht gewesen? | Rossz volt a film?
15 | Ich bin in der Schule nicht gewesen. | Nem voltam iskolában.
16 | Sind sie durstig gewesen? | Szomjasak voltak?
17 | Sind die Gäste hungrig gewesen? | Éhesek voltak a vendégek?
\end{exmp}

\title{Idők/Perfekt - befelyezett jelen/A haben perfektje}

GrammarItemIndex = 56423

\begin{desc}

ich habe gehabt - nekem volt
du hast gehabt - neked volt
er/sie/es hat gehabt - neki volt
wir haben gehabt - nekünk volt
ihr habt gehabt - nektek volt
sie/Sie haben gehabt - nekik volt

\end{desc}

\begin{exmp}
\end{exmp}

\title{Idők/Perfekt - befelyezett jelen/Elváló igekötős igék perfektje}

GrammarItemIndex = 274216

\begin{desc}
Elváló igekötős igéknél az igekötő és az ige közé tesszük a ge szócskát.

Pl.: Jürgen ist gestern angekommen. - Jürgen tegnap érkezett meg.

* ankommen -> ist angekommen = megérkezett
* aufstehen -> ist aufgestanden = felket
* einschlafen -> ist eingeschlafen = elaludt
* einkaufen -> hat eingekauft = bevásárolt
* fernsehen -> hat ferngesehen = tévézett
* weggehen -> ist weggegangen = elment
\end{desc}

%Maklári: 233. oldal
\begin{exmp}
1 | Ich bin um 5 Uhr aufgestanden. | 5 órakor keltem fel.
2 | Wann seid ihr aufgestanden? | Mikor keltetek fel?
3 | Ich bin in der Schule eingeschlafen. | Elaludtam az iskolában.
4 | Wir haben alles eingekauft. | Mindent bevásároltunk.
5 | Er hat gestern viel ferngesehen. | Tegnap sokat tévézett (f).
6 | Seid ihr schon aufgestanden? | Már felkeltetek?
7 | Sind die Gäste schon weggegangen? | Elmentek már a vendégek?
8 | Der Zug ist schon angekommen. | A vonat már megérkezett.
9 | Wann sind sie auf den Bahnsteig angekommen? | Mikor érkeztek (ők) meg a peronra?
10 | Wir haben immer ferngesehen. | Mindig tévéztünk.
11 | Józsi ist schon in die Schule weggegangen. | Józsi már elment az iskolába.
12 | Ich habe alles mit meinem Bruder eingekauft. | Mindent bevásároltam a testvéremmel (f).
13 | Wo habt ihr auf das Wochenende eingekauft? | Hol vásároltatok be a hétvégére? (s Wochenende)
14 | Ich bin schon vor 4 Uhr aufgestanden. | Már 4 óra előtt felkeltem.
\end{exmp}

\title{Idők/Perfekt - befelyezett jelen/Módbeli segédigék perfekte/Ha a módbeli segédige önmagában áll}

GrammarItemIndex = 412426

\begin{desc}
Képzése:
alany + haben ragozva + többi mondatrész + módbli segédige Perfektje

* wollen -> hat gewollt
* müssen -> hat gemusst
* sollen -> hat gesollt
* dürfen -> hat gedurft
* mögen -> hat gemocht
* können -> hat gekonnt

Pl.: Ich habe das nicht gewollt. - Nem akartam azt.
\end{desc}

%Maklári: 240. oldal
\begin{exmp}
1 | Hans hat den Chef nicht gemocht. | Hans nem szerette a főnököt.
2 | Wir haben das nicht gekonnt. | Nem tudtuk azt.
3 | Sie haben auswendig das Gedicht gekonnt. | Kívülről tudták a verset. (auswendig, s Gedicht)
4 | Ich habe ihn nicht gemocht. | Nem szerettem őt (f).
5 | Sie haben nichts gewollt. | Semmit sem akartak.
6 | Sie hat alles gedurft. | Minden szabad volt neki (n).
7 | Hans hat alles gekonnt. | Hans mindent tudott.
8 | Sie hat das nicht gemocht. | Azt nem szerette (n).
9 | Ich habe das nicht gewollt. | Nem akartam azt.
10 | Ich habe die Hunde nicht gemocht. | Nem szerettem a kutyákat.
\end{exmp}

\title{Idők/Perfekt - befelyezett jelen/Módbeli segédigék perfekte/Ha a módbeli segédige igével áll}

GrammarItemIndex = 423526

\begin{desc}
Képzése:
alany + haben ragozott alakja + többi mondatrész + ige Infinitivben + módbeli segédige Infinitivben

Pl.: Ich habe mit ihm nicht spielen wollen. - Nem akartam vele játszani.
\end{desc}

%Maklári: 240.oldal
\begin{exmp}
1 | Er hat uns nicht sehen wollen. | Nem akart (f) látni bennünket.
2 | Ich habe in dem Zimmer nicht rauchen dürfen. | A szobában nem volt szabad dohányoznom.
3 | Habt ihr in das Schwimmbad gehen dürfen? | Szabad volt az uszodába mennetek?
4 | Sie haben nicht spülen mögen. | Nem szerettek (ők) mosogatni.
5 | Wann habt ihr nach Hause gehen müssen? | Mikor kellet haza mennetek?
6 | Ich habe in der Stadt spazieren mögen. | Szerettem sétálni a városban.
7 | Ich habe mit Puskás Fußball spielen wollen. | Focizni akartam Puskással.
8 | Sie hat mit mir sprechen wollen. | Beszélni akart (n) velem.
9 | Er hat seine Hausaufgabe schreiben müssen. | Meg kellett írnia (f) a házi feladatát.
10 | Ich habe von dir etwas fragen wollen. | Akartam kérdezni tőled valamit.
11 | Ich habe früh in das Bett gehen müssen. | Korán ágyba kellet mennem.
12 | Wann hat er dich erwachen wollen? | Mikor akart (f) felébreszteni téged? (erwachen - felébreszt)
13 | Wir haben ihr nicht helfen können. | Nem tudtunk neki (n) segíteni.
14 | Der Krank hat schon nicht aufstehen dürfen. | A betegnek még nem szabadott felkelnie.
15 | Ich habe nich gut antworten können. | Nem tudtam jól válaszolni.
16 | Er hat seine Suppe essen müssen. | Meg kelett (f) ennie a levesét.
\end{exmp}

\title{Idők/Perfekt - befelyezett jelen/Kettős infinitiv szerkezet perfektben}

GrammarItemIndex = 432478

\begin{desc}
Képzése:
alany + haben ragozva + többi mondatrész + ige 1 Infinitivben + ige 2 Infinitivben

Pl.: Ich habe ihn kommen hören. - Hallottam őt jönni.
\end{desc}

%Maklári: 243. oldal
\begin{exmp}
1 | Ich habe ihn in einem Restaurant essen sehen. | Láttam őt (f) enni egy étteremben.
2 | Wer hat dich schreiben lehren? | Ki tanított téged írni?
3 | Sie hat gestern nicht kommen wollen. | Nem akart jönni (n) tegnap.
4 | Wo habt ihr spazieren gehen? | Hová mentetek sétálni? 
5 | Habt ihr sie singen hören? | Hallottátok már őt (n) énekelni?
6 | Sie hat viel nicht bleiben wollen. | Nem akart sokat maradni.
7 | Ich habe deinen Freund Piano spielen hören. | Hallottam zongorázni a barátodat. (Piano spielen)
8 | Ihr habt den Bus kommen hören. | Hallottátok jönni a buszt.
9 | Hast du sie Piano spielen lehren? | Te tanítottad őt (n) zongorázni? (Piano spielen)
10 | Wir haben ihm in der Mathematik helfen wollen. | Akartunk segíteni neki (f) a matematikában. (e Mathematik)
11 | Wir haben den Zug nicht stoppen sehen. | Nem láttuk a vonatot megállni.
12 | Ich habe schon Ági nicht Gitare spielen hören. | Nem hallotam még Ágit gitározni. (Gitare spielen)
13 | Wer hat Ági Gitare spielen lehren? | Ki tanította Ágit gitározni? (Gitare spielen)
14 | Wann habt ihr in das Kino gehen wollen? | Mikor akartatok moziba menni?
15 | Wo habt ihr kennen lernen? | Hol ismerkedtetek meg? (kennen lernen)
16 | Woher hat sie spazieren gehen? | Hová ment (n) sétálni?
17 | Wo hast du ihn spazieren sehen? | Hol látad őt (f) sétálni?
18 | Ich habe hier sie spazieren sehen. | Itt láttam őt (n) sétálni.
19 | Habt ihr unsere Freunde kommen sehen? | Láttátok jönni a barátainkat?
\end{exmp}

\title{Idők/Plusquamerfekt - előidejűség a múltban}

GrammarItemIndex = 654685

\begin{desc}
Plusquamerfektet akkor használjuk, ha egy múltbeli esemény előtt történt esemény akarunk kifejezni.

Képzése:
alany + haben/sein Präteriuma + többi mondatrész + ige Perfektje

Pl.: Zuerst war ich ins Geschäft gegangen, dann aß ich Mittag. - Először elmentem a boltba, aztán ebédeltem.
\end{desc}

%Maklári: 225. oldal
\begin{exmp}
1 | Mutter hatte die Kleider gewascht und sie brachte sie in das Zimmer ein. | Anya kimosta a ruhákat és behozta őket a szobába. (einbringen)
2 | Ich hatte auf der Straße gefallen und meine Nase schmerzte danach. | Elestem az utcán és utánna fájt az orrom. (e Nase, schmerzen, danach)
3 | Er hatte meinen Wein getrunken und er brüllte danach. | Megitta a boromat (f) és kiabált utána. (brüllen, danach)
4 | Wir hatten vorgestern viel gearbeitet und wir hatten heute Kopfschmerzen. | Tegnapelőtt sokat dolgoztunk és ma fájt a fejünk. (vorgestern)
5 | Sie hatte zwei Stunden geschwommen und sie ging danach nach Hause. | Két órát úszott (n) és utána haza ment. (danach)
6 | Ich hatte Rückenschmerzen vormittag, denn ich hatte gestern viel gearbeitet. | Fájt a hátam délelőtt, mert tegnap sokat dolgoztam. (Rückenschmerzen, vormittag, denn)
7 | Sie hatte zwei Tabletten eingenommen und sie schlieft bis heute morgen. | Bevett (n) két tablettát és ma reggelig aludt. (einnehmen, schlafen)
8 | Der Lehrer hatte etwas gesagt, dann ging er nach Hause. | A tanár mondott valamit, aztán hazament. (dann)
9 | Es war kalt in der Raum, denn jemand hatte früher das Fenster geöffnet. | Hideg volt a teremben, mert valaki korábban kinyitotta az ablakot. (e Raum, denn, jemand, früher)
10 | Wir aßen drei Teller Suppe, denn wir hatten nicht gefrühstückt. | Ettünk három tányér levest, mert nem reggeliztünk. (r Teller, denn, frühstücken)
11 | Pista rauchte nicht, denn der Arzt hatte ihm verboten. | Pista nem dohányzott, mert megtiltotta neki az orvos. (denn, verbieten)
12 | Der Lehrer gab meine Schularbeit zurück, denn Józsi hatte mir geschrieben. | A tanár visszaadta a dolgozatomat, mert Józsi írta neken. (zurückgeben, e Schularbeit)
13 | Die Gäste hatten viel gegessen, dann gingen sie in ihre Zimmer. | A vendégek sokat ettek, aztán a szobáikba mentek. (dann)
14 | Der Mechaniker hatte den Wagen repariert, aber es startete nicht. | A szerelő megszerelte a kocsit, de nem indult el. (start)
15 | Anna kam auf die Party nicht, denn ich hatte sie davor nicht angerufen. | Anna nem jött a partira, mert nem hívtam fel őt előtte. (denn, davor, anrufen)
\end{exmp}

\title{Idők/A jövő idő - Futur I}

GrammarItemIndex = 423429

\begin{desc}
Ezt a jövő idős szerkezetet akkor használjuk, ha azt akarjuk kifejezni, hogy valami mindenképpen megtörténik a jövőben. Ez magyar fog + főnévi igenév szerkezetnek felel meg.

Képzése: alany + werden ragozott alakja + többi mondatrész + ige főnévi igenév alakban

werden ragozása:
ich werde
du wirst
er/sie/es wird
wir werden
ihr werdet
sie/Sie werden

Pl.: Klaus wird sicherlich mein Heus kaufen. - Klaus biztosen meg fogja venni a házamat.
\end{desc}

%Maklári 266.oldal
\begin{exmp}
1 | Wir werden euch besuchen. | Meg fogunk látogatni titeket.
2 | Werdet ihr umziehen? | El fogtok költözni? (umziehen)
3 | Mein Freund wird ein Buch schreiben. | A barátom egy könyvet fog írni.
4 | Wirst du dein Auto verkaufen? | El fogod adni az autódat?
5 | Wann wirst du deine Hausaufgabe schreiben? | Mikor fogod megírni a házifeladatodat?
6 | Wird sie bei uns wohnen? | Nálunk fog (n) lakni?
7 | Sie werden in dem Hotel übernachten. | A hotelben fognak éjszakázni. (übernachten)
8 | Wann wirst du Ordnung machen? | Mikor fogsz rendet csinálni?
9 | Sie werden um 6 Uhr aufstehen. | 6 órakor fognak felkelni.
10 | Wo werdet ihr Urlaub machen? | Hol fogtok nyaralni? (Urlaub machen)
11 | Der Zug wird um 5 Uhr ankommen. | A vonat 5 órakor fog megérkezni.
12 | Wann wirst du aufstehen? | Mikor fogsz felkelni?
13 | Mein Nachbar wird nach Italien reisen. | A szomszédom Olaszországba fog utazni. (reisen)
14 | Mutter wird für mich kochen. | Anya fog nekem főzni. (für)
15 | Wann wird dieser Mann weggehen? | Mikor fog elmenni ez az ember? (weggehen)
16 | Ich werde morgen kommen. | Holnap fogok jönni.
17 | Wir werden sie besuchen. | Meg fogjuk őt (n) látogatni.
\end{exmp}

\title{Feltételes mód/Feltételes jelenidő/A würden + Infinitiv}

GrammarItemIndex = 43249

\begin{desc}
Képzése: alany + würden ragozva + többi mondatrész + főnévi igenév.

würden ragozása:
ich würde
du würdest
er/sie/es würde
wir würden
ihr würdet
sie würden

Pl.: Ich würde gern morgen zu euch fahren. - Szívesen elutaznék holnap hozzátok.
\end{desc}

%Maklári: 269.oldal
\begin{exmp}
1 | Ich würde nach Hause gehen. | Haza mennék.
2 | Sie würden uns besuchen. | Meglátogatnának minket.
3 | Würdet ihr zu uns kommen? | Eljönnétek hozzánk?
4 | Würdet ihr die Hausaufgabe schreiben? | Megírnátok a házifeladatot?
5 | Ich würde nicht rauchen. | Nem dohányoznék.
6 | Mein Freund würde sein Auto verkaufen. | A barátom eladná az autóját.
7 | Würdest du eine Zeitung kaufen? | Vennél egy újságot? (kaufen)
8 | Würdet ihr ein Brot bringen? | Hoznátok egy kenyeret? (bringen)
9 | Er würde eher schlafen. | Inkább aludna (f).
10 | Ich würde gern schlafen. | Szívesen aludnék.
11 | Ich würde ihm eine Ohrfeige geben. | Adnék neki (f) egy pofont.
12 | Wir würden zu der Oma gehen. | Elmennénk a nagyihoz.
13 | Ich würde zwei Kilogramm Apfel kaufen. | Vásárolnék két kilogramm almát
14 | Die Lehrerin würde die Schularbeit korrigieren. | A tanárnő kijavítaná a dolgozatot. (e Schularbeit, korrigieren)
15 | Der Mechaniker würde meinen Wagen reparieren. | A szerelő megjavítaná a kocsimat.
16 | Würdet ihr morgen in das Kino gehen? | Jönnétek holnap moziba?
17 | Würden Sie mir es glauben? | Elhinné (Ön) ezt nekem?
18 | Würdet ihr Ordnung machen? | Rendet csinálnátok?
19 | Wir würden alles einkaufen. | Mindent bevásárolnánk. (einkaufen)
20 | Ich würde in das Kino gehen. | Moziba mennék.
\end{exmp}

\title{Feltételes mód/Feltételes jelenidő/Konjunktiv II/Haben és sein Konjunktiv II alakja}

GrammarItemIndex = 679315

\begin{desc}
ich wäre       | lennék
du wärest      | lennél
er/sie/es wäre | lenne
wir wären      | lennénk
ihr wäret      | lennétek
sie/Sie wären  | lennének

ich hätte       | nekem lenne
du hättest      | neked lenne
er/sie/es hätte | neki lenne
wir hätten      | nekünk lenne
ihr hättet      | nektek lenne
sie/Sie hätten  | nekik lenne
\end{desc}

\begin{exmp}
1 | Ich wäre klug. | Okos lennék.
2 | Er wäre reich. | Gazdag (f) lenne.
3 | Wir wären zu Hause. | Otthon lennénk.
4 | Wo wäret ihr? | Hol lennétek?
5 | Wir wären traurig. | Szomorúak lennénk.
6 | Sie wäre schläfrig. | Álmos lenne (n). (schläfrig)
7 | Das Wetter wäre schön. | Szép lenne az idő.
8 | Warum wäre es gut? | Miért lenne ez jó?
9 | Ihr wäret arm. | Szegények lennétek.
10 | Jürgen wäre lustig. | Jürgen vidám lenne. (lustig - vidám)
11 | Sie wären in dem Büro. | Önök az irodában lennének.
12 | Wann wäre der Film? | Mikor lenne a film?
13 | Sie hätte zwei Kinder. | Két gyereke lenne (n).
14 | Ich hätte Geld. | Lenne pénzem.
15 | Hättet ihr Zeit? | Lenne időtök?
16 | Er hätte keine Zeit. | Nem lenne ideje (f).
17 | Sie hätte Hunger. | Éhes lenne (n). (Hunger haben)
18 | Er hätte Recht. | Igaza lenne (f). (Recht haben)
19 | Wir hätten Sorgen. | Gondjaink lennének.
20 | Ich hätte keine Hausaufgabe. | Nem lenne házifeladatom.
21 | Ihr hättet Wohnung. | Lenne lakásotok.
22 | Er hätte Stellung. | Lenne állása (f).
23 | Ihr hättet einen Wagen. | Lenne egy kocsitok.
24 | Wir hätten dazu Lust. | Lenne hozzá kedvünk.
\end{exmp}

\title{Feltételes mód/Feltételes jelenidő/Konjunktiv II/Módbeli segédigék konjunktív II alakja}

GrammarItemIndex = 653147

\begin{desc}
A módbeli segédigék feltételes módja csak konjunktiv II szerkezettel képezhető.

müssen -> müsste = kellene
sollen -> sollte = kellene
können -> könnte = tudna
wollen -> wollte = akarna
dürfen -> dürfte = szabadna
mögen -> möchte = szeretne

wollte ige ragozása:
ich wolte
du woltest
er/sie/es wolte
wir wolten
ihr woltet
sie/Sie wolten

Képzése: alany + módbeli segédige ragozva + többi mondatrész + ige infinitivben

Pl.: Ich müsste schon nach Hause gehen. - Már haza kellene mennem.
\end{desc}

% Maklári: 277.oldal
\begin{exmp}
1 | Wir müssten einkaufen. | Be kellene vásárolnunk. (einkaufen)
2 | Ich müsste Pista anrufen. | Fel kellene hívnom Pistát. (anrufen)
3 | Müsstest du zwei Stunden auf sie warten? | Két órát kellene várnod rá (n)? (warten auf + A)
4 | Möchten Sie ein Glas Wasser trinken? | Szeretne Ön egy pohár vizet inni?
5 | Wollte er uns besuchen? | Meg akarna (f) látogatni minket?
6 | Was möchtest du morgen essen? | Mit szeretnél holnap enni?
7 | Wo wollten sie Mittag essen? | Hol akarnának ők ebédelni? (Mittag essen)
8 | Wann dürften wir spazieren? | Mikor szabadna sétálnunk?
9 | Du müsstest sie anrufen. | Fel kellene hívnod őt (n).
10 | Wir müssten ihm helfen. | Nekünk kellene neki (f) segíteni.
11 | Vater wollte uns überraschen. | Apa meg akarna lepni minket. (überraschen)
12 | Sie dürfte zu euch nicht gehen. | Nem szabadna (n) hozzátok mennie.
13 | Ich könnte so viel nicht schwimmen. | Nem tudnék olyan sokat úszni.
14 | Wann müssten wir aufstehen? | Mikor kellene felkelnünk?
15 | Ihr dürftet nicht in Schwimmbad gehen. | Nem szabadna uszodába mennetek. (s Schwimmbad)
16 | Die Gäste könnten heute kommen. | A vendégek tudnának ma jönni.
17 | Wie könntest du den Satz übersetzen? | Hogyan tudnád lefordítani a mondatot?
18 | Ich müsste Blume kaufen. | Virágot kellene vásárolnom.
\end{exmp}

\title{Feltételes mód/Feltételes múlt idő/Feltételes múlt idő általában}

GrammarItemIndex = 423563

\begin{desc}
Képzése:
alany + hätte/wäre ragozva + többi mondatrész + ige befelyezett alakja

Pl.: * Ich hätte gelernt. - Tanultam volna.
* Ich wäre gekommen. - Jöttem volna.
\end{desc}

%Maklári: 282. oldal
\begin{exmp}
1 | Er hätte geholfen. | Segített volna (f).
2 | Hättest du geschlafen? | Aludtál volna?
3 | Wir hätten Fußball gespielt. | Fociztunk volna.
4 | Was wäre dann geschehen? | Mi történt volna akkor? (geschehen, dann)
5 | Sie hätte noch gegessen. | Evett volna (n) még.
6 | Ich wäre noch spaziert. | Sétáltam volna még.
7 | Die Gäste hätten gegessen. | A vendégek ettek volna.
8 | Wir wären nach Österreich geflogen. | Austriába repültünk volna.
9 | Wäre die Prüfung gelungen? | Sikerült volna a vizsga?
10 | Sie hätte mich besucht. | Meglátogatott (n) volna.
11 | Ich wäre noch geschwommen. | Úsztam volna még.
12 | Sie hätten ein Buch geschrieben. | Írtak volna egy könyvet.
13 | Hättet ihr ferngesehen? | Néztetek volna TV-t?
14 | Wärest du traurig gewesen? | Szomorú lettél volna?
15 | Wärest du bei uns geblieben? | Nálunk maradtál volna?
16 | Ági hätte Piano gespielt. | Ági zongorázott volna.
17 | Er wäre rechtzeitig aufgestanden. | Időben felkelt (f) volna. (rechtzeitig)
18 | Ich hätte die Hausaufgabe geschrieben. | Megírtam volna a házit.
19 | Ich hätte Karten auf dem Zug gespielt. | Kártyáztam volna a vonaton. (Karten spielen)
20 | Ich wäre noch geblieben. | Maradtam volna még.
21 | Sie wäre nach Hause gelaufen. | Futott (n) volna még.
22 | Er wäre heute gekommen. | Ma jött (f) volna.
\end{exmp}

\title{Feltételes mód/Feltételes múlt idő/A sein feltételes múlt ideje}

GrammarItemIndex = 563497

\begin{desc}
Képzése:
alany + wäre ragozva + többi mondatrész + gewesen

wären ragozása:
ich wäre
du wärest
er/sie/es wäre
wir wären
ihr wäret
sie/Sie wären

Pl.: Er wäre nervös gewesen. - Ideges lett volna.
\end{desc}

%Maklári: 284. oldal
\begin{exmp}
1 | Ich wäre traurig gewesen. | Szomorú lettem volna.
2 | Sie wären zu Hause gewesen. | Otthon lettek volna.
3 | Wärest du mit uns gewesen? | Velünk lettél volna?
4 | Ich wäre frisch gewesen. | Friss lettem volna.
5 | Wären sie lieb gewesen? | Kedvesek lettek volna? (lieb)
6 | Sie wären um 2 Uhr bei mir gewesen. | 2 órakor nálam lettek volna.
7 | Wie wäre die Vorstellung gewesen? | Milyen lett volna az előadás? (e Vorstellung)
8 | Wären sie in dem Auto gewesen? | Az autóban lettek volna?
9 | Wir wären durstig gewesen. | Szomjasak lettünk volna.
10 | Wir wären müde gewesen. | Fáradtak lettünk volna.
11 | Er wäre in der Bank gewesen. | Bankban lett volna (f).
12 | Wo wären sie gewesen? | Hol lettek volna (ők)?
13 | Es wäre derart gewesen. | Úgy lett volna. (derart)
14 | Sie wären hungrig gewesen. | Éhesek lettek volna.
15 | Sie wäre froh gewesen. | Boldog lett (n) volna. (froh)
\end{exmp}

\title{Feltételes mód/Feltételes múlt idő/A haben feltételes múlt ideje}

GrammarItemIndex = 156132

\begin{desc}
Képzése:
alany + hätte ragozva + többi mondatrész + gehabt

hätten ragozása:
ich hätte
du hättest
er/sie/es hätte
wir hätten
ihr hättet
sie/Sie hätten

Pl.: Ich hätte dazu Lust gehabt. - Lett volna hozzá kedvem.
\end{desc}

%Maklári: 284. oldal
\begin{exmp}
1 | Er hätte Angst gehabt. | Félt (f) volna.
2 | Ihr hättet Zeit gehabt. | Lett volna időtök.
3 | Sie hätte Kopfschmerzen gehabt. | Fájt volna (n) a feje.
4 | Ihr hättet Angst gehabt. | Féltetek volna.
5 | Wir hätten Lust gehabt. | Lett volna kedvünk.
6 | Er hätte Mut dazu gehabt. | Lett volna bátorsága (f) a hozzá. (Mut haben dazu)
7 | Sie hätten Fahrrad gehabt. | Lett volna kerékpárjuk.
8 | Ich hätte wenig Zeit dazu gehabt. | Kevés időm lett volna hozzá. (dazu)
9 | Hättet ihr Geld nicht gehabt? | Nem lett volna pénzetek?
10 | Wir hätten einen Garten gehabt. | Lett volna egy kertünk.
11 | Ich hätte Kopfschmerzen nicht gehabt. | Nem fájt volna a fejem.
12 | Sie hätte zwei Kinder gehabt. | Két gyermeke (n) lett volna.
13 | Hättest du Geduld gehabt? | Lett volna türelmed?
14 | Ihr hättet Recht gehabt. | Igazatok lett volna. (Recht haben)
15 | Sie hätte Sorgen gehabt. | Gondjai (n) lettek volna. (Sorgen haben)
\end{exmp}

\title{Feltételes mód/Feltételes múlt idő/A wenn-es mellékmondatok feltételes múlt időben}

GrammarItemIndex = 146132

\begin{desc}
Pl: Wenn du sie gesehen hättest, wärest du in Ohnmacht gefallen. - Ha láttad volna őt elájultál volna.
\end{desc}

\begin{exmp}
\end{exmp}

\title{Feltételes mód/Feltételes múlt idő/Módbeli segédigék feltételes múlt ideje}

GrammarItemIndex = 146423

\begin{desc}
Képzése: kettős Infinitiv szerkezettel
alany + hätte ragozva + többi mondatrész + ige Infinitivben + módbeli segédige Infinitivben

hätten ragozása:
ich hätte
du hättest
er/sie/es hätte
wir hätten
ihr hättet
sie/Sie hätten

Pl.: Ich hätte viel lernen müssen. - Sokat kellet volna tanulnom.
\end{desc}

%Maklári: 286. oldal
\begin{exmp}
1 | Ich hätte in das Geschäft gehen müssen. | El kellet volna mennem a boltba.
2 | Ich hätte Pista anrufen müssen. | Fel kellet volna hívnom Pistát.
3 | Er hätte den Motor reparieren können. | Meg tudta volna (f) javítani a motort.
4 | Ich hätte das Brief schreiben wollen. | Meg akartam volna írni a levelet.
5 | Wir hätten in den Park ausgehen können. | Kimehettünk volna a parkba. (ausgehen)
6 | Ich hätte sie willkommen müssen. | Üdvözölnöm kellet volna őt (n). (willkommen)
7 | Er hätte uns einladen wollen. | Meg akart volna (f) hívni minket. (einkladen)
8 | Sie hätten mit uns nicht kommen wollen. | Nem akartak volna velünk jönni.
9 | Du hättest die Hausaufgabe schreiben müssen. | Meg kellett volna írnod a házi feladatot.
10 | Wir hätten gestern in das Kino gehen können. | Tegnap elmehettünk volna moziba.
11 | Du hättest ihr Blume nicht kaufen müssen. | Nem kellet volna neki (n) virágot venned.
12 | Ihr hättet die Kinder bringen können. | El tudtátok volna hozni a gyerekeket. (bringen)
13 | Er hätte die Adresse nicht verlieren müssen. | Nem kellet volna elveszítenie (f) a címet. (verlieren, e Adresse)
14 | Wir hätten die Aufgabe lösen können. | Meg tudtuk volna oldani a feladatot.
15 | Du hättest von ihr ihre Name fragen können. | Megkérdezhetted volna tőle (n) az ő (n) nevét.
16 | Der Verkäufer hätte den Schuh umtauschen können. | Az eladó kicserélhette volna a cipőt. (r Schuh, umtauschen)
17 | Du hättest von Heim nicht weggehen dürfen. | Nem szabadott volna elmenned otthonról. (s Heim, weggehen)
\end{exmp}

\title{Cselekvő passiv mondatszerkezet/Passiv jelen időben}

GrammarItemIndex = 52351

\begin{desc}
alany + werden ragozva + többi mondatrész + ige Partizip Perfekt alakja

Pl.: Ein Haus wird von Peter gebaut. - "Egy ház péter által van építve."

Az alanyt kifejezhetjük egy passzív mondatban a következő módon:
* von + D: ha élő a cselekvés okozója,
* durch + A: ha élettelen a cselekvés okozója.

Pl.: * Der Tisch wird von Oma gedeckt. - Az asztalt nyagyi teríti meg.
* Oma wird durch das Heilwasser geheilt. - A gyógyvíz gyógyítja meg nagyit.

werden ragozása:
ich werde
du wirst
er/sie/es wird
wir werden
ihr werdet
sie/Sie werden
\end{desc}

%Maklári: 325. oldal
\begin{exmp}
1 | Das Auto wird von dem Mechaniker repariert. | A szerelő megjavítja az autót.
2 | Ein Brief wird von meinem Freund geschrieben. | A barátom ír egy levelet.
3 | Der Text wird von uns übersetzen. | Lefordítjuk a szöveget.
4 | Das Ticket wird durch die Maschine kontrolliert. | A gép ellenőrzi a jegyet. (kontrollieren)
5 | Die Familie wird durch die Polizei benachrichtigt. | A rendőrség értesíti a családot. (benachrichtigen)
6 | Deutsch wird in Deutschland gesprochen. | Németországban németül beszélnek.
7 | Die Tür wird von dem Gast geöffnet. | A vendég kinyitja az ajtót.
8 | Das Gebäude wird durch eine Firma gebaut. | Az épületet egy cég építi. (s Gebäude)
9 | Das Radio wird von dem Großvater repariert. | A nagyapa megszereli a rádiót.
10 | Die Rechnungen werden von mir gezahlt. | A számlákat én fizetem. (e Rechnung, zahlen)
11 | Die Wand wird von den Arbeitern gemalt. | A munkások befestik a falat.
12 | Die Schularbeiten werden von der Lehrerin ausgeteilt. | A tanárnő kiosztja a dolgozatokat. (e Schularbeit, austeilen)
13 | Die Kranken werden von der Pflegerin gedient. | A betegeket az ápolónő szolgálja ki. (e Kranke (tsz), dienen)
14 | Wasser wird von den Freiwilligen geteilt. | Vizet osztanak az önkéntesek. (teilen - oszt)
15 | Mein Freund wird durch das Medikament geheilt. | A gyógyszer meggyógyitja a barátomat. (s Medikament)
16 | Die Gäste werden von Imre abgeholt. | Imre hozza el a vendégeket. (abholen)
\end{exmp}

\title{Cselekvő passiv mondatszerkezet/Passiv elbeszélő múlt időben}

GrammarItemIndex = 423216

\begin{desc}
Képzése: alany + wurden + többi mondatrész + ige befelyezett alakja

Pl.: Ein Haus wurde von Peter gebaut. - "Egy ház péter által volt építve."

wurden ragozása:
ich wurde
du wurdest
er/sie/es wurde
wir wurden
ihr wurdet
sie/Sie wurden
\end{desc}

%Maklári: 330.oldal
\begin{exmp}
1 | Das Lied wurde von den Kindern gesungen. | A dalt gyerekek énekelték.
2 | Viel wurde in dem Büro gesprochen. | Az irodában sokat beszéltek.
3 | Karten wurden in der Pause gespielt. | A szünetben kártyáztak. (Karten spielen)
4 | Das Geschäft wurde von dem Verkäufer geöffnet. | Az eladó kinyitotta az üzletet. (r Verkäufer)
5 | Die Tablette wurde von den Kranken nicht eingenommen. | A betegek nem vették be a tablettát. (einnehmen)
6 | Der Kranke wurde von dem Arzt geheilt. | A beteget meggyógyította az orvos.
7 | Die Uhr wurde von dem Vater repariert. | Az apa megjavította az órát.
8 | Meine Bücher wurden von Peter gekauft. | A könyveimet Peter vette meg. (kaufen)
9 | Die Tür wurde geklopft. | Az ajtót kopogtatták.
10 | Mein Buch wurde von meiner Freundin gelesen. | A barátnőm olvasta a könyvemet.
11 | Die Uhr wurde nicht von der Mutter gesehen. | Az anya nem nézte meg az órát.
12 | Der Tisch wurde von Józsi nicht gedeckt. | Józsi nem terítette meg az asztalt.
13 | Die Burg wurde durch den Wind zerstört. | A várat a szél lerombolta. (r Wind)
14 | Mein Zimmer wurde gemalt. | A szobámat kifestették.
\end{exmp}

\title{Cselekvő passiv mondatszerkezet/Passiv befejezett múlt időben - Passiv perfect}

GrammarItemIndex = 523523

\begin{desc}
alany + sein + többi mondatrész + ige + worden

Das Haus ist von Peter gabaut worden. | "Egy ház péter által lett építve."
\end{desc}

\begin{exmp}
\end{exmp}

\title{Cselekvő passiv mondatszerkezet/Passiv módbeli segédigével - Jelen időben}

GrammarItemIndex = 4232134

\begin{desc}
Képzése:
alany + segédige + többi mondatrész + ige Perfektje + werden

Pl.: Das Auto must von dem Mechaniker repariert werden. - A szerelőnek meg kell javítani az autót.
\end{desc}

%Maklári: 334.oldal
\begin{exmp}
1 | Der Apfel darf nicht gegessen werden. | Az almát nem szabad megenni.
2 | Das Zimmer muss ausgeräumt werden. | A szobát ki kell takarítani. (ausräumen)
3 | Das Fenster muss geöffnet werden. | Az ablakot ki kell nyitni. (öffnen)
4 | Parken muss hier verboten werden. | Parkolásnak tilosnak kell lenni itt. (s Parken, verbieten)
5 | Das Brief muss gesandt werden. | A levelet el kell küldeni. (senden)
6 | Die Bücher müssen ausgeliehen werden. | A könyveket ki kell kölcsönözni. (ausleihen)
7 | Die Großeltern müssen geholfen werden. | Az nagyszülőket segíteni kell.
8 | Dieses Motorrad muss repariert werden. | Ezt a motorbiciklit meg kell javítani. (s Motorrad)
9 | Das Brief muss für die Post gegeben werden. | A postának oda kell adni a levelet. (für, geben)
10 | Die Suppe kann schon gegessen werden. | A levest már meg lehet enni.
11 | Die Häuser können hier gemietet werden. | A házakat ki lehet itt bérelni. (mieten)
12 | Geschenk muss für die Oma gekauft werden. | A nagyinak ajándékot kell venni. (s Geschenk, kaufen)
13 | Der Kuchen darf von den Kindern nicht gegessen werden. | A gyerekek nem ehetik meg a süteményt. (r Kuchen)
14 | Sie muss besucht werden. | Meg kell látogatni őt (n).
15 | Das Auto darf von dir schnell gefahren werden. | Vezetheted gyorsan az autót. (fahren)
16 | Diese Angelegenheit muss gelöst werden. | Ezt az ügyet meg kell oldani. (e Angelegenheit)
17 | Das Brief muss geschrieben werden. | A levelet meg kell írni.
18 | Der Kranke kann operiert werden. | A beteget meg lehet operálni.
\end{exmp}

\title{Cselekvő passiv mondatszerkezet/Passiv módbeli segédigével elbeszélő múlt időben}

GrammarItemIndex = 523452

\begin{desc}
alany + segédige Präteriumban ragozva + többi mondatrész + ige befelyezett alakja + werden

Pl.: Das Auto musste von dem Mechaniker repariert werden. | A szerelőnek meg kelett javítania az autót.
\end{desc}

\begin{exmp}
\end{exmp}

\title{Cselekvő passiv mondatszerkezet/Passiv módbeli segédigével befelyezett múlt időben}

GrammarItemIndex = 52326

\begin{desc}
alany + haben + többi mondatrész + ige befelyezett alakja + werden + segédige Infinitivben

Pl.: Das Auto hat von dem Mechaniker repariert werden müssen.
\end{desc}

\begin{exmp}
\end{exmp}

\title{Cselekvő passiv mondatszerkezet/Módbeli segédigével KATI szórenddel/Jelen időben}

GrammarItemIndex = 523456

\begin{desc}
Pl.: Ich weiß, dass das Haus von Peter renoviert werden muss. - Tudom, hogy Péternek renoválnia kell a házat.
\end{desc}

\begin{exmp}
\end{exmp}

\title{Cselekvő passiv mondatszerkezet/Módbeli segédigével KATI szórenddel/Befejezett múlt időben}

GrammarItemIndex = 395275

\begin{desc}
\end{desc}

\begin{exmp}
\end{exmp}

\title{Cselekvő passiv mondatszerkezet/Módbeli segédigével KATI szórenddel/Elbeszélő múlt időben}

GrammarItemIndex = 759344

\begin{desc}
\end{desc}

\begin{exmp}
\end{exmp}

\title{Cselekvő passiv mondatszerkezet/Passiv jövő időben}

GrammarItemIndex = 541385

\begin{desc}
Képzése:
alany + werden ragozott alakja + többi mondatrész + ige befelyezett alakja + werden

Pl.: Die Vase wird in der Ence gestellt werden. - A váza a sarokba lesz állítva.
\end{desc}

%Maklári: 337. oldal
\begin{exmp}
1 | Die Milch wird in den Kühlschrank getan werden. | A tejet a hűtőszekrénybe fogják rakni. (tun)
2 | Die Suppe wird von dem Koch gekostet werden. | A szakács meg foja kóstolni a levest. (r Koch, kosten)
3 | Das Buch wird viel gesprochen werden. | A könyvről sokat fognak beszélni.
4 | Er wird auf seinen Geburtstag gratuliert werden. | Gratulálni fognak neki (f) a születésnapjára. (auf)
5 | Deine Briefe werden nicht geantwortet werden. | A leveleidre nem fognak válaszolni.
6 | Der Kranke wird von dem Arzt geprüft werden. | A beteget meg fogja vizsgálni az orvos. (r Kranke, prüfen)
7 | Der Hund wird auch operiert werden. | A kutyát is meg fogják operálni.
8 | Der Platz wird von den Gästen besetzt werden. | A helyet a vendégek fogják lefoglalni. (r Platz, besetzen)
9 | Interessante Vorstellung wird in dem Theater gespielt werden. | Érdekes előadást fognak játszani a színházban. (e Vorstellung)
10 | Das Brief wird nicht unterschrieben werden. | A levél nem lesz aláírva. (unterschreiben)
11 | Wird sie von Péter morgen eingeladen werden? | Meghívja őt (n) Péter holnap? (einladen)
12 | Wird es von dir bestellt werden? | Meg fogod ezt rendelni? (bestellen)
\end{exmp}

\title{Állapot passzív/Jelen időben}

GrammarItemIndex = 423428

\begin{desc}
Képzése:
alany + sein ragozva + többi mondatrész + ige Partizip Perfektben

Pl.: Die Tür ist geschlossen. - Az ajtó be van zárva.
\end{desc}

%Maklári: 342. oldal
\begin{exmp}
1 | Das Geschäft ist geschlossen. | Az üzlet zárva van. (schließen)
2 | Die Arbeit ist geendet. | A munka be van fejezve. (enden)
3 | Die Blumen sind gegossen. | A virágok meg vannak locsolva. (gießen)
4 | Die Fenster sind geöffnet. | Az ablakok ki vannak nyitva.
5 | Das Rauchen ist verboten. | A dohányzás meg van tiltva. (s Rauchen, verbieten)
6 | Die Uhr ist aufgezogen. | Az óra fel van húzva. (aufziehen)
7 | Das Zimmer ist eingerichtet. | A szoba be van rendezve. (einrichten) 
8 | Das Museum ist geöffnet. | A múzeum ki van nyitva.
9 | Die Schularbeit ist korrigiert. | A dolgozat ki van javítva. (e Schularbeit, korrigieren)
10 | Die Tür ist frisch gemalt. | Az ajtó frissen van festve. (malen)
11 | Die Karten sind verkauft. | A jegyek el vannak adva. (e Karte, verkaufen)
12 | Der Dieb ist verurteilt. | A tolvaj el van ítélve. (r Dieb, verurteilen)
13 | Die Teller sind gespült. | A tányérok el vannak mosogatva. (r Teller, spülen)
14 | Der Saal ist geschmückt. | A terem ki van díszítve. (r Saal, schmücken)
\end{exmp}

\title{Állapot passzív/Elbeszélő múlt időben}

GrammarItemIndex = 342423

\begin{desc}
Képzése:
alany + waren ragozott alakja + ige Partizip Perfekt alakja

Pl.: Die Tür war geschlossen. - Az ajtó be volt zárva.
\end{desc}

%Maklári: 343. oldal
\begin{exmp}
1 | Die Lampe war ausgeschaltet. | A lámpa ki volt kapcsolva. (ausschalten)
2 | Die Gäste waren abgeholt. | A vendégeket elhozták. (abholen)
3 | Der Teppich war geputzt. | A szőnyeg ki volt tisztítva. (r Teppich)
4 | Das Vorhaben war besprochen. | A terv meg volt beszélve. (s Vorhaben, besprechen)
5 | Das Bier war getrunken. | A sört megitták.
6 | Der Tisch war gedeckt. | Az asztat megterítették.
7 | Zwei Zimmer waren bestellt. | Két szobát rendeltek. (bestellen)
8 | Die Lektion war gemocht. | Megcsinálták a leckét. (e Lektion)
9 | Das Radio war eingeschaltet. | A rádió be volt kapcsolva. (einschlaten)
10 | Die Uhr war aufgezogen. | Az óra fel volt húzva. (aufziehen)
11 | Das Brief war unterschrieben. | A levél alá volt írva. (unterschreiben)
12 | Die Pakete waren gesandt. | A csomagok el voltak küldve. (s Paket, senden)
13 | Die Pferde waren geschlossen. | A lovak be voltak zárva. (schließen)
14 | Das Auto war beschädigt. | Az autó megsérült. (beschädigen)
15 | Dein Brief war eingeworfen. | A leveledet bedobták. (einwerfen)
16 | Die Maschine war ausgestellt. | A gépet kiállítonák. (e Maschine, ausstellen)
\end{exmp}

\title{Állapot passzív/Befelyezett múlt időben}

GrammarItemIndex = 45353

\begin{desc}
Képzése:
alany + sein ragozott alakja + ige Partizip Perfekt alakja + gewesen

Pl.: Die Tür ist geschlossen gewesen. - Az ajtó be volt zárva.
\end{desc}

%Maklári: 343. oldal
\begin{exmp}
1 | Die Zimmer sind schon besetzt gewesen. | A szobák már foglaltak voltak. (besetzen)
2 | Das Fenster ist geöffnet gewesen. | Az ablak ki volt nyita.
3 | Die Gulaschsuppe ist gekocht gewesen. | A gulyás meg volt főzve. (e Gulaschsuppe)
4 | Das Brief ist gesandt gewesen. | A levél el volt küldve. (senden)
5 | Das Kleid ist genäht gewesen. | A ruha meg volt varrva. (nähen)
6 | Ein Taxi ist gerufen gewesen. | Hívtak egy taxit.
7 | Ein Zimmer ist besetzt gewesen. | Egy szoba le volt foglalva. (besetzen)
8 | Die Rechnung ist gezahlt gewesen. | A számla ki volt fizetve. (e Rechung, zahlen)
9 | Die Wörter sind erklärt gewesen. | A szavakat elmagyarázták. (erklären)
10 | Die Zimmer sind vermietet gewesen. | A szobák ki voltak adval. (vermieten)
\end{exmp}

\title{Melléknévfokozás/Reguláris melléknevek fokozása}

GrammarItemIndex = 534452

\begin{desc}
* Alapfok: szótári alak
* Középfok: melléknév + er
  Pl.: schöner - szebb
* Felsőfok: melléknév + ste
  Pl.: die schönste Frau - a legszebb nő
\end{desc}

%Maklári: 202. oldal
\begin{exmp}
1 | dicker | kövérebb
2 | dickste | legkövérebb
3 | magerer | soványabb
4 | magerste | legsoványabb
5 | niedriger | alacsonyabb
6 | niedrigste | legalacsonyabb
7 | billiger | olcsóbb
8 | billigste | legolcsóbb
9 | breiter | szélesebb
10 | breitste | legszélesebb
11 | geschickter | ügyesebb
12 | geschicktste | legügyesebb
13 | schneller | gyorsabb
14 | schnellste | leggyorsabb
15 | langsamer | lassabb
16 | langsamste | leglassabb
17 | teuerer | drágább
18 | teuerste | legdrágább
\end{exmp}

\title{Melléknévfokozás/Umlautos melléknevek}

GrammarItemIndex = 423423

\begin{desc}
Az egyszótagos a, o, u tövű melléknevek umlautot kapnak.

ALAPFOK | KÖZÉPFOK  | FELSŐFOK
--------+-----------+-------------
alt     | älter     | älteste
kalt    | kälter    | kälteste
jung    | jünger    | jüngste
stark   | stärker   | stärkste
lang    | länger    | längste
warm    | wärmer    | wärmste
kurz    | kürzer    | kürzeste
hart    | härter    | härtste
schwach | schwächer | schwächste
dumm    | dümmer    | dümmste
klug    | klüger    | klügste
arm     | ärmer     | ärmste
scharf  | schärfer  | schärfste
\end{desc}

\begin{exmp}
1 | älter | öregebb
2 | älteste | legöregebb
3 | kälter | hidegebb
4 | kälteste | leghidegebb
5 | jünger | fiatalabb
6 | jüngste | legfiatalabb
7 | stärker | erősebb
8 | stärkste | legerősebb
9 | länger | hosszabb
10 | längste | leghosszabb
11 | wärmer | melegebb
12 | wärmste | legmelegebb
13 | kürzer | rövidebb
14 | kürzeste | legrövidebb
15 | härter | keményebb
16 | härtste | legkeményebb
17 | schwächer | gyengébb
18 | schwächste | leggyengébb
19 | dümmer | butább
20 | dümmste | legbutább
21 | klüger | okosabb
22 | klügste | legokosabb
23 | ärmer | szegényebb
24 | ärmste | legszegényebb
25 | schärfer | élesebb
26 | schärfste | legélesebb
\end{exmp}

\title{Melléknévfokozás/Rendhagyó melléknevek}

GrammarItemIndex = 132141

\begin{desc}
ALAPFOK         | KÖZÉPFOK | FELSŐFOK
----------------+----------+----------
gut (jó)        | besser   | beste
sok (sok)       | mehr     | meiste
gern (szívesen) | lieber   | liebste
nah (közeli)    | näher    | nächste
hoch (magas)    | höher    | höchste
groß (nagy)     | größer   | größte
\end{desc}

%Maklári: 202. oldal
\begin{exmp}
1 | besser | jobb, jobban
2 | beste | legjobb
3 | mehr | több
4 | meiste | legtöbb
5 | lieber | inkább, szívesebben
6 | liebste | legkedvesebb
7 | näher | közelebb
8 | nächste | legközelebbi
9 | höher | magasabb
10 | höchste | legmagasabb
11 | größer | nagyobb
12 | größte | legnagyobb
\end{exmp}

\title{Melléknévfokozás/Rendhagyó melléknevek/Melléknévfokozás - Példák}

GrammarItemIndex = 423654

\begin{desc}
A fokozott mellékneveket egyeztetni is kell a főnévvel.
\end{desc}

%Maklári: 203. oldal
\begin{exmp}
1 | der älteste Mensch | a legidősebb ember (r Mensch)
2 | der schnellste Wagen | a leggyorsabb kocsi
3 | das langsamste Auto | a leglassab autó
4 | das höchste Haus | a legmagasabb ház
5 | das hübscheste Mädchen | a legcsinosabb lány
6 | den klügsten Schüler | a legokosabb tanulót
7 | die langweiligste Stunde | a legunalmasabb óra
8 | die kleinste Lampe | a legkisseb lámpa
9 | das meiste Wasser | a legtöbb víz
10 | die nächste Haltestelle | a legközelebbi megálló (e Haltestelle)
11 | mein älteste Bruder | a legidősebb bátyám
12 | meine jüngste Schwester | a legfiatalabb nővérem
13 | sein kälteste Zimmer | a leghidegebb szobája (s)
14 | unser beste Bier | a legjobb sörünk
15 | den wärmsten Wein | a legmelegebb bort
16 | der schnellste Wagen der Welt | a világ leggyorsabb kocsija
17 | die feinste Suppe der Welt | a világ legfinommabb levese
18 | der schnellste Schwimmer der Welt | a világ leggyorsabb úszója
19 | der beste Schüler der Klasse | az osztály legjobb tanulója
20 | die teuerste Kleid des Geschäfts | az üzlet legdrágább ruhája
21 | das hübscheste Mädchen der Klasse | az osztály legcsinosabb lányát
22 | die schönste Blume des Gartens | a kert legszebb virágát
23 | das schönste Möbelstück der Wohnung | a lakás legszebb bútorát (s Möbelstück)
\end{exmp}

\title{Személytelen szerkezetek/Személytelen igék}

GrammarItemIndex = 432123

\begin{desc}
Ezen igék esetében az alanyt az es szóval fejezzük ki. Ilyenek az időjárást kifejező igék.
\end{desc}

%Maklári: 187. oldal
\begin{exmp}
1 | Es regnet. | Esik. (regnen)
2 | Es scheint. | Havazik. (scheinen)
3 | Es blitzt. | Villámlik.
4 | Es hagelt. | Jégeső esik. (hageln)
5 | Es donnert. | Dörög.
6 | Es dunkelt. | Sötétedik.
7 | Es rieselt. | Szemerkél. (rieseln)
8 | Es klopft. | Kopognak.
9 | Es klingelt. | Csöngetnek.
10 | Es taut. | Olvad.
11 | Es ist warm. | Meleg van. (warm)
12 | Es lärmt immer. | Mindig lármáznak. (lärmen)
13 | Es ist kalt. | Hideg van.
14 | Wer klopft? | Ki kopog?
15 | Es klopft, hörst du nicht? | Kopognak, nem hallod?
16 | Wer klingelt? | Ki csönget?
17 | Es klingelt, öffnest du die Tür? | Csöngetnek, kinyitod az ajtót?
\end{exmp}

\title{Személytelen szerkezetek/Az es gibt szerkezet}

GrammarItemIndex = 423424312

\begin{desc}
Az es gibt (= van) szerkezetet akkor használjuk, ha valaminek a létezését vagy a nem létezését hangsúlyozzuk.

Pl.: Es gibt viele Sehenswürdigkeiten in Salzburg. - Sok látnivaló van Salzburgban.
\end{desc}

\begin{exmp}
1 | Es gibt Wunder. | Léteznek csodák.
2 | Gibt es hier eine Toilette? | Van itt egy toalett?
3 | Wo gibt es in der Stadt Restaurant? | Hol van a városban étterem?
4 | Gibt es Wurst? | Van kolbász?
5 | Gibt es hier Giraffen? | Vannak itt zsiráfok?
6 | Es gibt viele Möglichkeiten. | Sok lehetőség van.
7 | Gibt es U-Bahn in Zürich? | Van metró Zürichben?
8 | Es gibt viele Regenschirme in England. | Sok esernyő van Angliában. (r Regenschirm - esernyő)
9 | Es gibt 50 Hotel in der Stadt. | A városban 50 szálloda van.
10 | Es gibt in der Bibliothek viele Bücher. | A könyvtárban sok könyv van.
11 | Es gibt 30 Museen in der Stadt. | 30 múzeum van a városban.
12 | Es gibt hier nur Lebensmittel. | Itt csak élelmiszer van.
13 | Gibt es hier ein Platz? | Van itt egy hely?
14 | Gibt es ein Hotel irgendwo? | Van valahol egy hotel?
15 | Es gibt bei uns wenig Konditorei. | Nálunk kevés cukrászda van.
16 | Gibt es hier ein gute Bier? | Van itt egy jó sör?
17 | Es gibt nicht Personalausweis in Amerika. | Nincs személyiigazolvány Amerikában.
18 | Es gibt bei der Ecke eine Kneipe. | A sarkon (saroknál) van egy kocsma.
\end{exmp}

\title{Célhatározói mellékmondatok/Az um zu + Infinitiv szerkezet}

GrammarItemIndex = 423412

\begin{desc}
Célhatározói mellékmondat képzésére használatos akkor, ha a fő és a mellékmondat alanya megegyezik.

Képzése:
főmondat + um + többi mondatrész + zu + ige Infinitiv alakja

Pl.: Sokat dolgozik, azért, hogy több pénzt keressen. - Er arbeitet viel, um mehr Geld zu verdienen.
\end{desc}

%Maklári: 320. oldal, 321. oldal
\begin{exmp}
1 | Ich reise nach Deutschland, um die Sprache zu üben. | Németországba utazom, hogy gyakoroljam a nyelvet. (üben)
2 | Ich ziehe einen Mantel an, um schön auf der Party zu sein. | Felveszek egy kabátot, azért, hogy szép legyek a partin. (anziehen)
3 | Er schläft viel, um schnell zu heilen. | Sokat alszik (f), hogy gyorsan meggyógyuljon. (heilen)
4 | Ich hebe mein Geld aus der Bank ab, um ein Auto zu kaufen. | Felveszem a pénzem a bankból, hogy vegyek egy autót. (abheben)
5 | Ich stehe früh auf, um den Bus zu erreichen. | Korán kelek, hogy elérjem a buszt. (erreichen)
6 | Wir laufen schnell, um die Strassenbahn zu erreichen. | Gyorsan futunk, hogy elérjük a villamost. (erreichen)
7 | Ich reise nach Österreich, um mich zu erholen. | Ausztiába utazom, hogy kipihenjem magamat. (sich erholen)
8 | Sie kommt zu mir, um mir zu helfen. | Eljön (n) hozzám, hogy segítsen nekem.
9 | Ich haste, um den Zug zu erreichen. | Sietek, hogy elérjem a vonatot. (hasten, erreichen)
10 | Die Großmutter nimmt ihre Börse mit, um Kartoffel zu kaufen. | A nagymama magával viszi az pénztárcáját, hogy krumplit vásárolhasson. (mitnehmen)
11 | Ich gebe dir eine Ohrfeige, um besser mich zu füllen. | Adok neked egy pofont, hogy jobban étezzem magamat. (sich füllen)
\end{exmp}

\title{Célhatározói mellékmondatok/Damit + KATI}

GrammarItemIndex = 862153

\begin{desc}
Célhatározói mellékmondat képzésére használatos akkor, ha a fő és a mellékmondat alanya nem egyezik meg.

Képzése:
főmondat + damit + alany + többi mondatrész + ige ragozva

Pl.: Ich gehe zur Oma, damit sie mir etwas kocht. - Elmegyek a nagyihoz, hogy főzzön nekem valamit.
\end{desc}

%Maklári: 321. oldal
\begin{exmp}
1 | Wir helfen der Oma, damit sie gesund bleibt. | Segítünk a nagyinak, hogy egészséges maradjon.
2 | Ich gebe ihr Essen, damit sie hungrig nicht ist. | Adok neki (n) ételt, hogy ne legyen éhes. (s Essen)
3 | Ich ende die Arbeit, damit der Nachbar nicht erwacht. | Befejezem a munkát, hogy a szomszéd fel ne ébredjen. (erwachen)
4 | Ich gebe dir eine Uhr, damit du dich nicht verspätest. | Adok neked egy órát, hogy ne késs el. (sich verspäten)
5 | Ich spüle, damit Mutter sich freut. | Elmosogatok, hogy örüljön anya. (sich freuen)
6 | Der Arzt operiert meine Katze, damit sie lange lebt. | Az orvos megoperálja a macskámat, hogy sokáig éljen. (lange)
7 | Ich lege mein Geld in die Bank ein, damit meine Freundin nicht verjubelt. | Beteszem a pénzemet a bankba, nehogy a barátnőm elköltse. (einlegen, verjubeln)
8 | Ich ziehe den Vorhang weg, damit es Wärme in dem Zimmer nicht ist. | Elhúzom a függönyt, nehogy meleg legyen a szóbában. (wegziehen, e Wärme)
9 | Ich helfe dir, damit du froh bist. | Segítek neked, hogy boldog legyél. (froh - boldog)
10 | Ich lade in deine Tasche ein, damit du dich nicht verspätest. | Bepakolok a táskádba, nehogy elkéssél. (einladen, sich verspäten)
11 | Ich gebe dir meinen Kamm, damit du dich kämmst. | Odaadom a fésűmet neked, hogy fésülködj meg. (r Kamm, sich kämmen)
12 | Ich helfe jetzt ihr, damit sie später mir hilft. | Most én segítek neki (n), hogy később ő segítsen nekem. (später)
13 | Sie gibt mir ihre Telefonnumber, damit ich morgen anrufe. | Odaadja (n) a telefonszámát nekem, hogy holnap felhívjam.
\end{exmp}

\title{Célhatározói mellékmondatok/Módbeli segédigék célhatározói mellékmondatokban}

GrammarItemIndex = 626423

\begin{desc}
Um + zu + Infinitiv esetében:
főmondat + um + többi modatrész + ige Infinitivben + zu + módbeli segédige Infinitivben

Pl.: Ich nehme eine Tablette ein, um besser schlafen zu können. - Beveszek egy tablettát, hogy jobban tudjak aludni.

Damit + KATI esetében:
főmondat + damit + alany + többi mondatrész + ige Infinitivben + módbeli segédige ragozva

Pl.: Ich gebe dir eine Tablette, damit du besser schlafen kannst. - Adok neked egy tablettát, hogy jobban tudjál aludni.
\end{desc}

%Maklári: 323. oldal
\begin{exmp}
1 | Mein Bruder hilft mir, damit ich die Hausaufgabe schreiben kann. | A tesvérem (f) segít nekem, hogy meg tudjam írni a házit.
2 | Er schläft viel, um morgen gut fahren zu können. | Sokat alszik (f), hogy holnap jól tudjon vezetni.
3 | Sie räumt das Zimmer auf, um in das Geschäft nicht gehen zu müssen. | Kitakarítja (n) a szobát, hogy ne kelljen boltba mennie. (aufräumen)
4 | Ich kaufe mit ihr alles ein, damit ihr nicht müsst. | Bevásárolok vele (n) mindent, hogy ne nektek kelljen.
5 | Ich repariere seinen Wagen, damit er das Rennen gewinnen kann. | Megjavítom a kocsiját (f), hogy meg tudja nyerni a versenyt. (s Rennen, gewinnen)
6 | Ich schließe die Tür ein, damit der Hund nicht ausgehen kann. | Becsukom az ajtót, hogy ne tudjon kimenni a kutya. (einschließen, ausgehen)
7 | Ich gehe mit Taxi auf den Flughafen, um das Flugzeug erreichen zu können. | Taxival megyek a repülőtérre, hogy el tudjam érni a repülőgépet. (erreichen)
8 | Wir machen hocher Zaun, damit die Nachbarn nicht überklettern können. | Magas keítést csinálunk, hogy ne tudjanak átmászni szomszédok. (r Zaun, überklettern)
9 | Er sitzt ganzer Tag auf der Toilette, um nicht arbeiten zu müssen. | Egész nap a VC-n ül (f), hogy ne kelljen dolgoznia. (sitzen, ganz)
10 | Er isst immer wenig, um schneller laufen zu können. | Mindig keveset eszik (f), hogy gyorsabban tudjon futni.
\end{exmp}

\title{Függő beszéd}

GrammarItemIndex = 421412

\begin{desc}
\end{desc}

\begin{exmp}
\end{exmp}

\title{Páros kötőszavak/Sowohl - als auch}

GrammarItemIndex = 532453

\begin{desc}
Jelentése: is-is.
A mellékmondatban fordított szórendet használunk.

Pl.: Ich gehe heute sowohl zu dir, als auch besuche ich meine Oma. - Ma hozzád is megyek és meg is látogatom a nagyit. 
\end{desc}

%Maklári: 368. oldal
\begin{exmp}
1 | Sie gießt sowohl die Blumen, als auch macht sie Ordnung in meinem Zimmer. | A virágokat is megöntözi (n) és a szobámban is rendet rak. (gießen)
2 | Ich besuche morgen sowohl Mutter, als auch reise ich auf Umgebung | Holnap meglátogatom anyát is és vidékre is utazom. (reisen, e Umgebung)
3 | Ich gehe sowohl in die Stadt, als auch sehe ich mir das Museum an. | A városba is megyek és a múzeumot is megnézem. (sich ansehen)
4 | Ich gebe dir sowohl Mittagessen, als auch wasche ich deine Haare. | Adok neked ebédet is és meg is mosom a hajadat.
5 | Sie gibt mir sowohl Schokolade, als auch nimmt sie mich in den Zoo. | Ad (n) nekem csokit is és el is visz az állatkertbe. (nehmen, r Zoo)
6 | Ich habe sowohl eingekauft, als auch habe ich Ordnung in meinem Zimmer gemacht. | Be is vásároltam és rendet is tettem a szobámban. (einkaufen (Perfekt), Ordnung machen (Perfekt))
7 | Ich gieße sowohl die Blume, als auch gebe ich für den Hund Essen. | A virágot is megöntözöm és a kutyának is adok enni. (gießen, Essen geben)
8 | Ich habe sowohl den Wagen repariert, als auch habe ich deine Freundin gebracht. | A kocsit is megjavítottam és a barátnődet is elhoztam. (reparieren (Perfekt), bringen (Perfekt))
9 | Er hat sowohl meine Suppe gegessen, als auch hat er die Vase zerstört. | A levesemet is megette (f) és a vázát is összetörte. (esse (Perfekt), zerstören (Perfekt))
10 | Sie hat sowohl gespült, als auch hat sie eingekauft. | El is mosogatott (n) és be is vásárolt. (spülen (Perfekt))
11 | Wir spielen heute sowohl Tennis, als auch sehen wir uns die Bilder des Museums an. | Ma teniszezünk is és meg is nézük a múzeum képeit. (Tennis spielen, sich ansehen)
12 | Wir schwimmen heute sowohl in dem Plattensee, als auch fahren wir Fahrrad. | Ma úszunk a Balatonban és biciklizünk is. (r Plattensee, Fahrrad fahren)
\end{exmp}

\title{Páros kötőszavak/Weder - noch}

GrammarItemIndex = 123531

\begin{desc}
Jelentése: sem - sem.
A mellékmondatban fordított szórendet használunk.

Pl: * Ich trinke weder Alkohol, noch esse ich Fleisch.
* Ich bin weder schön noch intelligent.
* Sie liebt weder mich noch meine Freundin noch Géza.
\end{desc}

%Maklári: 370. oldal
\begin{exmp}
1 | Er sieht weder fern, noch liest er Buch. | Se tévét nem néz (f), se könyvet nem olvas. (fernsehen)
2 | Er räumt weder in seinem Zimmer, noch spült er. | Se nem takarít (f) a szobájában, se nem mosogat. (spülen)
3 | Sie hilft weder mir, noch ist sie höflich zu mir. |  Se nem segít (n) nekem, se nem udvarias velem. (höflich sein zu + D)
4 | Ich bitte weder um Rat von dir, noch helfe ich dir. | Sem tanácsot nem kérek tőled, sem nem segítek neked. (um Rat bitten)
5 | Deine Freundin ist weder schlank, noch kann sie Fenster putzen. | A barátnőd sem karcsú, sem nem tud ablakot pucolni. (schlank, putzen)
6 | Du lernst weder, noch leerst du den Mülleinmer aus. | Se nem tanulsz, se nem viszed ki a szemeted. (den Mülleinmer ausleeren)
7 | Diese Tier fliegt weder, noch schwimmt er. | Ez az állat se nem repül, se nem úszik. (e Tier)
8 | Er trinkt weder Alkohol, noch tanzt er mit den Mädchen. | Alkoholt sem iszik (f), a lányokkal sem táncol.
9 | Józsi treibt weder Sport, nocht lernt er. | Józsi se nem sportol se nem tanul. (Sport treiben)
10 | Dieser Obdachlose bettelt weder, noch arbeitet er. | Ez a hajléktalan se nem koldul se nem dolgozik. (r Obdachlose, betteln)
11 | Ich schlafe weder auf diese Stunde, noch lerne ich. | Ezen az órán se nem alszok, se nem tanulok.
12 | Es gibt hier weder Huhn, noch jagt Jäger. | Itt tyúk sincsen és vadász sem vadászik. (es gibt, s Huhn, jagen)
\end{exmp}

\title{Páros kötőszavak/Entweder - oder}

GrammarItemIndex = 543532

\begin{desc}
Jelentése vagy-vagy.
Mindkét tagmondatban egyenes szórendet használunk.

Pl: * Entweder fahren wir heute zur Oma, oder wir besuchen meine Tochter.
* Entweder fahren wir mit ihnen oder mit deinen Eltern.
\end{desc}

\begin{exmp}
\end{exmp}

\title{Páros kötőszavak/nicht nur | sondern auch}

GrammarItemIndex = 543423

\begin{desc}
Jelentése: nem csak - hanem ... is.
\end{desc}

\begin{exmp}
\end{exmp}

\title{Melléknév ragozás/Erős, avagy névelőpótló ragozás}

GrammarItemIndex = 432425

\begin{desc}
Ha nem tartozik névelő a főnévhez, akkor a melléknév tartalmaz minden információt a főnévről (neme, száma esete).

Erős ragozás használatos, ha:
* a főnévhez nem tartozik névelő (Gute nacht!),
* tőszámnév tartozik a főnévhez (drei gute Bücher),
* határozatlan számnév tartozik a főnévhez (viele gute Freunde).

  | HÍMNEM       | NŐNEM        | SEMLEGES NEM    | TÖBBESSZÁM
--+--------------+--------------+-----------------+----------------
N | guter Wein   | warme Milch  | frisches Brot   | gute Bücher
A | guten Wein   | warme Milch  | frisches Brot   | gure Bücher
D | gutem Wein   | warmer Milch | frischem Brot   | guten Büchern
G | guten Weines | warmer Milch | frischen Brotes | guter Bücher
\end{desc}

%Maklári: 190. oldal
\begin{exmp}
1 | guter Mann | jó ember
2 | kleines Auto | kicsi autó
3 | gute Fahrt | jó utazást (e Fahrt)
4 | schönes Bild | szép kép
5 | dicke Frau | kövér nő (dick)
6 | magerer Hund | sovány kutya
7 | altes Haus | öreg ház
8 | feine Suppe | finom leves (fein)
9 | teueres Kleid | drága ruha
10 | billige Hose | olcsó nadrágot
11 | rote Rosen | vörös rózsák (e Rose-n)
12 | blauer Stuhl | kék szék
13 | faule Kinder | lusta gyerekek (faul)
14 | dummer Politiker | buta politikus (dumm)
15 | schneller Wagen | gyors kocsi
16 | neue Zeitung | új újság
17 | laute Musik | hangos zene
18 | großen Tisch | nagy asztalt
19 | kleinen Kindern | kicsi gyerekeknek
20 | schwarzem Hund | fekete kutyának
21 | zwei schöne Abende | két szép estét
22 | fünf lichte Hosen | öt világos nadrágot (licht)
23 | viele gute Kleider | sok jó ruha
24 | einige kleine Kuli | néhány kis tollat
25 | drei interessante Tage | három érdekes napot
26 | zwei dicke Katzen | két kövér macskát (dick)
27 | wenige grünen Hasen | kevés zöld nyulat (r Hase)
28 | einige gute Zimmer | néhány jó szoba
29 | viele langweilige Abende | sok unalmas estét
30 | viele Männer | sok ember
31 | einige Bücher | néhány könyv
32 | großen Erfolg | nagy sikert
33 | lauten Hunden | hangos kutyáknak
34 | mehreren Kindern | több gyereknek
35 | liebe Kati | kedves Kati
36 | neues Jahr | új év
37 | guten Appetit | jó étvágyat (r Appetit)
38 | gute Idee | jó ötlet
39 | guten Abend | jó estét
40 | lieber István | kedves István
41 | gute Nacht | jó éjszakát
42 | viele dicke Frau | sok kövér nő
43 | einige dicke Männer | néhány kövér férfi 
44 | lautem, kleinem, nervösem Hund | hangos, kicsi, ideges kutyának (nervös)
45 | billige, gute Ware | olcsó, jó áruk
46 | die Seiten guter Bücher | jó könyvek oldalali
47 | die Fenster hoher, großer Häuser | magas, nagy házak ablakai
48 | schnellen, mageren Hunden | gyors, sovány kutyáknak
49 | viele interessante Geschichten | sok érdekes történet
50 | niedrige Kinder hoher Eltern | magas szülők alacsony gyermekei
\end{exmp}

\title{Melléknév ragozás/Gyenge, avagy határozott névelős ragozás}

GrammarItemIndex = 432421

\begin{desc}
Gyenge melléknév ragozást akkot használunk, ha a főnévhez tartozik névelő. Ez a vévelő minden információt tartalmaz a főnévről.

  | HÍMNEM           | NŐNEM            | SEMLEGES NEM     | TÖBBESSZÁM
--+------------------+------------------+------------------+-------------------
N | der gutE Vater   | die gutE Mutter  | das gutE Kind    | die gutEN Kinder
A | den gutEN Vater  | die gutE Mutter  | das gutE Kind    | die gutEN Kinder
D | dem gutEN Vater  | der gutEN Mutter | dem gutEN Kind   | den gutEN Kindern
G | des gutEN Vaters | der gutEN Mutter | des gutEN Kindes | der gutEN Kinder

Gyenge ragozást az alábbi esetekben használunk:
* a főnév előtt határozott névelő áll (der gute Wein);
* a főnév előtt mutató névmás áll (diesen schönen Mantel):
  - dieser/e/es = ez;
  - jener/e/es = az;
* az alábbi szavak után:
  - aller/e/es = minden, összes;
  - jeder/e/es = mindegyik, minden;
  - solcher/e/es = olyan;
  - welcher/e/es = melyik.
\end{desc}

%Maklári: 193. oldal
\begin{exmp}
1 | den dicken Mann | a kövér férfit
2 | das heiße Wasser | a forró vizet
3 | den guten Kindern | a jó gyerekeknek
4 | die blaue Tasche | a kék táska
5 | der kleine Wagen | a kicsi kocsi
6 | das braune Bett | a barna ágyat
7 | der jungen Frau | a fiatal nőnek
8 | den kleinen Leuten | a kicsi embereknek
9 | das hoche Haus | a magas házat
10 | dem niedrigen Mann | az alacsony férfinak
11 | das dünne Mädchen | a vékony lány (dünn)
12 | den mageren Hund | a sovány kutyát
13 | den blauen Kuli | a kék tollat
14 | den nervösen Leuten | az ideges embereknek
15 | die dicke Frau | a kövér nőt
16 | das Kind des guten Vaters | a jó apa gyereke
17 | das Fenster des hachen Hauses | a magas ház ablaka
18 | das Rad des roten Fahrrads | a piros kerékpár kereke
19 | die Zähne der guten Kinder | a jó gyerekek fogai
20 | die Beine der langsamen Hunde | a lassú kutyák lábai (das Bein)
21 | der Arm des dünnen Kindes | a vékony gyerek karja (dünn)
22 | der Bauch der dicken Leuten | a kövér emberek hasa
23 | das Ende der interessanten Geschichte | az érdekes történet vége
24 | die Lampe das schönen Wagen | a szép kocsi lámpája
25 | diese gelbe Jacke | ezt a sárga dzsekit (e Jacke)
26 | jene gute Musik | azt a jó zenét
27 | diesem hochen Mädchen | ennek a magas lánynak
28 | jenen mageren Männern | azoknak a sovány férfiaknak
29 | jenen nervösen Hunden | azoknak az ideges kutyáknak
30 | jenen traurigen Mäusen | azoknak a szomorú egereknek
31 | diesen neuen Schrank | ezt az új szekrényt
32 | jenen starken Mann | azt az erős férfit
33 | jener interessante Mensch | az az érdekes ember (r Mensch)
34 | allen interresanten Leuten | minden érdekes embert (e Leute)
35 | solches große Haus | olyan nagy házat
36 | jeden guten Kindern | minden jó gyereknek
37 | solche blaue Hose | olyan kék nadrágot
38 | allen starken Mann | minden erős férfit
39 | jedes kleine Kind | minden kicsi gyereket
40 | den netten, jungen Mädchen | a kedves, fiatal lányoknak
41 | dem lauten, hochen Lehrer | a hangos, magas tanárnak
42 | den ruhigen, stillen Hund | a nyugodt, csöndes kutyát (ruhig, still)
\end{exmp}

\title{Melléknév ragozás/Vegyes ragozás, avagy határozatlan névelős ragozás}

GrammarItemIndex = 431225

\begin{desc}
\end{desc}

\begin{exmp}
\end{exmp}

\title{Az általános alany/A Man}

GrammarItemIndex = 5864

\begin{desc}
Man után az állítmány csak E/3 személyben állhat.

Pl.: * Man kann hier gut essen. - Itt jót lehet enni.
* Man isst nicht viel in der Oper. - Az operában nem eszünk sokat. 
\end{desc}

%Maklári: 116. oldal
\begin{exmp}
1 | Man spricht hier Englisch. | Itt beszélnek angolul.
2 | Man raucht hier nicht. | Itt nem dohányoznak.
3 | Man kann hier billig einkaufen. | Itt olcsón lehet bevásárolni.
4 | Man darf in der Restaurant singen. | Az étteremben szabad énekelni.
5 | Man arbeitet in Deutschland viel. | Németországban sokat dolgoznak.
6 | Man lebt in diesem Land gut. | Ebben az országban jól élnek.
7 | Man steht bei uns um 8 Uhr auf. | Nálunk 8-kor kelnek.
8 | Man darf hier nicht rauchen. | Itt nem szabad dohányozni.
9 | Wie sagt Man es in Deutsch? | Hogy mondják ezt németül?
10 | Man trinkt nicht vor Autofahren. | Autó vezetés előtt nem iszunk. (s Autofahren)
11 | Man isst vor Mittagessen Schokolade nicht. | Ebéd előtt nem eszünk csokit.
12 | Man spricht bei Essen nicht. | Étkezés közben nem beszélünk. (bei Essen)
13 | Man darf hier nicht Fußball spielen. | Itt nem szabad focizni.
14 | Man isst in Ungarn viel Fleisch. | Magyarországon sok húst esznek. (s Fleisch)
15 | Man kann ohne Wasser nicht leben. | Víz nélkül nem lehet élni.
16 | Man darf dorthin nicht treten. | Oda nem szabad lépni. (dorthin)
17 | Was kann Man hier tun? | Mit lehet itt tenni?
18 | Wie sagt Man es in Englisch? | Hogy mondják ezt angolul?
\end{exmp}

\title{A számok/Tőszámnevek/Tőszámnevek 30-ig}

GrammarItemIndex = 541994

\begin{desc}
\end{desc}

\begin{exmp}
1 | null | 0
2 | eins | 1
3 | zwei | 2
4 | drei | 3
5 | vier | 4
6 | fünf | 5
7 | sechs | 6
8 | sieben | 7
9 | acht | 8
10 | neun | 9
11 | zehn | 10
12 | elf | 11
13 | zwölf | 12
14 | dreizehn | 13
15 | vierzehn | 14
16 | fünfzehn | 15
17 | sechzehn | 16
18 | siebzehn | 17
19 | achtzehn | 18
20 | neunzehn | 19
21 | zwanzig | 20
22 | einundzwanzig | 21
23 | zweiundzwanzig | 22
24 | dreiundzwanzig | 23
25 | vierundzwanzig | 24
26 | fünfundzwanzig | 25
27 | sechsundzwanzig | 26
28 | siebenundzwanzig | 27
29 | achtundzwanzig | 28
30 | neunundzwanzig | 29
31 | dreißig | 30
\end{exmp}

\title{A számok/Tőszámnevek/Tőszámnevek 30 után}

GrammarItemIndex = 128493

\begin{desc}
\end{desc}

\begin{exmp}
\end{exmp}

\title{A számok/Sorszámnevek/Általában}

GrammarItemIndex = 948324

\begin{desc}
Képzése: * erste -> első !!
* zweite -> második
* dritte -> harmadik !!
* tőszámnév + te -> 4-től 19-ig
* tőszámnév + ste -> 20-tól

A sorszámneveket melléknevekhez hasonlóan ragozzuk.

Pl.: * der Fünfte Wagen - az ötödik kocsi
* das zwanzigste Fenster - a huszadik ablal
\end{desc}

%Maklári: 96. oldal
\begin{exmp}
1 | das dritte Fenster | a harmadik ablak
2 | das vierte Kind | a negyedik gyerek
3 | der zwanzigste Stuhl | a huszadik szék
4 | der fünfte Lehrer | az ötödik tanár
5 | der zweitausendste Abnehmer | a kétszázadik vásárló (r Abnehmer)
6 | sein dritte Zahn | a harmadik foga (f)
7 | das achte Glas | a nyolcadik pohár 
8 | die zwanzigste Frau | a huszadik feleség
9 | die vierte Wohnung | a negyedik lakás
10 | die einhundertste Fliege | a századik légy (e Fliege)
11 | die einhundertundvierundzwanzigste Straße | a százhuszonnegyedik utca
12 | der zweiundvierzigste Stock | a negyvenkettedik emelet
13 | der dreißigste Wagen | a harmincadik kocsi
14 | das einhundertundfünfzigste Buch | a százötvenedik könyv
15 | die erste Aufgabe | az első feladat
16 | der fünfunddreißigste Chef | a harmincötödik főnök
17 | der zweihundertundzwölfte Stock | a kétszáztizenketteddik emelet
18 | mein erste Buch | az első könyvem
\end{exmp}

\title{A számok/Sorszámnevek/Sorszámnevek prepozíciókkal}

GrammarItemIndex = 533298

\begin{desc}
Képzése: * erste -> első !!
* zweite -> második
* dritte -> harmadik !!
* tőszámnév + te -> 4-től 19-ig
* tőszámnév + ste -> 20-tól

A sorszámneveket melléknevekhez hasonlóan ragozzuk.

Pl.: mit dem fünften Auto - az ötödik autóval
\end{desc}

%Maklári: 96. oldal
\begin{exmp}
1 | zu dem vierten Haus | a negyedik házhoz
2 | auf den zweiten Stock | a második emeletre (r Stock)
3 | ihr zehnte Chef | a tizedik főnöke
4 | in dem dreiundzwanzigste Buch | a huszonharmadik könyvben
5 | von seinem vierten Frau | a negyedik feleségétől
6 | in den zweiten Bus | a második buszba
7 | bei der achten Firma | a nyolcadik cégnél
8 | zu seiner dritten Freundin | a harmadik barátnőjéhez (f)
9 | in die vierten Haltestelle | a negyedik megállóba (e Haltestelle)
10 | von dem einhundersten Abnehmer | a századik vásárlótól (r Abnehmer)
11 | für ihren fünften Mann | az ötödik férjéért (r Mann)
12 | in die zehnten Reihe | a tizedik sorba (e Reihe)
13 | aus dem fünften Kino | az ötödik moziból
14 | in das sechsten Theater | a hatodik színházba
15 | in die zehnten Wohnung | a tizedik lakásba
16 | von dem vierundachtzigsten Stock | a nyolcvannegyedik emeletről (r Stock)
17 | auf den vierten Tisch | a negyedik asztalra
18 | auf den dreiundzwanzigsten Stock | a huszonharmadik emeleten (r Stock)
\end{exmp}

\title{A zu + infinitiv szerkezet/A zu + infinitiv általában}

GrammarItemIndex = 354964

\begin{desc}
Csak akkor használhatunk zu + infinitives szerkezetet, ha a főmondat és a mellékmondat alanya megegyezik.

Ha a mellékmondetban csak zu és ige áll, akkor nem kell vesszőt tenni.

Pl.: Ich hoffe, den Zug noch zu erreichen. - Remélem még elérem a vonatot.
\end{desc}

%Maklári: 304. oldal, 305. oldal
\begin{exmp}
1 | Sie glaubt, klug zu sein. | Azt hiszi (n), hogy okos.
2 | Ich hoffe, eine gute Note zu bekommen. | Remélem, kapok egy jó jegyet. (e Note)
3 | Sie beginnt jetzt zu arbeiten. | Elkezdett most (n) dolgozni. (beginnen)
4 | Er behauptet, alle meine Freunde zu kennen. | Azt állítja (f), hogy ismeri minden barátomat. (behaupten)
5 | Sie ist müde, nicht auf zu stehen. | Olyan fáradt (n), hogy nem kel fel.
6 | Otto hat Angst, den Bus nicht zu erreichen. | Otto mérges, hogy nem éri el a buszt. (erreichen)
7 | Das Kind hört nicht aufzuschreien. | A gyerek nem hallja, hogy kiáltozik. (aufschreien)
8 | Peter glaubt, schön zu sein. | Peter azt hisz, hogy szép.
9 | Er meint, die Prüfung zu bestehen. | Azt gondolja (f), hogy átmegy a vizsgát. (meinen, die Prüfung bestehen)
10 | Ildi hofft, den Zug zu erreichen. | Ildi reméli, hogy eléri a vonatot.
11 | Anna hofft, größer zu werden. | Anna reméli, hogy nagyobbá válik.
12 | Wir hoffen, noch ein Ticket zu bekommen. | Reméljük, hogy kapunk még egy jegyet. (s Ticket)
13 | Wir vergessen darüber nicht, morgen euch zu besuchen. | Nem feledkeztünk meg arról, hogy meglátogassunk holnap titeket. (darüber)
14 | Er denkt daran, übermorgen mich zu besuchen. | Arra gondol (f), hogy holnapután meglátogat engem. (denken, daran, übermorgen)
15 | Meine Freundin meint, immer Recht zu haben. | A barátnőm azt kondolja, hogy mindig igaza van. (meinen)
16 | Glaubst du, dein Geld in meiner Tasche zu finden? | Azt hiszed, hogy a pénzedet a táskámban találod?
17 | Anna verspricht immer, mich zu besuchen. | Anna mindig megígéri, hogy meglátogat. (versprechen)
\end{exmp}

\title{A zu + infinitiv szerkezet/A zu helye a mondatban}

GrammarItemIndex = 305655

\begin{desc}
1) Ha módbeli segédige van a mondatban, akkor a zu a főige és a segédige között áll.

Pl.: Ich freue mich, dich hier sehen zu können. - Örülök, hogy itt láthatlak.

2) Ha Perfektban áll a mellékmondat, akkor a zu a Partizip Perfekt és a segédige között áll.

Pl.: Irma behauptet, diesen Mann noch nie gesehen zu haben. - Irma azt állítja, hogy ezt a férfit még sosem látta.

3) Ha elválós az ige, akkor a zu az ige és az igekötő között áll (egybe írva).

Pl.: Ich habe jetzt keine Zeit fernzusehen. - Nincs időm tévét nézni.
\end{desc}

%Maklári: 308. oldal
\begin{exmp}
1 | Ich galube, morgen die Prüfung bestehen zu können. | Azt hiszem, hogy holnap át tudok menni a vizsgán. (e Prüfung bestehen)
2 | Er meint, mir helfen zu können. | Azt gondolja (f), hogy tud nekem segíteni. (meinen)
3 | Er freut sich morgen in das Theater gehen zu dürfen. | Örül (f) annak, hogy holnap színházba szabad mennie. (sich freuen)
4 | Anita glaubt, schon gut Auto fahren zu können. | Anita azt hiszi, hogy már jól tud autót vezetni.
5 | Wir freuen uns, jetzt in die Alpen fahren zu können. | Örülünk annak, hogy most az Alpokba utazhatunk. (sich freuen, die Alpen, fahren)
6 | Sie meint, den Aufsatz noch heute abgeben zu müssen. | Azt gondolja (n), hogy a tanulmányt még ma le kell adnia. (meinen, der Aufsatz, abgeben)
7 | Sie behauptet, das Geld verloren zu haben. | Azt állítja (n), hogy elvesztette a pénzt. (behaupten)
8 | Jörg erinnert sich, dieses Auto schon gesehen zu haben. | Jörg emlékszik, hogy látta már ezt az autót. (sich erinnern)
9 | Ich freue mich, meine Geldbörse gefunden zu haben. | Örülök, hogy megtaláltam a pénztárcámat. (sich freuen, e Geldbörse)
10 | Der Angeklagte leugnet, das Auto gestohlen zu haben. | A vádlott tagadja, hogy ellopta az autót. (r Angeklagte, leugnen, stehlen (Perfekt))
11 | Mein Freund behauptet, mich nie gesehen zu haben. | A barátom azt állítja, hogy engem soha nem látott. (behaupten)
12 | Ich freue mich, noch gestern sie angenruen zu haben. | Örülök, hogy még tegnap felhívtam őt (n). (sich freuen, anrufen)
13 | Ich glaube nicht, so früh aufzustehen. | Nem hiszem, hogy ilyen korán felkelek. (unmöglich)
14 | Er hofft, sein Haus noch heute verkaufen zu können. | Reméli (f), hogy a házát még ma el tudja adni. (verkaufen)
15 | Ich habe vor, alles abzuwaschen. | Azt tervezem, hogy mindent elmosok. (vorhaben, abwaschen)
16 | Wir haben daran gedacht, deine Briefe bei der Post einzuwerfen. | Arra gondoltunk, hogy a leveleidet a postánál bedobjuk. (daran, denken (Perfekt) , einwerfen)
17 | Er hat vor, meine Geschwister zu der Party einzuladen. | Azt tervezi (f), hogy a testvéreimet a partira meghívja. (vorhaben, zu der Party, einladen)
18 | Ich habe vor, morgen meine Lehrerin anzurufen. | Azt tervezem, hogy holnap a tanárnőmet felhívom. (vorhaben)
\end{exmp}

\title{A zu + infinitiv szerkezet/Scheinen zu + Infinitiv}

GrammarItemIndex = 537695

\begin{desc}
scheinen zu + Infinitiv = úgy tűnik, hogy

Pl.: Der Professor scheint dafür keine Lösung zu finden. - Úgy tűnik, hogy a professzor nem talál rá megoldást.
\end{desc}

%Maklári: 312. oldal
\begin{exmp}
1 | Dein Freund scheint heute auch nicht zu kommen. | Úgy tűnik, hogy ma sem jön a barátod.
2 | Ági scheint Kinder zu erwarten. | Úgy tűnik, hogy Ági gyereket vár. (Kinder erwarten)
3 | Er scheint von der Schlange sich nicht zu fürchten. | Úgy tűnik, hogy nem fél (f) a kígyótól. (e Schlange, sich fürchten)
4 | Deine Freundin scheint viel zu essen. | Úgy tűnik, hogy sokat eszik a barátnőd.
5 | Medizin scheint darauf nicht zu sein. | Úgy tűnik, hogy nincs orvosság rá. (s Medizin, darauf)
6 | Er scheint keine Freunde zu haben. | Úgy tűnik, hogy nincsenek barátai (f). (kein)
7 | Klára scheint sich gut nicht zu fühlen. | Úgy tűnik, hogy Klára nem érzi jól magát. (sich fühlen)
8 | Béla scheint die Orange zu lieben. | Úgy tűnik, hogy Béla szereti a narancsot. (lieben)
9 | Sie scheint die Banane nicht zu lieben. | Úgy tűnik, hogy nem szereti (n) a banánt. (lieben)
10 | Dein Freund scheint zu viel zu sprechen. | Úgy tűnik, hogy a barátod túl sokat beszél. (zu viel)
11 | Dein Zimmer scheint an den Bildern kleiner zu sein. | A képeken kisebbnek tűnt a szobád.
12 | Seine Worten scheinen kindisch zu sein. | Gyerekesnek tűntek a szavai (f). (kindlich)
13 | Sie scheint klug an dem Wettkampf nicht zu sein. | Nem tűnt (n) okosnak a versenyen. (an, r Wettkampf)
14 | Die Gäste scheinen Nachmittag schläfrig zu sein. | Ébéd után álmosnak tűntek a vendégek. (schläfrig)
15 | Dein Freund scheint traurig zu sein. | A barátod szomorúnak tűnik.
16 | Deine Freundin scheint betrunken zu sein. | A barátnőd részegnek tűnik. (betrunken sein)
17 | Seine Freundin scheint nett zu sein. | Kedvesnek tűnik a barátnője. (nett)
18 | Die Aufgabe scheint schwer zu sein. | Nehéznek tűnik a feladat.
19 | Dieser Fisch scheint frisch zu sein. | Ez a hal frissnek tűnik. (r Fisch)
20 | Dieses Fleisch scheint verdorben zu sein. | Ez a hús romlottnak tűnik. (s Fleisch)
\end{exmp}

\title{A zu + infinitiv szerkezet/Az ohne dass és az ohne zu + Infinitiv szerkezetek}

GrammarItemIndex = 565359

\begin{desc}
1) Ha a főmondat és a mellékmodat alanya megegyezik, akkor az ohne zu + Infinitiv (= anélkül, hogy) szerkezetet használjuk.

Pl.: Er geht von Hause weg, ohne die Tür zu schließen. - Elmegy otthonról, anélkül, hogy becsukná az ajtót.

2) Ha a főmondat és a mellékmodat alanya nem egyezik meg, akkor az ohne dass + KATI (= anélkül, hogy) szerkezetet használjuk.

Pl.: Er fährt das Auto von Péter, ohne dass es ihm Péter erlaubt. - Péter kocsijával jár, anélkül, hogy Péter megengedné.
\end{desc}

%Maklári: 316. oldal
\begin{exmp}
1 | Er tritt in mein Zimmer ein, ohne zu klopfen. | Belép (f) a szobámba anélkül, hogy kopogna. (eintreten, klopfen)
2 | Sie liest Bücher, ohne sie zu verstehen. | Könyveket olvas (n) anélkül, hogy megértené őket.
3 | Sie geht in das Büro ein, ohne von mir Abschieden zu nehmen. | Bemegy (n) az irodába anélkül, hogy elköszönne tőlem. (eingehen, Abschieden nehmen von + D)
4 | Er öffnet meine Briefe, ohne mich zu fragen. | Felnyitja a leveleimet anélkül, hogy megkérdezne engem.
5 | Er fährt Wagen, ohne Führerschein zu haben. | Kocsit vezet (f), anélkül, hogy lenne jogosítványa. (r Führerschein)
6 | Sie sieht meine Bilder, ohne etwas zu sagen. | Megnézi (n) a képeimet anélkül, hogy mondana valamit.
7 | Er geht vor 5 Uhr nach Hause, ohne den Chef zu fragen. | 5 óra előtt haza megy (f) anélkül, hogy megkérdezné a főnököt.
8 | Ich komme immer genau an, ohne Armbanduhr zu haben. | Mindig pontosan érkezem anélkül, hogy lenne karórám. (ankommen, genau, e Armbanduhr)
9 | Ich gebe ihr ein Küschen, ohne dass sie erlaubt. | Adok neki (n) egy puszit anélkül, hogy megengedné. (s Küschen, erlauben)
10 | Ich gehe vor ihrem Haus weg, ohne dass sie mich bemerkt. | Elmegyek a háza (n) előtt anélkül, hogy észrevenne engem. (weggehen, bemerken)
11 | Sein Fernseher geht ganzer Tag, ohne dass jemand sieht. | Megy a tévéje (f) egész nap anélkül, hogy valaki nézné. (gehen, jemand)
12 | Ich repariere mein Fahrrad, ohne dass Karl mir hilft. | Megjavítom a kerékpáromat anélkül, hogy Karl segítene nekem.
13 | Er bringt meine Bücher nach Hause, ohne dass ich ihm erlaube. | Haza viszi (f) a könyveimet anélkül, hogy megengedném neki. (bringen, erlauben)
14 | Das Radio geht, ohne dass jemand hört. | Megy a rádió anélkül, hogy valaki hallgatná. (jemand)
15 | Sie kommen zu mir, ohne dass ich ihnen erlaube. | Hozzám jönnek, anélül hogy megengedném nekik. (erlauben)
\end{exmp}

\title{Néhány Akkusativ-os és Dativ-os vonzatú ige/A gefallen + D}

GrammarItemIndex = 861276

\begin{desc}
gefallen + D (ä) = tetszik vkinek vmi

gefallen ragozása:
ich       | gefalle
du        | gefällst
er/sie/es | gefällt
wir       | gefallen
ihr       | gefallt
sie/Sie   | gefallen

Pl.: Das Heft gefällt dem Kind. - A füzet tetszik a gyereknek.
\end{desc}

%Maklári: 33. oldal
\begin{exmp}
1 | Ich gefalle dem Mädchen. | Tetszem a lánynak.
2 | Das Mädchen gefällt mir. | A lány tetszik nekem.
3 | Der Lehrer gefällt den Eltern. | A szülőknek tetszik a tanár.
4 | Die Eltern gefallen dem Lehrer. | A tanárnak tetszenek a szülők.
5 | Der Schauspieler gefällt dem Publikum. | A közönségnek tetszik a színész.
6 | Das Publikum gefällt dem Schauspieler. | A színésznek tetszik a közönség.
7 | Meine Börse gefällt ihm. | Tetszik neki (f) a pénztárcám.
8 | Sie gefällt mir nicht. | Nekem nem tetszik ő (n).
9 | Ich gefalle ihr nicht. | Nem tetszem neki (n).
10 | Ich gefalle deiner Schwiegermutter nicht. | Nem tetszem az anyósodnak.
11 | Die Schwiegermutter gefällt mir nicht. | Az anyós nem tetszik nekem.
12 | Das Bild gefällt den Gästen. | A kép tetszik a vendégeknek.
\end{exmp}

\title{Néhány Akkusativ-os és Dativ-os vonzatú ige/A gehören + D}

GrammarItemIndex = 637854

\begin{desc}
gehören + D = valakié, valakinek a tulajdona

Pl.: Das Fahrrad gehört dem Kind.
\end{desc}

%Maklári: 33. oldal
\begin{exmp}
1 | Der Rock gehört dem Mädchen. | A szoknya a lányé.
2 | Die Bücher gehören dem Lehrer. | A könyvek a tanáré.
3 | Die Tasche gehört der Frau. | A táska a nőé.
4 | Der Kuli gehört dem Schüler. | A toll a diáké.
5 | Das Auto gehört dem Polizist. | Az autó a rendőré.
6 | Die Mütze gehört dem Vater. | A sapka az apáé.
7 | Der Kühlschrank gehört der Müller Familie. | A hűtőszekrény a Müller családé.
8 | Der Wagen gehört dem Chef. | A kocsi a főnöké.
9 | Die Garage gehört dem Nachbar. | A garázs a szomszédé.
10 | Der Kuli gehört uns. | A toll a miénk.
11 | Der Wagen gehört dem Lehrer. | A kocsi a tanáré.
12 | Der Wolkenkratzer gehört der Firma. | A felhőkarcoló a cégé.
\end{exmp}

\title{Néhány Akkusativ-os és Dativ-os vonzatú ige/A gehören zu + D}

GrammarItemIndex = 863524

\begin{desc}
gehören zu + D = tartozik vkihez/vmihez

Pl.: Peter gehört auch zur Gesellschaft. - Péter is a társasághoz tartozik.
\end{desc}

%Maklári: 33.oldal
\begin{exmp}
1 | Gehörst du auch zu uns? | Te is hozzánk tartozol?
2 | Diese Frage gehört nicht zu dem Thema. | Ez a kérdés nem tartozik a témához. (s Thema)
3 | Ivan gehört zu meinen Feinden. | Ivan az ellenségeimhez tartozik. (r Feind)
4 | Wer gehört zu dieser Gesellschaft? | Ki tartozik ehhez a társasághoz?
5 | Joe gehört zu euch. | Joe hozzátok tartozik.
6 | Vien gehört zu Österreich. | Bécs Ausztriához tartozik.
7 | Es gehört nicht zu der Sache. | Ez nem tartozik a tárgyhoz. (es, e Sache)
8 | Deutschland gehört zu Europa. | Németország Europához tartozik.
9 | Bill gehört zu den Terroristen. | Bill a terroristákhoz tartozik. (r Terrorist)
10 | Ich gehöre nicht zu dieser Partei. | Nem tartozom ehhez a párthoz. (e Partei)
\end{exmp}

\title{A lassen ige/Önállóan, főigeként}

GrammarItemIndex = 861386

\begin{desc}
ich       | lasse
du        | lässt
er/sie/se | lässt
wir       | lassen
ihr       | lasst
sie/Sie   | lassen

Pl.: Ivan lässt mich nie in Ruhe. - Iván sose hagy békén.
\end{desc}

%Maklári: 215. oldal
\begin{exmp}
1 | Ich lasse die Uhr auf den Tisch. | Az órát az asztalon hagyom.
2 | Er lässt den Wagen bei dem Mechaniker. | A kocsit a szerelőnél hagyja (f).
3 | Ich lasse bei euch mein Radio. | Nálatok hagyom a rádiómat.
4 | Lässt du mich in Ruhe? | Békén hagysz? (in Ruhe)
5 | Warum lässt du ihn im Stich? | Miért hagyod cserben (f)? (im Stich)
6 | Ich lasse dich nicht im Stich. | Nem hagylak cserben. (im Stich)
7 | Er lässt mich nicht in Ruhe. | Nem hagy (f) békén. (in Ruhe)
8 | Wo lässt du das Fahrrad? | Hol hagyod a kerékpárt?
9 | Lässt du bei mir den Wagen? | Nálam hagyod a kocsit?
10 | Sie lässt ohne Geld hier mich. | Pénz nélkül (n) hagy itt engem.
11 | Józsi lässt sein Heft zu Hause. | Józsi a füzetét otthon hagyja.
12 | Ihr lasst immer zu Hause Béla. | Mindig otthon hagyjátok Bélát.
13 | Sie lässt uns im Stich. | Cserben hagy (n) minket. (im Stich)
14 | Ich lasse den Wagen bei dem Mechaniker. | A kocsit a szerelőnél hagyom.
15 | Ich lasse bei dir das Kind. | Nálad hagyom a gyereket.
16 | Lasst ihr meinen Freund in Ruhe? | Békén hagyjátok a barátomat? (in Ruhe)
\end{exmp}

\title{A lassen ige/Segédigeként}

GrammarItemIndex = 132451

\begin{desc}
Képzése:
alany + lassen ragozva + többi mondatrész + ige Infinitivben

ich       | lasse
du        | lässt
er/sie/se | lässt
wir       | lassen
ihr       | lasst
sie/Sie   | lassen

Pl.: Ich lasse dich ins Kino gehen. - Hagylak moziba menni.
\end{desc}

%Maklári: 215. oldal
\begin{exmp}
1 | Lässt sie uns spielen? | Hagy (n) minket játszani?
2 | Lasst ihr hier uns Fußball spielen? | Engedtek itt focizni minket?
3 | Ich lasse ihm mit meinem Fahrrad fahren. | Megengedem neki (f), hogy kerékpározzon a kerékpárommal.
4 | Lasst ihr uns hier baden? | Megengeditek, hogy itt fürödjünk?
5 | Ich lasse dir neben mir sitzen. | Megengedem, hogy mellém ülj.
6 | Lasst ihr sie nicht zu Hause lernen? | Nem hagyjátok őt (n) tanulni otthon?
7 | Lässt sie dir Eis essen? | Megengedi (n), hogy fagyit egyél?
8 | Lasst ihr uns eintreten? | Megengeditek, hogy belépjünk? (eintreten)
9 | Er lässt auf der Stunde nicht schlafen. | Nem enged (f) az órán aludni.
10 | Sie lässt mich nicht rauchen. | Nem enged (n) dohányozni.
11 | Wo lasst ihr uns Basketball spielen? | Hol engedtek minket kosárlabdázni?
12 | Lässt du ein klein fahren? | Megengeded, hogy kicsit vezessek?
13 | Lässt du mir sehen? | Megengeded, hogy megnézzem?
14 | Er lässt seinen Hund in unsere Garten essen. | A kutyáját (f) a kertünkbe engedi enni.
15 | Lasst ihr mir reparieren? | Megengeditek, hogy megjavítsam?
16 | Ich lasse dir mein Kleid anziehen. | Megengedem, hogy felvedd a ruhámat. (anziehen)
17 | Sie lassen uns hier nicht schlafen. | Nem engedik meg, hogy itt aludjunk.
18 | Lässt du die Suppe aufkochen? | Hagyod a levest felforrni? (aufkochen)
19 | Warum lässt du ihn nicht schlafen? | Miért nem engeded őt (f) aludni?
20 | Ich lasse ihn Alkohol nicht trinken. | Nem engedem őt (f) alkoholt inni.
21 | Lässt du nicht in der Schule gehen? | Nem engeded iskolába menni?
\end{exmp}

\title{A lassen ige/A lassen mint műveltető}

GrammarItemIndex = 136545

\begin{desc}
1) Képzése: alany + lassen ragozva + többi mondatrész + ige Infinitivben 

Pl.: Ich lasse morgen das Radio reparieren. - Megjavíttatom holnap a rádiót.

2) A lassen ige vonzata eltér a magyartól, ugyanis tárgyesetbe tesszük
azt, akivel csináltatunk valamit.

Pl.: Péter lässt seine Freundin das Auto waschen. - Péter a barátnőjével mosatja az autót.

lassen ragozása:
ich lasse
du lässt
er/sie/es lässt
wir lassen
ihr lasst
sie/Sie lassen
\end{desc}

%Maklári: 219. oldal
\begin{exmp}
1 | Ich lasse meinen Wagen reparieren. | Megjavíttatom a kocsimat.
2 | Ihr lasst ihm zwei Torten backen. | Süttetnek neki (f) két tortát. (backen)
3 | Ich lasse mich fotografieren. | Lefotóztatom magamat. (fotografieren)
4 | Sie lassen euch ein Hause bilden. | Építtettek (ők) nektek egy házat.
5 | Wir lassen die Koffer bringen. | Elhozatjuk a bőröndöket. (r Koffer)
6 | Wir lassen uns Kleid machen. | Csináltatunk magunknak ruhát.
7 | Du lässt den Teppich putzen. | Kitisztíttatod a szőnyeget. (r Teppich, putzen)
8 | Lässt du dir Bier bringen? | Hozatsz magadnak sört? (bringen)
9 | Ich lasse meine Wohnung malen. | Kifestem a lakásomat. (malen)
10 | Béla lässt ein Haus renovieren. | Béla renováltat egy házat. (renovieren)
11 | Ich lasse Józsi das Fenster waschen. | Józsival lemosatom az ablakot.
12 | Ich lasse Péter meine Lektion schreiben. | Péterrel megíratom a leckémet. (e Lektion)
13 | Er lässt den Dolmetscher zwei Briefe übersetzen. | A tolmáccsal két levelet fordíttat (f). (r Dolmetscher, übersetzen)
14 | Warum lässt du mich deine Briefe schreiben? | Miért velem íratod a leveleidet?
15 | Sie lässt seine Freundin einen Arzt rufen. | A barátnőjével hívat (f) egy orvost.
16 | Sie lässt den Nachbar das Haus renovieren. | A szomszéddal renováltatja (n) a házát. (renovieren)
17 | Der Lehrer lässt den Schüler die Tür öffnen. | A tanár a diákkal kinyittatja az ajtót.
18 | Ich lasse meine Schwester mich fotografieren. | A húgommal fotóztatom le magamat. (e Schwester)
19 | Du lässt den Monteur das Telefon montieren. | Megszerelteted a szerelővel a telefont. (montieren, r Monteur)
20 | Du lässt deinen Freund deinen Hund waschen. | Megmosatod a barátoddal a kutyádat.
\end{exmp}

\title{Határozatlan névmások/A jeder/e/es}

GrammarItemIndex = 583325

\begin{desc}
jeder/e/es = mindegyik, minden egyes

A főnév, amire a jeder/e/es vonatkozik egyesszámban áll. A jede szót ragozni kell azon főnév esete száma szerint, amire vonatkozik.

  | HÍMNEM | NŐNEM | SEMLEGES NEM
--+--------+-------+--------------
A | jeder  | jede  | jedes
N | jeden  | jede  | jedes
D | jedem  | jeder | jedem
G | jedes  | jeder | jedes

Pl.: * Jeder Mann ist anders - Minden férfi más.
* Jede Frau möchte hübsch sein. - Minden nő szeretne csinos lenni.
* Jedes Kind mag spielen. - Minden gyerek szeret játszani.
\end{desc}

%Maklári: 179. oldal
\begin{exmp}
1 | Wir geben jeder Frau Blume. | Minden nőnek adunk virágot.
2 | Jedes Auto gefällt ihm. | Minden autó tetszik neki (f).
3 | Wir geben jedem Gast Tee. | Minden vendégnek adunk teát.
4 | Jeder Schüler ist in der Schule. | Minden tanuló az iskolában van.
5 | Sie hilft jedem Kind. | Minden gyereknek segít (n).
6 | Ich decke jeden Tisch. | Minden asztalt megterítek.
7 | Er hat in jedem Land eine Frau. | Minden országban van (f) egy felesége.
8 | Es gibt in jeder Wohnung ein Badezimmer. | Minden lakásban van egy fürdőszoba. (es gibt, s Badezimmer)
9 | Jedes Buch kostet fünf Euro. | Minden könyv öt euróba kerül. (kostet + A, r Euro)
10 | Vater repariert morgen jeden Tisch. | Apa holnap minden aszalt megjavít.
11 | Er spricht mit jedem Schüler in der Schule. | Minden tanulóval beszél (f) az iskolában.
12 | Sie streichelt jeden Hund. | Minden kutyát megsimogat (n). (streicheln)
\end{exmp}

\title{Határozatlan névmások/Alle és variánsai/Alle}

GrammarItemIndex = 764195

\begin{desc}
alle = minden, az egész, az összes

Az alle után a főnév mindig többesszámban van. A határozott névelő többesszámú végződéseit veheti fel:
N -> alle;
A -> alle;
D -> allen;
A -> aller.

Pl.: Alle Kinder schlafen schon. - Minden gyerek hamar elalszik.
\end{desc}

%Maklári: 180. oldal
\begin{exmp}
1 | Alle Bücher legen auf dem Boden. | Az összes könyv a földön fekszik.
2 | Wir korrigieren alle Fehler. | Minden hibát kijavítunk. (korrigieren)
3 | Alle Schularbeit sind schlecht. | Az összes dolgozat rossz. (e Schularbeit, schlecht)
4 | Er möchte alle Tiere sehen. | Minden állatot szeretne (f) látni.
5 | Alle Autos sind in der Garage. | Minden autó a garázsban van.
6 | Sie erzählt allen Menschen das. | Minden embernek elmeséli (n) azt. (r Mensch)
7 | Er gibt allen Kollegen etwas. | Minden kollégának ad (f) valamit.
8 | Alle Koffer sind schon in dem Wagen. | Minden bőrönd már a kocsiban van. (r Koffer)
9 | Alle Gäste werden mit Apfel angeboten. | Minden vendéget megkínálnak almával. (anbieten)
10 | Der Vater aller Kinder kommt morgen. | Minden gyerek apja jön holnap.
11 | Alle Fenster sind geöffnet. | Minden ablak nyitva van.
12 | Meine Freundin ruft alle fünf Minuten mir an. | A barátnőm minden öt percben felhív engem.
13 | Der Bus fährt alle zehn Minuten. | A busz tíz percenként jár. (fahren)
14 | Wir besuchen sie alle drei Jahre. | Háromévenként meglátogatjuk őket.
\end{exmp}

\title{Határozatlan névmások/Alle és variánsai/Alles}

GrammarItemIndex = 321953

\begin{desc}
alles = dolgok összessége

Pl.: * Alles ist in ordnung. - Minden rendben van.
* Alles klar. - Minden világos.
\end{desc}

%Maklári: 181. oldal
\begin{exmp}
1 | Kauft Kati alles ein? | Mindent bevásárol Kati?
2 | Alles kostet viel. | Minden sokba kerül.
3 | Péter erledigt alles. | Péter mindent elintéz. (erledigen)
4 | Alles oder nichts! | Mindent vagy semmit!
5 | Ist alles fertig? | Minden készen van?
6 | Sie spricht von alles morgen. | Mindenről beszél (n) holnap.
7 | Wir sind mit alles fertig. | Mindennel készen vagyunk.
8 | Sie denkt alles. | Mindenre gondol (n).
9 | Meine Torte schmeckt ihr über alles. | Mindennél jobban ízlik neki (n) a tortám. (schmecken, über alles)
10 | Ist es alles? | Ez minden?
11 | Verkaufst du alles? | Mindent eladsz?
12 | Alles ist interessant. | Minden érdekes.
13 | Es gibt in dem Wagen schon alles. | A kocsiban van már minden. (es gibt)
14 | Er sucht in alles etwas schlecht. | Mindenben keres (f) valami rosszat.
15 | Dieser Mensch spricht immer alles. | Ez az ember mindig mindent beszél. (r Mensch)
16 | Es ist fast alles teuer. | Majdnem minden drága. (es, fast)
17 | Sie kämpft gegen alles. | Minden ellen harcol (n).
18 | Er sagte alles derzeit. | Mindent mondott (f) akkor. (derzeit)
19 | Dieser Mensch ist fähig zu alles. | Ez az ember mindenre képes. (r Mensch, fähig sein zu + D)
20 | Es gibt alles hier. | Minden van itt. (es gibt)
\end{exmp}

\title{Határozatlan névmások/Alle és variánsai/Etwas ist alle}

GrammarItemIndex = 325596

\begin{desc}
etwas ist alle = elfogyott, kifogyott

Pl.: Mein Geld ist alle. - Elfogyott a pénzem.
\end{desc}

%Maklári: 180. oldal, 181. oldal
\begin{exmp}
1 | Mein Geld ist alle. | Elfogyott a pénzem.
2 | Die Zeitungen sind alle. | Elfogytak az újságok.
3 | Die Karten sind alle. | Elfogytak a jegyek. (e Karte)
4 | Das Benzin ist alle. | Elfogyott a benzin. (s Benzin)
5 | Die Kartoffel ist alle. | Elfogyott a krumpli. (e Kartoffel)
6 | Die Fahrräder sind alle. | Elfogytak a kerékpárok.
7 | Das Fleisch ist alle. | Elfogyott a hús.
8 | Die Bilder sind alle. | Elfogytak a képek.
9 | Das Brot ist alle. | Elfogyott a kenyér.
10 | Das Mineralwasser ist alle. | Elfogyott az ásványvíz.
\end{exmp}

\title{Határozatlan névmások/Irgend-}

GrammarItemIndex = 234596

\begin{desc}
* irgendwer = valaki
* irgendwas = valami
* irgendwann = valamikor
* irgendwoher = valahonnan
* irgendwohin = valahová
* irgendwie = valahogy
* irgendwo = valahol
* irgendeinmal = egyszer majd

Pl.: Irgendwer wartet auf dich im Zimmer. - Valaki vár rád a szobában.
\end{desc}

%Maklári: 183. oldal
\begin{exmp}
1 | Ich wolle dir irgendwas sagen. | Akarok neked valamit mondani.
2 | Es muss irgendwo sein! | Valahol meg kell lennie! (es)
3 | Es gibt irgendwas in seinem Mund. | Valami van a szájában (f). (es gibt, r Mund)
4 | Sie gehen irgendwohin morgen weg. | Valahová elmennek (ők) holnap. (weggehen)
5 | Wir gewinnen irgendeinmal das Lotto. | Mi is megnyerjük egyszer majd a lottót. (gewinnen, s Lotto)
6 | Hilfe kommt irgendwoher. | Valahonnan jön segítség. (e Hilfe)
7 | Wir müssen irgendwo tanken. | Valahol kell tankolnunk. (tanken)
8 | Ich rufe dich irgendwann an. | Felhívlak valamikor.
9 | Wir müssen irgendwann weggehen. | Valamikor el kell mennünk. (weggehen)
10 | Es ist irgendwas in deiner Tasche. | Valami van a zsebedben. (es, e Tasche)
11 | Irgendwer kommt morgen zu uns. | Valaki jön holnap hozzánk.
12 | Der Gefangene ist irgendwie verschwunden. | Valahogy eltűnt a fogoly. (r Gefangene, i. verschwunden)
13 | Er schafft irgendwoher eine Waffe. | Valahonnan szerzett (f) egy fegyvert. (schaffen)
14 | Es gibt irgendwo jetzt Krieg. | Valahol most háború van. (es gibt)
\end{exmp}

\title{Melléknévi igenevek/Folyamatos melléknévi igenevek}

GrammarItemIndex = 321596

\begin{desc}
1) Módhatározóként: főnévi igenév + d

Pl.: * stehend - állva
* lesend - olvasva

2) Képzőként: főnévi igenév + d + melléknév végződés

Pl.: * Das sprechende Frau da ist meine Frau. - A beszélő nő ott a feleségem.
* die sitzende Frau - az ülő nő
* der stehende Mann - az álló férfi
* mit dem sprechenden Kind - a beszélő gyerekkel
\end{desc}

%Maklári: 360. oldal
\begin{exmp}
1 | stehend | állva
2 | lesend | olvasva
3 | sitzend | ülve
4 | laufend | futva
5 | liegend | feküdve
6 | träumend | álmodva
7 | schreibend | írva
8 | kämpfend | harcolva
9 | lesed | olvasva
10 | spielend | játszva
11 | pfeifend | fütyülve (pfeifen)
12 | lernend | tanulva
13 | arbeitend | dolgozva
14 | stotternd | dadogva (stottern)
15 | die sitzende Frau | az ülő nő
16 | der stehende Mann | az álló férfi
17 | mit dem sprechenden Kind | a beszélő gyerekkel
18 | der schlafende Lehrer | az alvó tanár
19 | die schriebende Mutter | az író anya
20 | das lerenende Kind | a tanuló gyerek
21 | der pfeifende Schüler | a fütyülő diák (pfeifen)
22 | das trinkende Mädchen | az ivó lány
23 | das essende Kind | az evő gyerek
24 | der schwimmende Mann | az evő férfi
25 | die schreiende Frau | a kiabáló nő
26 | der sprechende Papagei | a beszélő papagáj (r Papagei)
\end{exmp}

\title{Melléknévi igenevek/Befelyezett melléknévi igenevek}

GrammarItemIndex = 126358

\begin{desc}
Jelzőként: Partizip Perfekt + melléknév végződés

Pl.: * die geöffnete Tür - a kinyitott ajtó
* das gekaufte Auto - a megvásárolt autó
\end{desc}

%Maklári: 362. oldal
\begin{exmp}
1 | die geöffnete Tür | a kinyitott ajtó
2 | das gekaufte Auto | a megvásárolt autó
3 | die geschriebene Hausaufgabe | a megírt házifeladat
4 | die gekochte Suppe | a megfőzött leves
5 | der geprügelte Hund | a megvert kutya (prügeln)
6 | der reparierte Wagen | a megszerelt kocsi (reparieren)
7 | das restaurierte Haus | a restaurált ház (restaurieren)
8 | das gelesene Buch | az elolvasott könyv
9 | die getrunkene Milch | a megivott tej
10 | der gedeckte Tisch | a megterített asztal
11 | der eingeladene Gast | a meghívott vendég
\end{exmp}

\title{Melléknévi igenevek/Beálló melléknévi igenevek}

GrammarItemIndex = 536756

\begin{desc}
Képzése: zu + főnévi igenév + d + melléknév végződés

Pl.: * der zu reparierende Kühlschrank - a megjavítandó hűtőszekrény
* ein zu lösendes Problem - a megoldandó probléme
\end{desc}

%Maklári: 366. oldal
\begin{exmp}
1 | ein zu reparierendes Radio | egy megjavítandó rádió
2 | die zu lernende Wörter | a megtanulandó szavak
3 | die zu lösende Aufgaben | a megoldandó feladatok
4 | die zu anerkennende Leistung | az elismerendő teljesítmény (anerkennen, e Leistung)
5 | die zu korrigierende Arbeiten | a kijavítandó munkák (korrigieren)
6 | ein zu lesendes Buch | egy elolvasandó könyv
7 | die zu waschende Kleider | a kimosandó ruhák
8 | die zu lernende Materie | a megtanulandó anyag (e Materie)
9 | ein zu wiedergutmachender Fehler | egy jóváteendő hiba (wiedergutmachen, r Fehler)
10 | ein zu vermeidender Fehler | egy elkerülendő hiba (vermeiden, r Fehler)
11 | das zu nutztende Wörterbuch | a használandó szótár (nutzen)
12 | das zu schreibende Brief | a megírandó levél
\end{exmp}

\title{Az idő/Az idő kérdőszavai}

GrammarItemIndex = 541543

\begin{desc}
* wann? = mikor?
* seit wann? = mióta?
* bis wann? = meddig?
* wie lange? = mennyi ideig?
* wie oft? = milyen gyakran?
* ab wann? = mikortól?

Pl.: Seit wann lebst du hier? - Mióta élsz itt?
\end{desc}

%Maklári: 74. oldal
\begin{exmp}
1 | Bis wann dauert das Konzert? | Meddig tart a koncert? (dauern)
2 | Wie lange dauert das Konzert? | Mennyi ideig tart a koncert? (dauern)
3 | Seit wann wartet ihr auf uns? | Mióta vártok ránk?
4 | Wie oft geht ihr in Theater? | Milyen gyakran mentek színházba?
5 | Wann kann Man hier einkaufen? | Mikor tud az ember itt bevásárolni? (man, einkaufen)
6 | Bis wann bleibt ihr in New York? | Meddig maradtok New Yorkban?
7 | We oft fahrt ihr nach Ausland? | Milyen gyakran utaztok külföldre? (s Ausland)
8 | Seit wann wohnt ihr hier? | Mióta laktok itt?
9 | Bis wann ist das Museum geöffnet? | Meddig van a múzeum nyitva?
10 | Wie lange dauert die Reise nach Asien? | Meddig tart az út ázsiába? (dauern, e Reise, Asien)
11 | Seit wann stehen diese Häuser hier? | Mióta állnak itt ezek a házak?
12 | Seit wann ist dieses Geschäft geschlossen? | Mióta van ez az üzlet zárva? (schließen)
13 | Ab wann ist das Abonnement gültig? | Mikortól érvényes a bérlet? (s Abonnement, gültig)
14 | Wann kommt ihr wieder zu uns? | Mikor jöttök újra hozzánk? (wieder)
15 | Bis wann dauert der Film? | Meddig tart a film? (dauern)
16 | Ab wann geht dein Sohn in den Kindergarren? | Mikortól megy a fiad az óvodába? (r Kindergarten)
17 | Seit wann wartet ihr auf den Bus? | Mióta vártok a buszra?
18 | Wie oft gehst du in das Schwimmbad? | Milyen gyakran mész uszodába? (s Schwimmbad)
19 | Wie lange spricht er noch? | Mennyi ideig beszél (f) még?
20 | Wie lange kochst du das Fleisch? | Mennyi ideig főzöd a húst? (kochen)
21 | Wann schließen die Geschäfte? | Mikor zárnak az üzletek? (schließen)
22 | Seit wann trinkst du den Kaffee mit Milch? | Mióta iszod a kávét tejjel? (r Kaffee, e Milch) 
23 | Ab wann geht Pista in die Schule? | Mikortól megy Pista iskolába?
24 | Wie oft besuchst du deine Großmutter? | Milyen gyakran látogatod meg a nagymamádat?
\end{exmp}

\title{Az idő/Az idő kérdőszavai/Elöljárószavak/Akkusativ-val}

GrammarItemIndex = 564862

\begin{desc}
* um + A = -kor
* gegen + A = körül
* bis + A = -ig
* für + A = -ra, -re
* über + A = -ra, re (időtartam)
* über + A = több mint

Pl.: * um zwei - kettőkor
* gegen fünf - öt körül
* bis 11 - 11-ig
* für morgen Abend - holnap estére
* über zwei Wochen - két hétre
* über 90 Jahre alt - 90 év fölött
\end{desc}

%Maklári: 76. oldal
\begin{exmp}
1 | Mein Freund kommt gegen fünf. | A barátom öt körül jön.
2 | Ihr könnt bis 3 Uhr Handball spielen. | 3 óráig tudtok kézilabdázni.
3 | Ich lade die Gäste für heute Abend ein. | Meghívom a vendégeket ma estére. (einladen)
4 | Ich beende für Mittwoch die Arbeit. | Szerdára befelyezem a munkát. (beenden)
5 | Die Gäste besuchen uns übermorgen gegen elf. | A vendégek meglátogatnak minket holnapután tizenegy körül.
6 | Ich war vorgestern bis 12 in dem Kino. | Tegnapelőtt 12-ig moziban voltam. (vorgestern)
7 | Sie kommen nur für ein paar Minuten. | Ők csak néhány percre jönnek. (ein paar) 
8 | Er ist über 50. | Ő (f) több mint 50.
9 | Ich habe bei ihnen über 28 Jahre gearbeitet. | Náluk több mit 28 évet dolgoztam. (arbeiten (Perfekt))
10 | Ich beende für morgen diese Arbeit. | Befejezem holnapra ezt a munkát. (beenden)
11 | Er bleibt über 3 Jahre in Budapest. | 3 évig marad (f) Budapesten.
12 | Unsere Nachbarn verreisen für zwei Wochen. | A szomszédaink elutaznak két hétre. (verreisen)
13 | Wir werden bis Montag mit der Arbeit fertig sein. | Hétfőig a munkával készen leszünk.
14 | Ich lade für morgen meinen Freund ein. | Holnapra meghívom a barátomat. (einladen)
15 | Sie fahren für das Wochenende nach Österreich. | A hétvégére Ausztiába utaznak. (fahren, s Wochenende)
16 | Wir renovieren bis Mitte Oktober das Haus. | Október közepéig felújítjuk a házat. (renovieren, Mitte Oktober)
17 | Sie werden gegen Abend zu Hause. | Este körül otthon lesznek.
18 | Die Müller Familie verreist für Weihnacht. | A Müller család elutazik karácsonyra. (verreisen, e Weihnacht)
19 | Ich gehe über eine Stunde zu euch. | Egy órára megyek hozzátok.
\end{exmp}

\title{Az idő/Az idő kérdőszavai/Elöljárószavak/Dativ}

GrammarItemIndex = 456238

\begin{desc}
* seit + D = óta
* vor  + D = előtt
* zwischen + D = között
* in + D múlva
* von +D ... an = -tól, -től
* ab + D = -tól, től

Pl.: * seit 5 Jahren - 5 éve
* vor 2 Monaten - 2 hónappal ezelőtt
* zwischen 4 und 5 - 4 és 5 között
* in einer Woche - egy hét múlva
* von nächster Woche an - a következő héttől
* ab Mitte Mai - május közepétől
\end{desc}

%Maklári: 78. oldal
\begin{exmp}
1 | Sie hat vor zwei Tagen geboren. | Két nappal ezelőtt született (n).
2 | Ich bin seit zwei Tagen zu Hause. | Két napja itthon vagyok.
3 | Ich werde in einer Jahr fünfzehn Jahre alt sein. | Egy év múlva tizenöt éves leszek.
4 | Sie kommen vor fünf mit zehn Minuten nach Hause an. | Öt előtt tíz perccel érkeznek haza. (ankommen)
5 | Ich warte seit fünf Minuten auf dich. | Öt perce várok rád. (warten auf + A)
6 | Was macht ihr ab Montag bis Mittwoch? | Hétfőtől szerdáig mit csináltok?
7 | Wir freuen uns in fünf Wochen deinen Geburtstag. | Öt hét múlva ünnepeljük a szülinapodat. (sich freuen)
8 | Ich esse seit zwei Tagen nicht. | Két napja nem eszek.
9 | Ich aß vor zwei Tagen nichts. | Két nappal ezelőtt nem ettem semmit.
10 | Die Gäste kommen zwischen 4 und 5. | 4 és 5 között jönnek a vendégek.
11 | Sie trinkt seit 10 Jahren Wasser nicht. | 10 év óta nem iszik vizet. (n)
12 | Ich komme in 5 Minuten zurück. | 5 perc múlva visszajövök.
13 | Er arbeitet seit 25 Jahren hier. | 25 éve itt dolgozik (f).
14 | Er heiratet in zwei Jahren. | Két év múlva megházasodik (f). (heiraten)
15 | Es gibt zwischen 8 un 10 Uhr Wasser nicht. | 8 és 10 óra között nincs víz. (es gibt)
16 | Ági war vor zwei Wochen hier. | Két hete itt volt Ági.
17 | Ich werde in einer Jahr volljährig. | Egy év múlva nagykorú leszek. (volljährig)
18 | Sie raucht seit vier Jahren nicht. | Négy év óta nem dohányzik (n).
19 | Es ist ab morgen Schule nicht. | Holnaptól nincs iskola. (es ist)
20 | Ich habe seit einer Ewigkeit dich nicht gesehen. | Egy örökkévalósága óta nem láttalak. (e Ewigkeit)
21 | Dein Bus fährt in zwei Minuten ab. | Két perc múlva indul a buszod. (abfahren)
22 | Ich lerne ab Mitte August dort. | Augusztus közepétől ott tanulok.
23 | Ich werde in drei Monaten 16 Jahre alt sein. | Három hónap múlva 16 éves leszek.
24 | Meine Cousine wohnt seit 8 Jahren hier. | 8 éve lakik itt az unokatestvérem (n). (e Cousine)
\end{exmp}

